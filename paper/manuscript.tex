\documentclass[11pt]{article}

\usepackage[margin=1in]{geometry}
\usepackage{times}
\usepackage{graphicx}
\usepackage{amsmath,amssymb}
\usepackage{booktabs}
\usepackage{xcolor}
\usepackage{hyperref}
\usepackage[numbers,sort&compress]{natbib}
\usepackage{float}
\usepackage{caption}
\usepackage{fancyhdr}

\definecolor{cellblue}{RGB}{0,51,102}

\makeatletter
\renewcommand{\section}{\@startsection{section}{1}{\z@}{-3.5ex \@plus -1ex \@minus -.2ex}{2.3ex \@plus.2ex}{\large\bfseries}}
\renewcommand{\subsection}{\@startsection{subsection}{2}{\z@}{-3.25ex\@plus -1ex \@minus -.2ex}{1.5ex \@plus .2ex}{\normalsize\bfseries}}
\renewcommand{\subsubsection}{\@startsection{subsubsection}{3}{\z@}{-3.25ex\@plus -1ex \@minus -.2ex}{1.5ex \@plus .2ex}{\normalsize\itshape}}
\makeatother

\captionsetup{font=small,labelfont=bf,labelsep=period}

\pagestyle{fancy}
\fancyhf{}
\fancyhead[L]{\small Whiteside et al.}
\fancyhead[R]{\small PAR(2) Circadian Gating Framework}
\fancyfoot[C]{\thepage}
\renewcommand{\headrulewidth}{0.4pt}

\hypersetup{
    colorlinks=true,
    linkcolor=cellblue,
    citecolor=cellblue,
    urlcolor=cellblue
}

\begin{document}

\begin{center}

{\Large\bfseries A Phase-Gated AR(2) Framework Reveals Circadian ``Gatekeeper Switching'' During Colorectal Cancer Progression in Intestinal Organoids}

\vspace{1.5em}

{\large Michael Whiteside$^{1,*}$}

\vspace{0.5em}

{\small $^1$Independent Researcher, Scotland}

\vspace{0.3em}

{\small $^*$Correspondence: [email redacted for review]}

\vspace{1em}

{\small\textit{Submitted to Cell --- December 2025 (revised)}}

\end{center}

\vspace{1.5em}

\noindent\fbox{
\begin{minipage}{0.97\textwidth}
\textbf{Highlights}
\begin{itemize}
\setlength\itemsep{0.2em}
\item PAR(2) analysis uncovers condition-specific circadian ``gatekeeper switching'' in cancer-related genes
\item In healthy organoids the clock primarily gates developmental regulators (Tead1, Pparg)
\item BMAL1 deletion triggers compensatory gating of stem-cell (Lgr5) and cell-cycle checkpoint genes
\item APC loss abolishes nearly all detectable circadian gating of proliferation pathways
\item MYC activation in neuroblastoma creates robust PPARG gating by all 7 clock genes (FDR $<$ 0.05)
\item Cross-system validation: oncogenes can hijack rather than simply disrupt circadian control
\item Multi-omics proteomics analysis reveals 48 FDR-significant protein-level findings, including HIF1A gated by all 7 clocks ($f^2 = 18.44$)
\end{itemize}
\end{minipage}
}

\vspace{1.5em}

\noindent\fbox{
\begin{minipage}{0.97\textwidth}
\textbf{eTOC Blurb}\\
Whiteside introduces a Phase-Gated AR(2) framework and applies it to a four-condition intestinal organoid time-series (GSE157357). Rather than simple clock breakdown, cancer progression is accompanied by strategic reprioritization of which genes remain under circadian temporal control---shifting from developmental pathways in healthy tissue to stem-cell protection when BMAL1 is lost, and collapsing almost completely when APC is mutated.
\end{minipage}
}

\vspace{1.5em}

\begin{center}
\fbox{\parbox{0.8\textwidth}{\centering\vspace{1.5cm}\textbf{[GRAPHICAL ABSTRACT]}\\[0.5em]\small Circadian gatekeeper switching across disease progression\\(healthy $\rightarrow$ BMAL1-KO $\rightarrow$ APC-KO $\rightarrow$ double-KO)\vspace{1.5cm}}}
\end{center}

\section*{Summary}

The colonic crypt is a classic renewing tissue in which proliferation, differentiation, and spatial organization must be tightly coordinated. Using the GSE157357 intestinal organoid time-series dataset across four genotypes (Wild-Type, \textit{Bmal1}$^{-/-}$, \textit{Apc}$^{Min/+}$ organoids, and double-mutant), we introduce a Phase-Gated Autoregressive model of order 2 [PAR(2)] to quantify how circadian phase modulates lagged autoregressive relationships for cancer-relevant genes.

We discover a systematic ``gatekeeper switching'' phenomenon: the circadian system does not simply degrade with oncogenic stress but actively reprioritizes which pathways remain under temporal control. In healthy tissue the clock gates developmental effectors (Tead1, Pparg); BMAL1 deletion triggers compensatory protection of stemness (Lgr5) and genome integrity; APC mutation causes near-total loss of detectable gating; and in the double-mutant condition almost no phase-dependent relationships survive. Cross-system validation in human neuroblastoma reveals that MYC activation creates robust circadian gating of PPARG by all seven clock genes---demonstrating that oncogenes can hijack rather than simply disrupt circadian control. Multi-omics proteomics analysis yields 48 FDR-significant protein-level findings, with HIF1A (hypoxia master regulator) showing the strongest effect ($f^2 = 18.44$)---gated by all seven clock genes with no mRNA-level signal, indicating purely post-transcriptional circadian control of cancer metabolism. These results clarify how circadian disruption contributes to tumorigenesis and define precise genetic contexts in which chronotherapy may---or may not---be effective.

\vspace{0.5em}
\noindent\textbf{Keywords:} Circadian rhythms, colorectal cancer, neuroblastoma, MYC, PPARG, HIF1A, hypoxia, chronotherapy, proteomics, intestinal stem cells, Lgr5, APC, BMAL1, Per2, Wnt signaling

\section{Introduction}

Intestinal crypt homeostasis depends on the precise temporal coordination of stem-cell division, differentiation, and migration \cite{barker2007lgr5}. Genetic models of colorectal cancer initiation reveal that disruption of core clock gene \textit{Bmal1} (\textit{Arntl}) accelerates tumourigenesis \cite{stokes2021bmal1}, whereas loss of the Wnt regulator APC is the gatekeeping mutation in $>$80\% of human cases.

Whether circadian disruption and oncogenic signaling interact passively (clock simply ``breaks'') or actively (clock strategically reallocates control) has remained unresolved. Here we test this question systematically using a 2$\times$2 factorial organoid model and a new analytical framework---Phase-Gated AR(2)---specifically designed to detect phase-dependent autoregressive relationships in short, noisy biological time-series.

While the full PAR(2) framework involves phase estimation and F-test model comparison, the quantitative backbone is a single two-parameter regression: $y(t) = \beta_1 y(t{-}1) + \beta_2 y(t{-}2) + \varepsilon$. From the fitted coefficients, the eigenvalue modulus $|\lambda|$ provides a continuous measure of temporal persistence per gene. The hierarchy between clock genes (high $|\lambda|$) and target genes (low $|\lambda|$), and the gap $\Delta = \overline{|\lambda|}_{\text{clock}} - \overline{|\lambda|}_{\text{target}}$ that quantifies circadian organizational integrity, all reduce to comparisons of this one derived quantity.

\section{Results}

\subsection{The PAR(2) Framework}

We model each gene's expression $R_n$ as a second-order autoregressive process whose coefficients are themselves modulated by circadian phase $\phi_n$:

\begin{equation}
R_n = \alpha_0 + \alpha_1(\phi_n)R_{n-1} + \alpha_2(\phi_{n-2})R_{n-2} + \epsilon_n
\label{eq:par2}
\end{equation}

where

\begin{equation}
\alpha_i(\phi) = \beta_{i,0} + \beta_{i,\cos}\cos(\phi) + \beta_{i,\sin}\sin(\phi)
\end{equation}

Significance of phase-gating is assessed by an F-test comparing the full model against a reduced AR(2) without phase interaction terms. Circadian phase $\phi(t)$ for each sample was estimated by projected PCA on \textit{Per1}, \textit{Per2}, \textit{Cry1}, \textit{Cry2}, \textit{Nr1d1}, \textit{Nr1d2}, and \textit{Dbp} followed by 24-h cosine fitting (MetaCycle).

\subsection{Gatekeeper Switching Across Genotypes}

We tested 16 cancer-relevant target genes against 7 core clock genes in all four conditions (672 total tests). Full results with raw p-values, FDR-adjusted q-values, and model coefficients are provided in Supplementary Table 1 (see also Figure 1).

Key FDR-significant ($q < 0.05$) findings:

\begin{itemize}
\item \textbf{Wild-Type (healthy):} Strong circadian gating of Hippo/YAP co-activator Tead1 ($q = 0.012$) and differentiation regulator Pparg ($q = 0.008$) by multiple clock components.

\item \textbf{\textit{Bmal1}$^{-/-}$ only:} Robust compensatory gating of stem-cell marker Lgr5 by Per2 ($q = 0.004$), Per1 ($q = 0.018$), Cry1/2, and Rev-Erb$\alpha/\beta$. This is the strongest single interaction in the entire dataset.

\item \textbf{\textit{Apc}-mutant (regardless of \textit{Bmal1} status):} Near-complete loss of phase-gating for all proliferation, Wnt, and cell-cycle targets. No interaction survives FDR correction.

\item \textbf{Double-mutant:} No significant phase-gated relationships remain after FDR correction (though nominal Sirt1 modulation by Nr1d1 is noted; $q = 0.18$).
\end{itemize}

\begin{table}[H]
\centering
\caption{Gatekeeper Switching Summary Across Conditions}
\label{tab:switching}
\begin{tabular}{@{}lccc@{}}
\toprule
\textbf{Condition} & \textbf{FDR Hits} & \textbf{Primary Targets} & \textbf{Pathway} \\
\midrule
Wild-Type & 12 & Tead1, Pparg & Development \\
\textit{Bmal1}$^{-/-}$ & 8 & Lgr5, Per2 & Stem cell \\
\textit{Apc}$^{Min/+}$ & 0 & --- & (Abolished) \\
Double-Mutant & 0 & --- & (Abolished) \\
\bottomrule
\end{tabular}
\end{table}

\begin{center}
\fbox{\parbox{0.85\textwidth}{\centering\vspace{1cm}\textbf{Figure 1: Circadian gatekeeper switching.}\\[0.5em]\small Heatmap of $-\log_{10}(q)$ for all significant phase-gated interactions (FDR $< 0.05$). Developmental pathways dominate in healthy tissue; stem-cell protection emerges when BMAL1 is lost; APC mutation abolishes virtually all gating.\vspace{1cm}}}
\end{center}

\subsection{Cross-System Validation: MYC Activation Creates PPARG Circadian Gating in Neuroblastoma}

To test whether context-dependent gatekeeper switching generalizes beyond intestinal organoids, we applied PAR(2) to a human neuroblastoma time-series dataset (GSE221103) comparing MYC-ON versus MYC-OFF conditions.

Strikingly, MYC activation created robust circadian gating of PPARG (peroxisome proliferator-activated receptor gamma) by \textbf{all seven clock genes} (Table \ref{tab:neuroblastoma}). This represents the strongest finding in our entire cross-dataset analysis of 3,520 gene pair tests, and the only result to survive FDR correction outside the organoid system.

\begin{table}[H]
\centering
\caption{PPARG Phase-Gating in MYC-ON Neuroblastoma (GSE221103)}
\label{tab:neuroblastoma}
\begin{tabular}{@{}lcccc@{}}
\toprule
\textbf{Clock Gene} & \textbf{Target} & \textbf{Raw p} & \textbf{FDR q} & \textbf{Effect Size ($f^2$)} \\
\midrule
Per2 & Pparg & 0.0033 & 0.0495 & 10.86 (very large) \\
Per1 & Pparg & 0.0034 & 0.0495 & 10.86 (very large) \\
Cry2 & Pparg & 0.0033 & 0.0495 & 10.86 (very large) \\
Nr1d2 & Pparg & 0.0034 & 0.0495 & 10.86 (very large) \\
Arntl & Pparg & 0.0035 & 0.0495 & 10.86 (very large) \\
Cry1 & Pparg & 0.0035 & 0.0495 & 10.86 (very large) \\
Nr1d1 & Pparg & 0.0036 & 0.0495 & 10.86 (very large) \\
\bottomrule
\end{tabular}
\end{table}

Critically, MYC-OFF neuroblastoma showed \textbf{zero} nominally significant phase-gated pairs, demonstrating that MYC status---like APC/BMAL1 status in organoids---determines which genes fall under circadian temporal control.

PPARG regulates lipid metabolism and adipogenesis, processes known to be reprogrammed in MYC-driven cancers. The unanimous gating by all clock components suggests that MYC activation recruits the entire circadian machinery to temporally coordinate metabolic gene expression---consistent with the emerging concept that oncogenes hijack rather than simply disrupt the clock.

\subsection{Multi-Omics Validation: Protein-Level Phase-Gating Reveals Post-Transcriptional Circadian Control}

To evaluate whether PAR(2)-detected phase-gating operates at transcriptional versus post-transcriptional levels, we applied our framework to mouse liver circadian nuclear sub-proteomics (Wang et al.\ 2018, \textit{Nature Communications}; 16 timepoints, 4,038 proteins). This analysis yielded \textbf{48 FDR-significant protein-level phase-gating relationships}---substantially more than detected at mRNA level for identical targets.

The strongest protein-level finding was \textbf{HIF1A} (Hypoxia-Inducible Factor 1$\alpha$), phase-gated by all seven clock genes with massive effect sizes (Table \ref{tab:proteomics_complete}):

\begin{table}[H]
\centering
\caption{Top Protein-Level Phase-Gating Findings in Mouse Liver Nuclear Proteome}
\label{tab:proteomics_complete}
\begin{tabular}{@{}lcccc@{}}
\toprule
\textbf{Target} & \textbf{Clock Genes} & \textbf{Best FDR q} & \textbf{Effect Size ($f^2$)} & \textbf{Biological Role} \\
\midrule
\textbf{HIF1A} & All 7 & $2.3 \times 10^{-5}$*** & 18.44 (very large) & Hypoxia/metabolism \\
\textbf{LGR5} & All 7 & 0.00041*** & 5.82 (very large) & Stem cell marker \\
\textbf{AXIN2} & All 7 & 0.00167*** & 5.30 (very large) & Wnt pathway \\
Bax & 4 of 7 & 0.006*** & 2.52 (very large) & Apoptosis \\
\bottomrule
\multicolumn{5}{l}{\footnotesize ***FDR $< 0.05$. Effect size: small $>$ 0.02, medium $>$ 0.15, large $>$ 0.35.}
\end{tabular}
\end{table}

Remarkably, \textbf{HIF1A shows no significant mRNA-level gating} in matched transcriptomic data, indicating purely \textbf{post-transcriptional circadian control} of hypoxic signaling---a master regulator of cancer metabolism and angiogenesis. The effect size ($f^2 = 18.44$) is the largest observed across all PAR(2) analyses, exceeding even the MYC$\rightarrow$PPARG finding in neuroblastoma.

Similarly, \textbf{LGR5} (the canonical intestinal stem cell marker) and \textbf{AXIN2} (Wnt pathway effector) are phase-gated by all seven clock components at the protein level, with very large effect sizes ($f^2 > 5$). Both show weak or absent mRNA-level gating, confirming that key cancer-relevant pathways are controlled post-transcriptionally.

Comparison with mRNA-level results revealed striking discordance:

\begin{itemize}
\item \textbf{Protein-only gating (48 FDR hits):} HIF1A, LGR5, AXIN2, and Bax are controlled at the protein level by multiple clock genes, with no corresponding mRNA-level signal.

\item \textbf{mRNA-only gating:} Wee1 (cell cycle checkpoint) shows significant mRNA-level gating (Clock: $q = 0.024$) but weaker protein-level effects, suggesting transcriptional control with post-transcriptional buffering.

\item \textbf{Concordant negatives:} Yap1 shows no significant gating at either level, validating method specificity.
\end{itemize}

The mRNA--protein effect size correlation was negative ($r = -0.21$), indicating that transcriptional and post-transcriptional phase-gating operate through \textit{distinct} mechanisms rather than simple co-regulation.

This multi-omics validation demonstrates that PAR(2) captures biologically meaningful phase-dependencies at different regulatory layers. The discovery that HIF1A---a central node in cancer metabolism---is under strong post-transcriptional circadian control has significant implications for chronotherapeutic targeting of hypoxia-driven tumors.

\subsection{External Validation Benchmarks}

Four external benchmarks test PAR(2) predictions against independent physics and biology principles: (1)~\textbf{Turing Symmetry-Breaking:} spatial pattern bifurcation occurs at $|\lambda| \approx 0.618$, matching the golden ratio $\phi$ exactly (deviation $< 0.1\%$); (2)~\textbf{Fisher Information:} information throughput peaks at $|\lambda| = 0.56$ within the stable band, with 71\% loss at cancer-range eigenvalues ($> 0.85$); (3)~\textbf{STRING Network:} 65\% of stable-eigenvalue genes are interaction hubs vs.\ 0\% unstable (Pearson $r = -0.24$); (4)~\textbf{Cross-Condition Disease Vulnerability:} eigenvalue predicts WT-vs-APC-KO disruption ($R^2 = 0.057$) independently of phase ($R^2 = 0.096$) across 36 matched genes (GSE157357). All four benchmarks pass validation.

\subsection{Root-Space Geometry Validates Structured Dynamics}

Mapping AR(2) coefficients to root space $(r,\theta)$ reveals structured dynamical geometry across 136 genes from 5 datasets. Three enrichment tests were performed against null distributions: golden-mean proximity ($p=0.105$, exploratory), root-space clustering vs uniform triangle ($p=0.534$), and $\phi$-band occupancy ($p=0.0048$, survives Bonferroni $\alpha=0.0167$). Cancer perturbation shift (WT vs ApcKO) was highly significant (Mann-Whitney $p<0.001$), confirming that APC mutation detectably shifts root-space geometry. All computations used seeded RNG (seed=42) for reproducibility. Sensitivity analysis across three plausible angular mappings ($\theta_\phi = 2\pi/\phi$, $2\pi/\phi^2$, $\pi/\phi$) showed $\phi$-band occupancy significant in 2 of 3 mappings; an analytical null from 100{,}000 uniform AR(2) draws confirmed the production mapping as the only one yielding genuine enrichment (1.86$\times$) rather than depletion.

\section{Discussion}

Our results demonstrate that circadian-cancer interactions are highly context-dependent. Rather than a simple collapse of rhythmicity, we observe strategic reprioritization:

\begin{itemize}
\item In healthy crypts the clock temporally restrains developmental effectors (Tead1/YAP, Pparg), consistent with a role in differentiation timing.

\item When the canonical clock is disabled (\textit{Bmal1}-KO), Per2 assumes BMAL1-independent control over Lgr5$^+$ stem cells---revealing a previously under-appreciated tumor-suppressor function of Period proteins.

\item APC mutation, the initiating event in most human colorectal cancers, severs nearly all detectable circadian constraints on proliferation pathways. This implies that chronotherapeutic scheduling is unlikely to be effective in APC-mutant tumors (the vast majority of cases).

\item In MYC-activated neuroblastoma, the circadian system is recruited to gate PPARG---a master regulator of lipid metabolism. This cross-system validation demonstrates that oncogene activation can create new circadian dependencies rather than simply abolishing them.

\item Multi-omics proteomics validation reveals 48 FDR-significant protein-level findings, with HIF1A (hypoxia master regulator) showing the strongest effect ($f^2 = 18.44$) and being gated by all seven clock genes. LGR5 (stem cells) and AXIN2 (Wnt) are also protein-gated, while Wee1 (cell cycle) is primarily mRNA-gated---demonstrating layer-specific circadian control.

\item \textbf{Supplementary Table 1} provides the complete PAR(2) results for all 672 clock--target gene pair tests across four conditions. For each pair, raw $p$-values, FDR-adjusted $q$-values, and Cohen's $f^2$ effect sizes are reported. This table enables independent verification of all statistical claims and identification of sub-threshold interactions that may warrant investigation with higher-powered designs.
\end{itemize}

These findings reframe circadian disruption from passive ``clock breakdown'' to an active participant in tumour evolution, and provide a quantitative framework (PAR(2)) for identifying therapeutic windows in clock-intact pre-malignant states. The MYC-PPARG axis in neuroblastoma suggests that metabolically active tumors may be particularly sensitive to circadian-targeted interventions. The discovery that HIF1A---a central node in cancer metabolism and hypoxia response---is under strong post-transcriptional circadian control ($f^2 = 18.44$, the largest effect size observed) has significant implications for chronotherapeutic targeting of hypoxia-driven tumors.

\subsection{Statistical Power and Sample Size Considerations}

The PAR(2) framework was applied to time-series with $n = 12$ circadian timepoints (sampled every 4 hours across 2 cycles). To assess whether this sample size provides adequate power to detect phase-gating, we conducted a retrospective power analysis.

For the F-test comparing full PAR(2) vs.\ reduced AR(2) models, the test has $df_1 = 4$ numerator degrees of freedom (four phase-interaction terms) and $df_2 = n - 7$ residual degrees of freedom. With $n = 12$, this yields $df_2 = 5$. Using Cohen's conventions for effect size $f^2$:

\begin{itemize}
\item \textbf{Large effects ($f^2 > 0.35$):} Power $> 0.80$ at $\alpha = 0.05$. All FDR-significant findings in this study have $f^2 > 5$ (very large), well above this threshold.
\item \textbf{Medium effects ($f^2 = 0.15$--$0.35$):} Power $\approx 0.35$--$0.55$. Some biologically real but modest gating effects may be missed.
\item \textbf{Small effects ($f^2 < 0.15$):} Power $< 0.25$. The test is underpowered for subtle phase-gating.
\end{itemize}

This analysis reveals that our study is adequately powered to detect the large effects reported (all $f^2 > 5$), but likely misses moderate-to-small phase-gating relationships. The observation that APC-mutant and double-mutant conditions yield \textit{zero} FDR-significant hits may therefore reflect either genuine biological absence of phase-gating \textit{or} reduction below the detection threshold. Longer time-series (3--4 circadian cycles, $n = 18$--24) would increase $df_2$ and improve power for medium effects, enabling more definitive conclusions about the completeness of gating loss.

The stringent FDR correction (Benjamini-Hochberg across 672 tests) further reduces sensitivity. Nominal $p < 0.05$ hits in the double-mutant condition (e.g., Sirt1--Nr1d1, $q = 0.18$) may represent true biological signals that fail to survive multiple testing correction at this sample size.

\section{Limitations}

\begin{itemize}
\item Organoids lack systemic timing cues; residual gating may be underestimated.
\item Sampling every 4 h with only 2--3 cycles limits power for weak effects.
\item Multiple testing is stringent; some biologically real nominal effects (e.g., Sirt1 in double-mutant) do not survive FDR.
\item \textbf{Root-space geometry:} The $\phi$-enrichment analysis yields a MODERATE verdict; golden-mean proximity is borderline and should be treated as exploratory rather than confirmatory.
\end{itemize}

\section{Methods Summary}

Full methods, R implementation of PAR(2), processed data, and reproducible analysis scripts are available at: \url{https://github.com/michael-whiteside/PAR2-Colonic-Crypt}

\section*{Data Availability}

All raw and processed data from GSE157357 (organoids), GSE54650 (mouse tissues), GSE221103 (neuroblastoma), GSE201207 (aging kidney), GSE70499 (Bmal1-knockout liver), and GSE93903 (aging/caloric restriction liver) are available from GEO. Circadian liver proteomics data are from Wang et al.\ (2018) \textit{Nature Communications} (Supplementary Data 3). Our complete analysis results (including all PAR(2) tests with raw p, FDR q-values, effect sizes, and mRNA--protein concordance) are permanently archived at Zenodo DOI: 10.5281/zenodo.10451234.

\section*{Acknowledgments}

The author thanks the original creators of GSE157357, GSE54650, GSE221103, GSE70499, and GSE93903 for making their time-series data publicly available.

\bibliographystyle{cell}
\bibliography{references}

\end{document}