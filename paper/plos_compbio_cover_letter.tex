\documentclass[11pt]{letter}

\usepackage[margin=1in]{geometry}
\usepackage{hyperref}

\signature{Michael Whiteside\\Independent Researcher\\Scotland, United Kingdom}

\address{Michael Whiteside\\Independent Researcher\\Scotland, United Kingdom}

\begin{document}

\begin{letter}{Editorial Office\\PLOS Computational Biology}

\opening{Dear Editors,}

I am pleased to submit the manuscript entitled ``\textbf{A Phase-Gated AR(2) Framework Reveals Circadian `Gatekeeper Switching' During Cancer Progression}'' for consideration as a Research Article in \textit{PLOS Computational Biology}.

\textbf{Summary of the Work}

This manuscript introduces PAR(2), a novel Phase-Gated Autoregressive framework for detecting how circadian clock phase modulates temporal dependencies in gene expression. Unlike traditional rhythmicity detection methods (e.g., JTK\_CYCLE, RAIN), PAR(2) specifically tests whether clock gene phase predicts target gene autoregressive dynamics---capturing ``gatekeeper'' relationships that control \textit{when} genes can respond to their recent expression history.

\textbf{Key Findings}

Applying PAR(2) to intestinal organoid time-series data across four genotypes, we discover systematic ``gatekeeper switching'':

\begin{itemize}
\item In healthy tissue, the clock gates developmental regulators (Tead1, Pparg)
\item BMAL1 deletion triggers compensatory gating of stem-cell markers (Lgr5)
\item APC mutation abolishes virtually all detectable phase-gating
\item Cross-validation in human neuroblastoma reveals MYC-dependent PPARG gating
\item Multi-omics analysis demonstrates layer-specific control (mRNA vs protein)
\end{itemize}

\textbf{Significance for PLOS Computational Biology}

This work is well-suited for \textit{PLOS Computational Biology} because it:

\begin{enumerate}
\item \textbf{Introduces a novel computational method} (PAR(2)) with clear mathematical formulation and statistical framework
\item \textbf{Provides open-source implementation} in both TypeScript and Python with CLI interfaces
\item \textbf{Validates across multiple datasets} (4 organoid conditions, 12 mouse tissues, neuroblastoma, proteomics)
\item \textbf{Addresses a fundamental biological question} about circadian-cancer interactions
\item \textbf{Has translational implications} for chronotherapy patient stratification
\end{enumerate}

\textbf{Data and Code Availability}

All source code is available at \url{https://github.com/michael-whiteside/PAR2-Colonic-Crypt}. Complete analysis results are archived at Zenodo (DOI: 10.5281/zenodo.10451234). All datasets are publicly available from GEO.

\textbf{Declarations}

\begin{itemize}
\item This manuscript has not been published elsewhere and is not under consideration by another journal.
\item The author has no competing interests to declare.
\item No funding was received for this research.
\item All data analyzed are from publicly available sources.
\end{itemize}

I believe this work will be of significant interest to the computational biology community, particularly those working on circadian rhythms, cancer biology, and time-series analysis methods.

Thank you for considering this manuscript.

\closing{Sincerely,}

\end{letter}

\end{document}
