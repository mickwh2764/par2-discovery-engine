% ============================================================
% PAR(2) SUPPLEMENTARY SECTIONS
% To be inserted into PAR2_Complete_Manuscript-2.tex
% 
% Contents:
%   Section A: Clock Desynchrony Index (insert after clock-target hierarchy results)
%   Section B: p53 Pathway Eigenvalue Dynamics (insert after Section A)
%   Section C: AR Model Order Justification (insert in Methods/Robustness)
%   Section D: Future Validation Experiments (insert before Conclusions)
%   Section E: Root-Space Geometry and phi-Enrichment Analysis (insert after Section D)
%
% All numbers are computed from GEO datasets via the PAR(2) Discovery Engine.
% ============================================================


% ============================================================
% SECTION A: Clock Desynchrony Index
% Suggested placement: after the clock-target hierarchy results
% ============================================================

\subsection{Clock Desynchrony Index}

Circadian desynchronization---the loss of phase and amplitude coherence among clock genes within a tissue---has been identified as a hallmark of cancer \cite{filipski2017}. To quantify this using eigenvalues, we computed the coefficient of variation (CV) of clock gene eigenvalues within each condition. Higher CV indicates less coordinated clock dynamics.

\begin{table}[h]
\centering
\caption{Clock gene eigenvalue desynchrony across conditions}
\label{tab:desync}
\begin{tabular}{lcc}
\hline
\textbf{Condition} & \textbf{Clock Mean $|\lambda|$} & \textbf{CV} \\
\hline
Liver (Healthy) & 0.725 & 0.153 \\
Heart (Healthy) & 0.689 & 0.149 \\
Kidney (Healthy) & 0.777 & 0.154 \\
Lung (Healthy) & 0.804 & 0.148 \\
Muscle (Healthy) & 0.634 & 0.218 \\
Adrenal (Healthy) & 0.697 & 0.178 \\
Hypothalamus (Healthy) & 0.506 & 0.276 \\
MYC-ON Neuroblastoma & 0.619 & 0.312 \\
MYC-OFF Neuroblastoma & 0.614 & 0.298 \\
\hline
\end{tabular}
\end{table}

The cancer condition (MYC-ON neuroblastoma) showed higher desynchrony (CV=0.312) than the healthy tissue mean (CV=0.182, $n$=7 tissues), consistent with loss of clock coherence in malignancy. This observation requires validation across additional cancer types. Among healthy tissues, hypothalamus exhibited the highest desynchrony (CV=0.276), consistent with the central pacemaker integrating diverse oscillatory inputs.

\paragraph{Independence from the gap metric.}
Pairwise correlation analysis across 9 conditions showed that the gap and desynchrony CV are correlated ($r = -0.88$), indicating they capture related aspects of clock disruption. The p53 pathway axis, by contrast, is largely independent of both the gap ($r = 0.01$) and desynchrony ($r = -0.09$), confirming it captures distinct biological information.


% ============================================================
% SECTION B: p53 Pathway Eigenvalue Dynamics
% Suggested placement: immediately after Section A
% ============================================================

\subsection{p53 Pathway Eigenvalue Dynamics}

Given known chronic activation of the p53/Mdm2 axis under MYC-driven replication stress, we hypothesized that DNA damage response genes would show high eigenvalues (persistent dynamics) in MYC-ON cancer. To test this, we extended the analysis to eight DNA damage response genes (Tp53, Mdm2, Chek2, Cdkn1a, Bax, Bcl2, Atm, Gadd45a).

\begin{table}[h]
\centering
\caption{p53 pathway eigenvalues: healthy vs cancer}
\label{tab:p53}
\begin{tabular}{lcccc}
\hline
\textbf{Condition} & \textbf{Clock $|\lambda|$} & \textbf{Target $|\lambda|$} & \textbf{p53 Pathway $|\lambda|$} & \textbf{p53 Closer To} \\
\hline
Healthy tissues ($n$=8, mean) & 0.648 & 0.478 & \textbf{0.452} & \textbf{Target} (8/8) \\
MYC-ON Cancer ($n$=1) & 0.639 & 0.541 & \textbf{0.665} & \textbf{Clock} \\
\hline
\end{tabular}
\end{table}

In all eight healthy conditions, p53 pathway genes consistently behaved like target genes (low persistence, responsive). In MYC-ON neuroblastoma (the one cancer condition with p53 pathway gene coverage), p53 pathway eigenvalues jumped to 0.665---exceeding both clock and target means. Individual cancer eigenvalues were striking: Tp53 = 0.901, Mdm2 = 0.953, Gadd45a = 1.095 (unstable).

\paragraph{Interpretation.}
High p53 pathway eigenvalues in MYC-ON cancer suggest the DNA damage response has become ``stuck''---persisting rather than responding flexibly to signals. Eigenvalues exceeding 1 (Gadd45a = 1.095) correspond to locally unstable dynamics in the AR(2) surrogate, consistent with runaway stress signaling and failure to re-establish homeostasis. This is consistent with chronically activated p53/Mdm2 under constitutive MYC-driven replication stress.

Notably, the AR(2) analysis is blind to pathway identity---it processes each gene's time series without knowledge of function. The independent recovery of this canonical cancer biology pattern (persistent p53/Mdm2 activation in MYC-driven malignancy) provides strong support for the biological plausibility of eigenvalue-based analysis. This finding is preliminary ($n$=1 cancer condition) and requires replication across additional cancer types.


% ============================================================
% SECTION C: AR Model Order Justification
% Suggested placement: in Methods, Robustness/Validation subsection
% ============================================================

\subsection{AR Model Order Selection}

We compared AR(1), AR(2), and AR(3) model orders on GSE54650 liver (24 timepoints, 15 genes) using both Ljung-Box residual whiteness (lags=5, $\alpha$=0.05) and clock-target eigenvalue separation.

\begin{table}[h]
\centering
\caption{Model order comparison: residual whiteness vs eigenvalue separation}
\label{tab:arorder}
\begin{tabular}{lcccc}
\hline
\textbf{Model} & \textbf{LB Pass Rate} & \textbf{Clock Mean $|\lambda|$} & \textbf{Target Mean $|\lambda|$} & \textbf{Gap} \\
\hline
AR(1) & 67\% (10/15) & 0.726 & 0.302 & +0.424 \\
\textbf{AR(2)} & \textbf{93\% (14/15)} & \textbf{0.725} & \textbf{0.479} & \textbf{+0.245} \\
AR(3) & 87\% (13/15) & 0.553 & 0.565 & $-$0.012 \\
\hline
\end{tabular}
\end{table}

AR(1) underfits: 5 of 13 clock genes fail the Ljung-Box test (residual autocorrelation remains), indicating AR(1) misses significant temporal structure. AR(3) overfits: while residual whiteness is acceptable (87\%), the clock-target gap collapses to near-zero ($-$0.012), suggesting that the third lag absorbs biologically meaningful variance into noise estimation. AR(2) achieves the best residual whiteness (93\%) while preserving eigenvalue separation (+0.245).

The single AR(2) failure was Cry1 ($p$=0.018); AR(3) did not resolve this gene's residual autocorrelation.


% ============================================================
% SECTION D: Future Validation Experiments
% Suggested placement: after Limitations, before Conclusions
% ============================================================

\subsection{Future Validation}

The following experiments would strengthen the framework:

\begin{enumerate}
    \item \textbf{Prospective circadian time courses} in matched healthy/tumor tissue from the same patient, analyzed with pre-registered eigenvalue hypotheses (clock $>$ target in healthy, convergence or reversal in tumor).
    
    \item \textbf{CRISPR perturbation of specific clock genes} (e.g., BMAL1 knockout) followed by AR(2) analysis, to test whether loss of a single clock gene reduces the eigenvalue gap as predicted.
    
    \item \textbf{Single-cell time-lapse data} (e.g., from Fucci reporters) to determine whether eigenvalue signatures are preserved at single-cell resolution or emerge only as population-level statistics.
    
    \item \textbf{Pan-cancer GEO meta-analysis} covering $\geq$5 cancer types with matched healthy controls, to establish whether clock-target convergence is cancer-general or specific to MYC-driven and APC-mutant contexts.
    
    \item \textbf{Pharmacological chronotherapy trials} measuring eigenvalue gap before and after timed drug administration, to test whether the gap serves as a predictive biomarker for circadian-aligned treatment response.
\end{enumerate}


% ============================================================
% SECTION E: Root-Space Geometry and phi-Enrichment Analysis
% Suggested placement: after Section D, before Conclusions
% ============================================================

\subsection{Root-Space Geometry and $\phi$-Enrichment Analysis}

For each gene the fitted AR(2) coefficients $(a_1, a_2)$ were mapped to characteristic-polynomial root space, yielding a dominant eigenvalue $\lambda_{\max}$ whose modulus encodes persistence and whose argument encodes oscillatory frequency. We defined the golden-mean proximity metric $D_\phi = \bigl||\lambda_{\max}| - \phi^{-1}\bigr|$, where $\phi^{-1} \approx 0.618$, and tested whether the observed distribution of $D_\phi$ values across all retained genes is enriched relative to three null distributions.

\begin{table}[h]
\centering
\caption{Root-space $\phi$-enrichment tests (Bonferroni-corrected $\alpha = 0.0167$, $k=3$)}
\label{tab:phi_enrich}
\begin{tabular}{lccc}
\hline
\textbf{Test} & \textbf{$p$-value} & \textbf{Significant?} & \textbf{Note} \\
\hline
Golden-Mean Proximity (vs Phase-Randomized) & 0.105 & No & Does not survive correction \\
Root-Space Clustering (vs Uniform Triangle) & 0.534 & No & No spatial clustering detected \\
$\phi$-Band Occupancy & 0.0048 & Yes & Survives Bonferroni $\alpha = 0.0167$ \\
\hline
\end{tabular}
\end{table}

Only one of three tests ($\phi$-band occupancy, $p = 0.0048$) survives Bonferroni correction at the family-wise $\alpha = 0.05$ level. The remaining two tests are non-significant, indicating that any golden-mean proximity in the eigenvalue spectrum is subtle and does not constitute a dominant geometric feature of the root space.

\paragraph{Perturbation shift.}
To assess whether $D_\phi$ distributions shift under oncogenic perturbation, we compared wild-type (WT) intestinal organoids with ApcKO cancer organoids (GSE157357). A Mann--Whitney $U$ test yielded $p < 0.001$, indicating a significant shift in root-space geometry between the two conditions. This is consistent with the eigenvalue gap results reported in the main text and suggests that golden-mean proximity, while not globally enriched, is differentially affected by cancer-associated mutations.

\paragraph{Methodology transparency.}
All permutation and resampling procedures used a seeded pseudorandom number generator (seed = 42) to ensure exact reproducibility. Root dominance was determined by $|\lambda_{\max}|$, selecting the eigenvalue with the largest modulus from each gene's characteristic polynomial. Datasets with fewer than 6 timepoints or fewer than 10 genes passing quality filters were excluded from this analysis.

\paragraph{Mapping sensitivity.}
The angular reference $\theta_\phi = 2\pi/\phi$ is one of several plausible mappings. We tested three candidates: $\theta_\phi = 2\pi/\phi$ (3.883~rad, production mapping), $\theta_\phi = 2\pi/\phi^2$ (2.400~rad, phyllotaxis convention), and $\theta_\phi = \pi/\phi$ (1.942~rad). The $\phi$-band occupancy test was significant under 2 of 3 mappings ($p = 0.0039$ and $p = 0.0012$; both survive Bonferroni), failing only under $\pi/\phi$ ($p = 0.644$). An analytical null computed from 100{,}000 uniform random AR(2) draws showed that the production mapping ($2\pi/\phi$) is the only one where biology is \emph{enriched} relative to the analytical baseline (1.86$\times$ vs 9.9\% null occupancy); the other mappings show depletion (0.51$\times$ and 0.07$\times$). This confirms that $\theta_\phi = 2\pi/\phi$ is the most defensible angular reference and that the enrichment result is not an artifact of coordinate choice.

\paragraph{Overall verdict.}
The evidence for golden-mean enrichment in the AR(2) root space is \textbf{MODERATE}: one of three formal tests is significant after multiple-testing correction, a biologically motivated perturbation comparison shows a clear shift, and the $\phi$-band occupancy result is robust across 2 of 3 plausible angular mappings with unique enrichment (not depletion) under the production mapping. The result is insufficient to claim universal $\phi$-proximity but sufficient to warrant further investigation with higher-resolution time courses.


% ============================================================
% ADDITIONAL LIMITATIONS (append to existing Limitations section)
% ============================================================

% Add these two items to the existing limitations list:
%
% \item \textbf{Gap and desynchrony correlation:} The clock-target gap and 
%   desynchrony CV are strongly correlated ($r = -0.88$), indicating they 
%   measure overlapping aspects of clock disruption rather than fully 
%   independent dimensions.
%
% \item \textbf{AR(2) approximation:} The ODE-AR(2) bridge assumes 
%   linearization around equilibrium; nonlinear dynamics or 
%   far-from-equilibrium transitions may not be captured.


% ============================================================
% ADDITIONAL REFERENCE
% ============================================================

% Add to bibliography:
% @article{filipski2017,
%   author = {Filipski, E. and others},
%   title = {Circadian disruption in experimental cancer processes},
%   journal = {Front Endocrinol},
%   year = {2017}
% }
