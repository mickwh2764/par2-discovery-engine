\documentclass[11pt,letterpaper]{article}

\usepackage[utf8]{inputenc}
\usepackage[T1]{fontenc}
\usepackage{geometry}
\geometry{margin=1in}
\usepackage{graphicx}
\usepackage{booktabs}
\usepackage{longtable}
\usepackage{array}
\usepackage{amsmath}
\usepackage{amssymb}
\usepackage{natbib}
\usepackage{hyperref}
\usepackage{xcolor}
\usepackage{float}
\usepackage{caption}
\usepackage{lineno}
\linenumbers

\title{\textbf{Memory as a Biological Property: How a Single Number from Time-Series Econometrics Reveals Hierarchy in Gene Expression}}

\author{
Michael Whiteside$^{1,*}$ \\
\\
$^1$Independent Researcher, United Kingdom \\
\\
$^*$Corresponding author: mickwh@msn.com \\
ORCID: 0009-0000-0643-5791
}

\date{February 2026}

\begin{document}

\maketitle

\begin{abstract}
\noindent The eigenvalue modulus $|\lambda|$ has been the universal language of dynamical systems for over 200 years. It determines energy levels in quantum mechanics, natural frequencies in vibration analysis, stability in control theory, growth rates in ecology, persistence timescales in economics, and memory in climate models. Here we argue that the eigenvalue modulus of a second-order autoregressive model applied to gene expression time series measures a biologically meaningful quantity: \textit{temporal persistence}, or how much a gene's past determines its future. We present evidence from 37 datasets across 4 species that this single metric blindly recovers known circadian hierarchy, is partially independent of network connectivity and chromatin state, converges with four independent research programs, and reproduces the dynamical properties of five canonical ODE models from biology. We propose that temporal persistence is an underappreciated axis of biological variation---distinct from expression level, connectivity, or epigenetic state---and that the eigenvalue provides a direct measurement of this property using the same mathematics that has proven foundational across physical, engineering, and social sciences.

\noindent\textbf{Keywords:} eigenvalue modulus, temporal persistence, gene expression memory, dynamical systems, circadian hierarchy, cross-disciplinary, autoregressive model
\end{abstract}

\section{Introduction: The Most Universal Number in Science}

In 1904, David Hilbert posed a question that would shape modern mathematics: given a linear transformation, what are its characteristic values? The answer---eigenvalues---became the foundation of every field that studies dynamical systems.

The word ``eigenvalue'' comes from the German \textit{eigen}, meaning ``own'' or ``characteristic.'' An eigenvalue is literally the system's own number: the quantity that tells you how the system naturally behaves when left alone. Whether the system is an atom (Schr\"{o}dinger, 1926), a bridge (vibration analysis), an economy (Sims, Nobel 2011), or a climate (Hasselmann, Nobel 2021), the eigenvalue answers the same question: \textit{does the system grow, decay, or oscillate, and how fast?}

We propose that this same question---applied to gene expression---reveals fundamental biological organization.

\section{The Eigenvalue Across Disciplines}

Table~\ref{tab:eigenvalue_fields} summarizes eigenvalue usage across 10 fields.

\begin{table}[H]
\centering
\caption{The eigenvalue modulus across scientific disciplines}
\label{tab:eigenvalue_fields}
\small
\begin{tabular}{p{3cm}p{3cm}p{3.5cm}p{3cm}}
\toprule
\textbf{Field} & \textbf{Key figures} & \textbf{Eigenvalue measures} & \textbf{Relation to PAR(2)} \\
\midrule
Quantum mechanics & Schr\"{o}dinger, Dirac & Energy levels & Same math, different system \\
Vibration analysis & Engineers, seismologists & Natural frequencies & Very close (resonance zone) \\
Control theory & Control engineers & System stability & Identical framework \\
Google PageRank & Page, Brin & Page importance & Different interpretation \\
PCA / ML & Data scientists & Variance explained & Different dimension \\
Population ecology & Leslie, Caswell & Growth rates & Very close (persistence) \\
Neuroscience & Friston, Breakspear & Brain state stability & Close (different timescale) \\
Economics & Sims, Granger (Nobel) & Economic stationarity & Direct source of PAR(2) \\
Climate science & Hasselmann (Nobel) & Climate memory & Identical concept \\
Markov chains & Markov, probabilists & Mixing time & Close (inverse measure) \\
\bottomrule
\end{tabular}
\end{table}

\subsection{The Common Thread}

In every field, the eigenvalue modulus tells you one thing: \textbf{how much the system remembers}. High eigenvalue $\rightarrow$ strong memory $\rightarrow$ hard to disrupt. Low eigenvalue $\rightarrow$ weak memory $\rightarrow$ easily pushed around.

The stationarity boundary ($|\lambda| = 1$) means the same thing everywhere:
\begin{itemize}
\item $|\lambda| < 1$: Perturbations decay. The system forgets.
\item $|\lambda| = 1$: Perturbations persist indefinitely. Perfect memory.
\item $|\lambda| > 1$: Perturbations grow. The system is unstable.
\end{itemize}

\section{Evidence That Eigenvalue Measures Biological Memory}

\subsection{Blind Hierarchy Recovery}

The AR(2) eigenvalue modulus, computed without any biological labels, recovers the known circadian hierarchy:

\begin{equation}
|\lambda|_{\text{clock}} > |\lambda|_{\text{target}} > |\lambda|_{\text{other}}
\end{equation}

across 14/14 datasets, 4 species, and 12 tissues. This means genes that biologists have spent decades classifying as ``core clock'' happen to have the highest temporal persistence. The AR(2) model recovers this classification from two coefficients.

\subsection{Partial Independence from Other Metrics}

If eigenvalue merely recapitulated existing biological metrics, it would add nothing new. We tested $|\lambda|$ against four independent metrics:

\begin{center}
\begin{tabular}{lrp{5cm}}
\toprule
\textbf{Comparison} & \textbf{Spearman $\rho$} & \textbf{Interpretation} \\
\midrule
$|\lambda|$ vs. STRING network degree & $-0.29$ & Weak negative; high-memory genes are not highly connected \\
$|\lambda|$ vs. cosinor amplitude & $0.69$ & Moderate; oscillation strength partially correlates \\
$|\lambda|$ vs. H3K4me3 chromatin & $0.08$ & Near-zero; temporal memory is independent of chromatin marks \\
$|\lambda|$ vs. AR(2) $R^2$ & $0.83$ & Expected; model fit correlates with signal strength \\
\bottomrule
\end{tabular}
\end{center}

The key finding: \textbf{eigenvalue is nearly independent of chromatin state ($\rho = 0.08$)} and weakly anti-correlated with network connectivity ($\rho = -0.29$). This means temporal persistence is not a proxy for being ``important'' (highly connected) or ``active'' (open chromatin). It measures something distinct: how strongly the gene's past constrains its future.

\subsection{ODE Model Convergence}

Five canonical ODE models from biology---when simulated, discretized, and fitted with AR(2)---produce eigenvalues consistent with their known dynamics. This confirms that the AR(2) eigenvalue faithfully captures the underlying dynamical regime, not a statistical artifact of the fitting procedure.

\subsection{Convergence with Independent Research Programs}

The eigenvalue framework converges with four independently developed research programs:

\begin{enumerate}
\item \textbf{Boman Lab} (cell division timing): 5 convergence points, 55--95\% confidence
\item \textbf{Takahashi/Hogenesch} (circadian canon): 6 convergence points, 55--92\% confidence
\item \textbf{Five Biological Rules} (biological organization): 5 convergence points, 58--82\% confidence
\item \textbf{Waddington Landscape} (epigenetic topology): 5 convergence points, 72--85\% confidence
\end{enumerate}

These convergences were not designed or expected. They emerged when comparing PAR(2) predictions against published results from research programs with no connection to autoregressive modeling.

\section{Why Gene Expression Has Memory}

The biological basis of temporal persistence is multi-generational cellular memory. When a cell divides, its daughter cells inherit not just DNA sequence but also:

\begin{itemize}
\item Chromatin states (histone modifications, DNA methylation)
\item Transcription factor concentrations
\item mRNA and protein half-lives
\item Metabolite pools
\end{itemize}

These inherited states create autocorrelation in gene expression: what the cell expressed yesterday influences what it expresses today. The AR(2) model captures this with two lags---expression at time $t$ depends on expression at $t-1$ (one cell cycle ago) and $t-2$ (two cell cycles ago).

This two-step memory is not arbitrary. The AR(2) model is preferred over AR(1) and AR(3) for $>$70\% of genes by AIC/BIC, suggesting that biological memory extends approximately two cell cycles into the past but not further.

\section{Why This Matters}

If temporal persistence is a genuine biological property, it has immediate implications:

\subsection{Drug Targeting}

Genes with high $|\lambda|$ (strong memory) are hard to perturb with drugs---they bounce back. Genes with low $|\lambda|$ (weak memory) are easy to push around. This predicts that:
\begin{itemize}
\item Drugs targeting high-$|\lambda|$ genes need sustained dosing
\item Drugs targeting genes in the resonance zone should be timed to the circadian cycle
\item Combination therapies may need to target both a gene and its memory mechanism
\end{itemize}

\subsection{Aging}

If aging erodes temporal persistence---genes lose memory with age---then the eigenvalue could serve as a dynamical aging clock, complementary to epigenetic clocks (Horvath) that measure methylation age.

\subsection{Disease}

Cancer, neurodegeneration, and metabolic syndrome all involve circadian disruption. The eigenvalue provides a per-gene, per-tissue measure of how much temporal organization has been lost, potentially identifying which genes are most disrupted and which might be most treatable.

\section{The Minimal Model Argument}

Perhaps the most striking aspect of this work is the simplicity of the model. Two coefficients ($\phi_1$, $\phi_2$) fitted by ordinary least squares. No deep learning, no complex priors, no multi-parameter optimization. The eigenvalue modulus derived from these two numbers recovers:

\begin{itemize}
\item Clock $>$ target hierarchy across 4 species
\item Nine-category functional hierarchy across 1,594 genes
\item 60-fold clock gene enrichment in the resonance zone
\item 180-fold BMAL1 coupling enrichment over random predictors
\item 22 convergence points with independent research programs
\item ODE model dynamics from 5 canonical systems
\end{itemize}

If the underlying biology did not have a low-dimensional dynamical structure, two numbers could not capture all of this. The success of the minimal model is itself evidence that gene expression dynamics are simpler---more structured---than their apparent complexity suggests.

\section{Conclusion}

We propose that temporal persistence---how much a gene's past determines its future---is a fundamental axis of biological variation, distinct from expression level, network connectivity, or epigenetic state. The AR(2) eigenvalue modulus provides a direct, interpretable measurement of this property using mathematics that has been refined across physics, engineering, economics, and climate science over two centuries.

The eigenvalue is not new. What is new is recognizing that it applies to genes, and that the result is biologically meaningful. Gene expression is a dynamical system, and the universal language of dynamical systems---the eigenvalue---reads it just as well as it reads bridges, atoms, economies, and climate.

\bibliographystyle{plain}
\begin{thebibliography}{15}

\bibitem{takahashi2017transcriptional}
Takahashi JS.
Transcriptional architecture of the mammalian circadian clock.
\textit{Nat Rev Genet}. 2017;18(3):164--179.

\bibitem{sims1980macroeconomics}
Sims CA.
Macroeconomics and reality.
\textit{Econometrica}. 1980;48(1):1--48.

\bibitem{hasselmann1976stochastic}
Hasselmann K.
Stochastic climate models Part I. Theory.
\textit{Tellus}. 1976;28(6):473--485.

\bibitem{alon2006introduction}
Alon U.
\textit{An Introduction to Systems Biology: Design Principles of Biological Circuits}.
Chapman \& Hall/CRC; 2006.

\bibitem{zhang2014circadian}
Zhang R, Lahens NF, Ballance HI, Hughes ME, Hogenesch JB.
A circadian gene expression atlas in mammals.
\textit{Proc Natl Acad Sci USA}. 2014;111(45):16219--16224.

\bibitem{heard2014transgenerational}
Heard E, Martienssen RA.
Transgenerational epigenetic inheritance: myths and mechanisms.
\textit{Cell}. 2014;157(1):95--109.

\bibitem{mure2018diurnal}
Mure LS, Le HD, Benegiamo G, et al.
Diurnal transcriptome atlas of a primate across major neural and peripheral organs.
\textit{Science}. 2018;359(6381):eaao0318.

\bibitem{welsh2004bioluminescence}
Welsh DK, Yoo SH, Liu AC, Takahashi JS, Kay SA.
Bioluminescence imaging of individual fibroblasts reveals persistent, independently phased circadian rhythms of clock gene expression.
\textit{Curr Biol}. 2004;14(24):2289--2295.

\bibitem{westermark2009quantification}
Westermark PO, Welsh DK, Okamura H, Herzel H.
Quantification of circadian rhythms in single cells.
\textit{PLoS Comput Biol}. 2009;5(11):e1000580.

\bibitem{stelling2004robustness}
Stelling J, Sauer U, Szallasi Z, Doyle FJ, Doyle J.
Robustness of cellular functions.
\textit{Cell}. 2004;118(6):675--685.

\bibitem{leslie1945use}
Leslie PH.
On the use of matrices in certain population mathematics.
\textit{Biometrika}. 1945;33(3):183--212.

\end{thebibliography}

\end{document}
