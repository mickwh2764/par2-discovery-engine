\documentclass[11pt,letterpaper]{article}
\usepackage[utf8]{inputenc}
\usepackage[T1]{fontenc}
\usepackage{geometry}
\geometry{margin=1in}
\usepackage{hyperref}

\pagestyle{empty}

\begin{document}

\today

\vspace{1cm}

Dear Editor,

\vspace{0.5cm}

We submit for your consideration the manuscript entitled ``\textbf{Circadian Resonance Zone: Identifying Genes Tuned to 24-Hour Oscillation Through Damping-Period Decomposition of AR(2) Roots}'' for publication in \textit{Proceedings of the National Academy of Sciences}.

\vspace{0.3cm}

\textbf{Summary.} We introduce the concept of a ``circadian resonance zone''---genes whose intrinsic dynamical properties (natural period near 24 hours, low damping) make them inherently tuned to the circadian driving signal. By decomposing AR(2) complex conjugate roots into damping rate and natural period, we screen $\sim$21,000 genes genome-wide and identify 114 resonance zone genes in mouse liver, with 60-fold enrichment of known clock genes ($p < 10^{-6}$).

\vspace{0.3cm}

\textbf{Significance.} This is, to our knowledge, the first application of resonance theory from physics and engineering to gene expression dynamics. The 60-fold clock gene enrichment---achieved without any biological labels---validates the concept. The 22 multi-tissue resonance genes represent novel circadian gene candidates with immediate relevance for chronotherapy target identification. The resonance concept connects circadian biology to the foundational physics principle that systems respond most strongly to signals matching their natural frequency.

\vspace{0.3cm}

\textbf{Novel predictions.} We identify 22 genes appearing in the resonance zone in 2+ tissues, including Dtx4 (Notch signaling), Tspan4 (membrane organization), and Hlf (PAR bZIP transcription factor). These represent experimentally testable predictions for circadian gene discovery.

\vspace{0.3cm}

This manuscript has not been submitted elsewhere and all work is original.

\vspace{0.5cm}

Sincerely,

\vspace{0.5cm}

Michael Whiteside \\
Independent Researcher, United Kingdom \\
mickwh@msn.com \\
ORCID: 0009-0000-0643-5791

\end{document}
