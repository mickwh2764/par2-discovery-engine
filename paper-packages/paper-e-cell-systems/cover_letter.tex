\documentclass[11pt,letterpaper]{article}
\usepackage[utf8]{inputenc}
\usepackage[T1]{fontenc}
\usepackage{geometry}
\geometry{margin=1in}
\usepackage{hyperref}

\pagestyle{empty}

\begin{document}

\today

\vspace{1cm}

Dear Editors of \textit{Cell Systems},

\vspace{0.5cm}

We submit for your consideration the manuscript entitled ``\textbf{A Stability-Constrained Phase-Dependent AR(2) Framework for Discovering Cross-Tissue Circadian Gating Architectures}'' for publication as a Research Article in \textit{Cell Systems}.

\vspace{0.3cm}

\textbf{Summary.} We introduce a stability-constrained, phase-dependent autoregressive model of order two (PAR(2)) that systematically maps how circadian clocks gate downstream targets across tissues. Unlike standard AR(2) models with fixed coefficients, PAR(2) allows autoregressive coefficients and coupling terms to vary smoothly over circadian phase via sinusoidal basis functions, while enforcing dynamical stability at all phases. Applied to tens of thousands of clock--target expression time series from liver, heart, cerebellum, and intestinal organoids, PAR(2) recovers known gating motifs and reveals novel tissue-specific architectures.

\vspace{0.3cm}

\textbf{Key Contributions.}
\begin{enumerate}
\item \textbf{Phase-dependent gating discovery:} PAR(2) identifies tissue-specific circadian gating modules---Wee1-centred in liver (G2/M checkpoint), Tead1/YAP1-linked in heart (Hippo pathway), and Cdk1-centred in cerebellum (mitotic machinery)---using residual-based permutation tests and FDR control.

\item \textbf{Golden-ratio-like eigenstructures:} Over 80\% of Tier 0/1 hits exhibit eigenvalue ratios within 2--5\% of golden-ratio-related values, serving as geometric descriptors of robust oscillatory gating rather than design principles.

\item \textbf{Perturbation validation:} In intestinal organoids, Apc mutation strengthens circadian gating signatures while combined Apc/Bmal1 disruption collapses them, consistent with a two-hit model of circadian defence failure.

\item \textbf{Cross-tissue replication:} A tiered discovery scheme (Tier 0/1/2) based on cross-dataset and cross-tissue replication prioritises robust findings beyond conventional FDR control.
\end{enumerate}

\vspace{0.3cm}

\textbf{Reproducibility.} All source datasets are publicly available from NCBI GEO. The PAR(2) Discovery Engine web platform (\url{https://par2-discovery-engine.replit.app}) provides interactive access to all analyses, including phase portrait exploration, root-space geometry, and cross-tissue validation.

\vspace{0.3cm}

\textbf{Dual-Submission Transparency.} A related brief mathematical commentary (``Reply to Boman'') has been submitted to \textit{The Fibonacci Quarterly}. The present manuscript is the only submission containing the biological gating mechanism, tissue-specific architectures, organoid perturbation analysis, and cross-tissue discovery framework.

\vspace{0.3cm}

This manuscript has not been previously published and is not under consideration by any other primary scientific journal.

\vspace{0.5cm}

Sincerely,

\vspace{0.5cm}

Michael Whiteside \\
Independent Researcher, United Kingdom \\
mickwh@msn.com \\
ORCID: 0009-0000-0643-5791

\end{document}
