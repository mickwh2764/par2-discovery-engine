\documentclass[11pt,a4paper]{article}
\usepackage[utf8]{inputenc}
\usepackage[T1]{fontenc}
\usepackage{geometry}
\usepackage{graphicx}
\usepackage{amsmath,amssymb}
\usepackage{booktabs}
\usepackage{longtable}
\usepackage{natbib}
\usepackage{hyperref}
\usepackage{xcolor}
\usepackage{float}
\usepackage{caption}
\usepackage{subcaption}
\usepackage{array}
\usepackage{lineno}
\usepackage{enumitem}

\geometry{margin=1in}

\title{\textbf{Phase-Amplitude-Relationship (PAR2) Analysis Reveals Emergent Temporal Dynamics in Circadian-Cancer Gene Networks: A Systems-Level Discovery Framework}}

\author{
Michael Whiteside$^{1,*}$ \\
\\
\small $^1$Independent Researcher, United Kingdom \\
\\
\small $^*$Corresponding author: mickwh@msn.com \\
\small ORCID: 0009-0000-0643-5791
}

\date{February 2026}

\begin{document}

\maketitle
\linenumbers

\begin{abstract}
\textbf{Background:} Circadian disruption is epidemiologically linked to cancer risk, but the molecular mechanisms by which clock genes regulate cancer-related gene expression remain poorly characterized at the systems level. While individual clock-controlled genes have been extensively studied, a comprehensive framework for quantifying phase-dependent regulation across multiple tissues and cancer contexts has been lacking. Moreover, whether the circadian system actively \textit{compensates} for oncogenic mutations---fighting back rather than merely failing---remains unaddressed.

\textbf{Methods:} We developed PAR(2), a Phase-Amplitude-Relationship framework that models target gene expression as a function of clock gene phase through second-order autoregressive dynamics. The AR(2) model order is validated by mechanistic ODE systems (Boman C-P-D crypt model, Rosen Wnt anti-resonance model, Leloup-Goldbeter circadian clock), establishing that eigenvalue modulus $|\lambda|$ is approximately preserved across continuous, discrete, and autoregressive representations under standard linearization and sampling assumptions:
\[R_n = \alpha_0 + \alpha_1(\Phi_{n-1})R_{n-1} + \alpha_2(\Phi_{n-2})R_{n-2} + \varepsilon_n\]
We analyzed 28,138 clock-target gene pairs across 22 circadian transcriptomic datasets encompassing 22 tissue-condition combinations, including 12 mouse tissues from the Hughes Circadian Atlas (GSE54650), the gold-standard high-resolution liver dataset with 48 hourly timepoints (GSE11923), intestinal organoid models with genetic perturbations of APC and BMAL1 (GSE157357), and human neuroblastoma cell lines with inducible MYC expression (GSE221103). Rigorous permutation testing using four distinct null models (time-shuffle, pair-shuffle, phase-scramble, circular-shift) validated the robustness of both systems-level and cross-tissue consensus findings. An eleven-analysis robustness suite confirmed the hierarchy survives sub-sampling, bootstrap resampling, linear detrending, gap permutation, cross-dataset validation, and leave-one-tissue-out cross-validation.

\textbf{Results:} Across 28,138 gene pairs tested, 2,697 (9.6\%) showed Bonferroni-corrected significance and 33 (0.1\%) met stringent FDR thresholds. Pan-tissue analysis of 3,588 interactions across 12 mouse tissues identified 177 significant circadian gating relationships (4.9\% discovery rate). \textbf{Cry1$\rightarrow$Wee1 was the only relationship conserved across $\geq$6 tissues}, identifying the G2/M checkpoint kinase as the strongest candidate for a broadly conserved circadian gatekeeper among 23 cancer-relevant genes tested. Myc---often considered the canonical circadian-cancer target---was gated only in Muscle and Kidney, challenging its assumed primacy. Each tissue deployed distinct gating programs: liver guards DNA repair (ATM, Wee1), heart controls Hippo/YAP growth signaling (Tead1), and cerebellum gates CDK1 mitosis. Cross-tissue consensus requiring significance in 3+ tissues reduced FDR from $\sim$16\% to approximately 1--5\% (order-of-magnitude estimate).

Extending beyond the binary clock/target split, we classified $\sim$1,594 genes into nine functional categories and discovered a fine-grained persistence hierarchy: Clock ($|\lambda|=0.70$) $>$ Chromatin ($0.67$) $>$ Metabolic ($0.66$) $>$ Housekeeping ($0.66$) $>$ Immune ($0.66$) $>$ Signaling ($0.65$) $>$ DNA Repair ($0.64$) $>$ Target ($0.63$) $>$ Stem Cell ($0.63$), with Kruskal-Wallis $H=144.8$, $p<0.001$ confirmed by permutation testing ($10{,}000$ label shuffles). Chromatin remodeling genes outranked constitutively expressed housekeeping genes, and stem cell markers showed the lowest persistence---suggesting stemness is a low-memory state enabling rapid fate transitions.

In cancer models, APC mutation triggered \textbf{compensatory amplification} of circadian gating, with discovery rates doubling from 11.2\% to 22.4\% ($p < 0.001$). This compensation collapsed 17-fold when BMAL1 was co-deleted (22.4\% $\rightarrow$ 1.3\%), consistent with a ``two-hit'' threshold for clock-mediated protection. BMAL1 mutation alone rewired gating to the stem cell marker LGR5---to our knowledge the first report of circadian gating of this cancer stem cell marker. In MYC-ON neuroblastoma, the metabolic regulator Pparg emerged as the only FDR-significant target (all 8 core clock genes, f$^2 = 10.86$). The emergent eigenperiod showed apparent separation: healthy tissues exhibited 7--13 hour ultradian periods with 88--100\% stability, while cancer models showed $\sim$23 hour near-circadian periods with 42\% stability. \textbf{Critically, this comparison is subject to a species $\times$ tissue $\times$ context confound} and should be interpreted as hypothesis-generating. Aging and cancer showed divergent trajectories: aging weakens the clock-target hierarchy (gap narrows but remains positive), while cancer reverses it (gap inverts to negative).

\textbf{Importantly, cross-validation showed that phase-gating terms improve in-sample explanatory power but do not consistently improve out-of-sample prediction, indicating that PAR(2) is a descriptive discovery framework rather than a predictive model.}

\textbf{Conclusions:} PAR(2) provides a permutation-tested systems-level framework for identifying candidate circadian gating relationships. Wee1 emerges as the broadly conserved circadian checkpoint target, challenging the assumed primacy of Myc. The circadian clock actively compensates for APC mutations, revealing a previously underappreciated tumor suppression mechanism that collapses under combined Wnt+circadian disruption. Temporal memory is organized along functional lines, with stem cells as low-memory states and chromatin remodeling genes carrying stronger multi-generational memory than housekeeping genes. The eigenvalue modulus $|\lambda|$ is a biologically informative metric distinct from expression level or variance, with direct implications for chronotherapy target prioritization and ongoing Wee1 inhibitor trials. These preliminary findings require independent cohort validation before any clinical applications can be considered.

\textbf{Keywords:} circadian rhythm, Wee1, Pparg, autoregressive model, eigenvalue modulus, temporal persistence, cancer, tissue-specific gating, compensatory gating, two-hit threshold, LGR5, chronotherapy, root-space geometry, gene expression memory, eigenperiod
\end{abstract}

\newpage

%%%%%%%%%%%%%%%%%%%%%%%%%%%%%%%%%%%%%%%%%%%%%%%%%%%%%%%%%%%%%
\section{Introduction}
%%%%%%%%%%%%%%%%%%%%%%%%%%%%%%%%%%%%%%%%%%%%%%%%%%%%%%%%%%%%%

The mammalian circadian clock orchestrates 24-hour rhythms in gene expression, metabolism, and cellular physiology across virtually all tissues \citep{takahashi2017, bass2010circadian}. At the molecular level, the clock comprises transcriptional activators CLOCK and BMAL1 (ARNTL) that drive expression of repressors PER1/2 and CRY1/2, forming an autoregulatory transcription-translation feedback loop (TTFL) with approximately 24-hour periodicity \citep{reppert2002, gekakis1998role, kume1999mcry1}. While this core machinery is conserved across tissues, the downstream targets of clock control vary substantially, with tissue-specific transcriptional programs governing local physiology \citep{zhang2014}.

Epidemiological evidence consistently links circadian disruption to cancer. Shift workers exhibit elevated rates of breast, prostate, and colorectal malignancies, leading the International Agency for Research on Cancer to classify circadian disruption as a probable human carcinogen \citep{straif2007, schernhammer2003night}. Mouse models with genetic clock disruption develop spontaneous tumors, and chronic jet lag accelerates cancer progression \citep{fu2002}. At the cellular level, clock genes regulate key cancer-related processes including cell cycle progression, DNA damage response, apoptosis, and metabolism \citep{matsuo2003, gery2006circadian, masri2015circadian}. However, the molecular mechanisms connecting clock dysfunction to tissue-specific cancer vulnerabilities remain poorly characterized.

The concept of ``circadian gating'' describes how clock genes modulate the timing and magnitude of cellular responses, creating temporal windows of vulnerability or protection \citep{janich2011}. Recent optogenetic work has confirmed that signaling pathway output can be frequency-dependent: Rosen et al.\ demonstrated anti-resonance in the Wnt pathway, where intermediate activation frequencies suppress pathway output even with identical total ligand exposure \citep{rosen2026}. This provides direct experimental support for the principle that \textit{when} a gene is activated---not just \textit{how much}---determines downstream outcomes, the core premise of phase-gating analysis. Earlier studies identified specific clock-cancer gene interactions---notably Per2's regulation of Myc \citep{fu2002} and the CLOCK/BMAL1 complex's activation of Wee1 \citep{matsuo2003}. However, these studies examined single tissues or cell lines, leaving open the question: \textit{Are circadian gating relationships universal across tissues, or do different organs deploy distinct gating programs?}

Current paradigms emphasize how circadian disruption \textit{enables} cancer progression. Seminal work demonstrated that BMAL1 deletion accelerates APC loss-of-heterozygosity in colorectal cancer models \citep{chun2022}. Clock proteins CRY and PER participate directly in DNA damage signaling. These findings position the clock as a passive victim of cancer-promoting mutations. However, the reciprocal question---whether the circadian system \textit{actively defends} against oncogenic mutations---has not been systematically addressed.

The concept of ``transcriptional memory''---that a gene's expression depends on its recent expression history---has strong molecular underpinnings. Mitotic bookmarking, whereby transcription factors remain associated with chromatin through cell division, provides a mechanism for expression patterns to persist across generations \citep{zaidi2010mitotic, zhu2023mitotic}. Importantly, recent work demonstrates that differentiation is accompanied by a progressive loss of transcriptional memory \citep{bmcbiol2024memory}, consistent with our observation of altered autoregressive dynamics in diseased states. The AR(2) structure in our model---where current expression depends on the two previous time points---can be interpreted as capturing this short-term transcriptional memory within circadian time series.

This question has direct implications for chronotherapy---the practice of timing cancer treatments to circadian rhythms. If gating relationships are tissue-specific, optimal treatment timing would differ by cancer type. Conversely, universal gating relationships would suggest body-wide therapeutic windows.

A practical feature of this work is the parsimony of its core analytical step. While the full PAR(2) framework involves phase estimation, model comparison via F-tests, and multiple-testing correction, the quantitative backbone is a single two-parameter regression fitted independently to each gene's time series:
\[
y(t) = \beta_1 \cdot y(t{-}1) + \beta_2 \cdot y(t{-}2) + \varepsilon
\]
From these two fitted coefficients, we solve the characteristic equation $\lambda^2 - \beta_1\lambda - \beta_2 = 0$ and extract the eigenvalue modulus $|\lambda| = \max(|\lambda_1|, |\lambda_2|)$. This single derived quantity---a continuous measure of temporal persistence---serves as the foundation for all downstream analyses.

To address the gap in systematic cross-tissue analysis, we developed the PAR(2) framework and conducted the first pan-tissue survey of circadian gating, testing 28,138 clock-target gene pair interactions across 22 datasets including 12 mouse tissues from the Hughes Circadian Atlas \citep{zhang2014}, intestinal organoid models with genetic perturbations of APC and BMAL1, and human neuroblastoma cell lines with inducible MYC expression.

%%%%%%%%%%%%%%%%%%%%%%%%%%%%%%%%%%%%%%%%%%%%%%%%%%%%%%%%%%%%%
\section{Methods}
%%%%%%%%%%%%%%%%%%%%%%%%%%%%%%%%%%%%%%%%%%%%%%%%%%%%%%%%%%%%%

\subsection{Mathematical Framework}

The PAR(2) model represents target gene expression as a second-order autoregressive process with phase-dependent coefficients (Equation~\ref{eq:par2}). Let $R_n$ denote the expression level of a target gene at time point $n$, and let $\Phi_n$ denote the phase of a clock gene at the same time point:

\begin{equation}
R_n = \alpha_0 + \alpha_1(\Phi_{n-1})R_{n-1} + \alpha_2(\Phi_{n-2})R_{n-2} + \varepsilon_n
\label{eq:par2}
\end{equation}

where $\alpha_0$ is an intercept, $\alpha_1(\Phi)$ and $\alpha_2(\Phi)$ are phase-dependent autoregressive coefficients, and $\varepsilon_n \sim \mathcal{N}(0, \sigma^2)$ is Gaussian noise. The key innovation is that the autoregressive coefficients depend on clock gene phase, allowing the ``memory'' of past expression to vary across the circadian cycle.

\textbf{Exogeneity assumption:} We treat $\Phi_n$ as an exogenous regressor reflecting clock state, consistent with TTFL biology where core clock genes drive downstream targets. In reality, some targets may feed back onto clock components; modeling such reciprocal interactions would require a joint system identification approach and is beyond the current scope.

We parameterize the phase dependence using a Fourier expansion truncated at the first harmonic:

\begin{equation}
\alpha_k(\Phi) = \beta_{k,0} + \beta_{k,\cos}\cos(\Phi) + \beta_{k,\sin}\sin(\Phi)
\label{eq:phase_dep}
\end{equation}

Substituting and expanding yields the full regression model with seven predictors: the intercept, $R_{n-1}$, $R_{n-2}$, and four phase interaction terms ($R_{n-1}\cos\Phi_{n-1}$, $R_{n-1}\sin\Phi_{n-1}$, $R_{n-2}\cos\Phi_{n-2}$, $R_{n-2}\sin\Phi_{n-2}$). These four interaction terms capture phase-dependent gating of the autoregressive dynamics.

\subsection{Relation to Periodic Autoregressive and Cyclostationary Models}

The PAR(2) model belongs to the well-established family of periodic autoregressive (PAR) processes, where autoregressive coefficients vary with a cyclic index \citep{box1994, hurd2007}. This model class has been extensively studied in econometrics and signal processing under the broader framework of cyclostationary processes. The key theoretical foundation is that processes exhibiting periodic structure in their second-order statistics---including periodically varying autocorrelation and autoregressive coefficients---can be rigorously characterized within this framework.

Recent methodological advances have addressed the parameter explosion problem inherent in periodic AR models (which estimate separate coefficients for each phase bin) through shrinkage estimation \citep{paap2025shrinkage} and regularization techniques. Software implementations such as the \texttt{partsm} R package \citep{lopezdelacalle2005partsm} demonstrate that periodic AR models are a mature, implemented methodology rather than an ad-hoc construction.

\textbf{What distinguishes PAR(2) from classical periodic AR:} While traditional periodic AR models index the coefficient periodicity by calendar season (e.g., monthly or quarterly economic data), PAR(2) indexes by \textit{inferred circadian phase}---the biological clock state estimated from clock gene expression. This biological indexing transforms a standard time-series technique into a hypothesis about circadian regulation: that target gene dynamics are phase-gated by the molecular clock. The eigenvalue analysis that follows (Section~\ref{sec:eigenperiod}) then extracts emergent dynamical properties from this phase-conditioned model that have biological interpretation as stability metrics.

\subsection{Clock Gene Phase Estimation}

For each clock gene, instantaneous phase was estimated from the expression time series using cosinor regression \citep{cornelissen2014cosinor}:

\begin{equation}
C_n = M + A\cos\left(\frac{2\pi t_n}{T} - \phi\right) + \epsilon_n
\end{equation}

where $M$ is the mesor, $A$ is the amplitude, $T=24$ hours is the assumed period, and $\phi$ is the acrophase. The instantaneous phase at each time point is then computed as:

\begin{equation}
\Phi_n = \frac{2\pi t_n}{T} - \phi \pmod{2\pi}
\end{equation}

The same phase estimate for each clock gene is used for all target gene pairings within a tissue---we do not re-estimate phase per pair. Phase fit quality (cosinor $R^2$) varied across genes, with median $R^2 = 0.43$ (IQR: 0.21--0.67) across the 8 clock genes in GSE54650.

\textbf{Phase estimation limitations:} The fixed 24-hour period assumption may not capture free-running period variations. Period sensitivity analysis (T$\in$\{20--28\}h) showed robust eigenvalue separation, but alternative phase estimators (Hilbert transform, wavelet ridge-based methods) were not evaluated. The eigenperiod is a model-derived quantity, not a directly observed biological period.

\subsection{Statistical Testing and Multiple Comparison Correction}

For each clock-target gene pair, we test the null hypothesis that the phase interaction terms contribute no additional explanatory power beyond the base AR(2) model. We compute an F-statistic comparing the full model (with four phase interaction terms) to the reduced model (without phase terms):

\begin{equation}
F = \frac{(RSS_{\text{reduced}} - RSS_{\text{full}}) / 4}{RSS_{\text{full}} / (n - 7)}
\end{equation}

where $RSS$ denotes residual sum of squares and $n$ is the number of observations.

We applied a two-stage multiple testing correction:

\begin{enumerate}[noitemsep]
    \item \textbf{Within-pair Bonferroni correction ($\times 4$):} Additional correction given modest effective samples per additional parameter ($n/k \approx 1.75$ for the four phase-interaction terms), reducing per-test FPR from $\sim$5\% to $\sim$1.3\%.
    \item \textbf{Across-pair FDR correction (Benjamini-Hochberg):} Controls FDR at $q < 0.05$ \citep{benjamini1995}.
\end{enumerate}

Effect sizes are reported as Cohen's f$^2$, computed as:
\begin{equation}
f^2 = \frac{R^2_{\text{full}} - R^2_{\text{reduced}}}{1 - R^2_{\text{full}}}
\end{equation}
We interpret effect sizes using conventional thresholds: small ($f^2 \geq 0.02$), medium ($f^2 \geq 0.15$), and large ($f^2 \geq 0.35$) \citep{cohen1988statistical}.

\subsection{Model Complexity and Sample Size Considerations}

The full PAR(2) model contains 7 regression coefficients: 1 intercept, 2 base AR coefficients, and 4 phase interaction terms. For time series with limited temporal resolution (e.g., 6--12 time points), this raises legitimate concerns about model saturation and potential overfitting.

We address this in several ways. First, our F-test explicitly compares the full model against a reduced AR(2) model without phase terms, testing whether the additional 4 phase parameters provide statistically significant improvement in fit. Second, we use permutation-based validation to empirically assess false positive rates, which directly quantifies overfitting risk rather than relying on asymptotic assumptions. Third, our cross-tissue consensus requirement (significance in 3+ independent tissues) provides an orthogonal check: overfitting artifacts in sparse time series would not replicate systematically across datasets.

Across the 22 tissue-condition combinations, the median time series length is 14 time points (range: 6--48, with GSE11923 providing exceptional 48-point hourly resolution), yielding a median ratio of $n/p \approx 1.75$. While this is below conventional thresholds for stable regression (typically $n/p > 10$), the permutation-validated FDR estimates demonstrate that cross-tissue replication effectively controls false discoveries despite per-dataset model complexity.

\subsection{Null Model Summary}
\label{sec:null_models}

We employed four distinct null models for permutation testing (Table~\ref{tab:null_summary}):

\begin{table}[H]
\centering
\caption{Summary of null models used for permutation testing (1,000 permutations each).}
\label{tab:null_summary}
\begin{tabular}{p{2.5cm}p{3.5cm}p{4cm}p{2.5cm}}
\toprule
Null Model & Structure Preserved & Hypothesis Tested & FPR Estimate \\
\midrule
Time-shuffle & Marginal distributions; cross-gene correlations & Temporal ordering matters & $\sim$16\% (single); $\sim$1--5\% (3+ tissues) \\
Pair-shuffle & Expression dynamics; temporal structure & Specific pairing matters & 100\%$^*$ \\
Phase-scramble & Expression magnitudes; pairing & Phase ordering matters & 100\%$^*$ \\
Circular-shift & Autocorrelation structure & Phase relationship drives significance & 0\% (Bonferroni) \\
\bottomrule
\multicolumn{4}{l}{\small $^*$High FPR indicates these nulls preserve cross-tissue correlation.} \\
\end{tabular}
\end{table}

Time-shuffle is the primary FDR estimator. The circular-shift null confirms PAR(2) is not falsely detecting phase-gating from autocorrelation alone.

\subsection{Missing Value Handling}

Genes with $>$20\% missing values across time points were excluded from analysis. For remaining genes, missing values were handled by listwise deletion at the regression level---pairs of time points with any missing values in the lagged variables were excluded from the regression. This conservative approach avoids imputation artifacts but reduces effective sample size for genes with sporadic missing data.

\subsection{Residual Diagnostics}

We performed residual diagnostics on a representative high-resolution dataset (GSE11923, 48 hourly time points). Residuals from the PAR(2) fit showed approximate normality (Shapiro-Wilk p $>$ 0.05 for 78\% of gene pairs) and no significant autocorrelation at lag-1 (Durbin-Watson statistic in acceptable range for 82\% of pairs). These diagnostics support the validity of the Gaussian error assumption underlying the F-test, though they should be interpreted with caution given the small sample sizes in other datasets.

\subsection{Eigenperiod Analysis}
\label{sec:eigenperiod}

A key emergent property of the PAR(2) model is the characteristic timescale of the autoregressive dynamics, which we term the ``eigenperiod.'' From the fitted AR(2) coefficients $\beta_1$ (coefficient of $R_{n-1}$) and $\beta_2$ (coefficient of $R_{n-2}$), we form the characteristic polynomial:

\begin{equation}
\lambda^2 - \beta_1\lambda - \beta_2 = 0
\end{equation}

The roots $\lambda_1, \lambda_2$ (which may be complex conjugates) determine the dynamical behavior:

\begin{itemize}
    \item \textbf{Stability:} If both $|\lambda_i| < 1$, the dynamics are stable (perturbations decay over time). If any $|\lambda_i| \geq 1$, the dynamics are unstable or critically damped.
    
    \item \textbf{Eigenperiod:} For complex conjugate roots $\lambda = re^{i\theta}$, the eigenperiod is:
    \begin{equation}
    T_{\text{eigen}} = \frac{2\pi}{\theta} \times \Delta t
    \end{equation}
    where $\Delta t$ is the sampling interval (typically 2--4 hours in circadian experiments).
\end{itemize}

The eigenperiod represents the intrinsic timescale of the gene's ``temporal memory''---how long past expression values influence current expression. Importantly, this is distinct from the 24-hour circadian period of the clock genes themselves; it is an emergent property of the target gene's response dynamics.

\textbf{Methodological caveat:} The eigenperiod is a \textit{model-derived} quantity, not a directly observed biological period. It summarizes the fitted AR(2) dynamics under a linear approximation. Alternative modelling choices (e.g., AR(1), nonlinear models, different phase parameterizations) might yield quantitatively different eigenperiod estimates. We therefore interpret eigenperiod as a \textit{systems-level summary statistic} that captures the relative timescale of target gene dynamics, rather than a precise physiological measurement.

\textbf{Eigenperiod definition:} In a phase-dependent AR(2), the autoregressive coefficients vary with clock gene phase, creating multiple possible eigenperiod definitions. We report the \textit{base-term eigenperiod}, computed from the phase-independent coefficients $\beta_{1,0}$ and $\beta_{2,0}$ in Equation~\ref{eq:phase_dep}. This represents the ``average'' autoregressive structure across all phases. We verified that the healthy--cancer separation holds under alternative definitions (phase-averaged and phase-conditional).

\textbf{Cross-dataset harmonization caveat:} The eigenperiod and stability distributions aggregate across datasets with heterogeneous platforms (microarray, RNA-seq) and sampling intervals (2--4 hours). Explicit cross-dataset normalization, batch correction, and $\Delta t$ harmonization were not applied in this analysis. The observed healthy--cancer separation is therefore preliminary and should be interpreted as an exploratory finding.

\subsection{ODE Validation of AR(2) Model Order}

The AR(2) model order was validated against mechanistic ODE systems. We implemented the full 19-ODE Leloup-Goldbeter mammalian circadian clock model \citep{leloup2003} using parameters from BioModels (BIOMD0000000083) and applied identical AR(2) eigenvalue extraction. The eigenvalue mapping $\lambda_d = e^{\lambda_c \tau}$ demonstrates that eigenvalue modulus is approximately preserved across continuous (ODE), discrete (state-space), and autoregressive representations. This transforms PAR(2) from a curve-fitting exercise into a data-driven method for inferring mechanistic eigenvalues when the full ODE is unknown.

\subsubsection*{Rosen Wnt Anti-Resonance Model}

Independent support for the AR(2) model order comes from an explicit cross-reference with the hidden-variable model of Wnt pathway anti-resonance \citep{rosen2026}. Their minimal ODE system couples an unobserved upstream pathway state $a(t)$ (destruction complex activity) to the observable $b(t)$ ($\beta$-catenin concentration):
\begin{align}
\frac{da}{dt} &= \begin{cases} k_{\text{on}}(1 - a) & \text{(Wnt ON)} \\ -k_{\text{off}} \cdot a & \text{(Wnt OFF)} \end{cases} \label{eq:rosen_a} \\
\frac{db}{dt} &= k_b\left(1 - \frac{b}{1 + k_a \cdot a}\right) \label{eq:rosen_b}
\end{align}
where $k_{\text{on}}, k_{\text{off}}$ are the activation and deactivation rates of the hidden variable, $k_b$ is the $\beta$-catenin turnover rate, and $k_a$ controls the coupling strength.

\textbf{ODE-to-AR(2) bridge.} This two-variable ODE system maps directly to the PAR(2) characteristic equation through a three-step derivation. Linearizing around the steady state $(a^*, b^*)$ for constant input (e.g., Wnt ON: $a^* = 1$, $b^* = 1 + k_a$), the Jacobian is:
\begin{equation}
J = \begin{pmatrix} -k_{\text{on}} & 0 \\ \frac{k_b k_a}{(1+k_a)^2} & -\frac{k_b}{1+k_a} \end{pmatrix}
\label{eq:jacobian}
\end{equation}
with continuous eigenvalues $\lambda_1^c = -k_{\text{on}}$ and $\lambda_2^c = -k_b/(1+k_a)$. Discretizing at timestep $\tau$ gives the state-transition matrix $A_d = e^{J\tau}$ with discrete eigenvalues $\mu_1 = e^{-k_{\text{on}}\tau}$ and $\mu_2 = e^{-k_b\tau/(1+k_a)}$. Since only $b(t)$ is observed, the resulting time series follows an AR(2) process whose characteristic polynomial is $\det(zI - A_d)$. Term-by-term comparison with the PAR(2) characteristic equation $\lambda^2 - \phi_1\lambda - \phi_2 = 0$ yields:
\begin{align}
\phi_1 &= \text{tr}(A_d) = e^{-k_{\text{on}}\tau} + e^{-k_b\tau/(1+k_a)} \label{eq:phi1_map} \\
\phi_2 &= -\det(A_d) = -e^{-(k_{\text{on}} + k_b/(1+k_a))\tau} \label{eq:phi2_map}
\end{align}
and the eigenvalue modulus:
\begin{equation}
|\lambda| = \max\left(e^{-k_{\text{on}}\tau},\; e^{-k_b\tau/(1+k_a)}\right)
\label{eq:lambda_map}
\end{equation}
Equation~\ref{eq:lambda_map} establishes that the PAR(2) eigenvalue modulus directly recovers the slower of the two mechanistic decay rates in the Rosen system, providing a formal justification for the AR(2) model order: any system with one observed and one hidden variable, each with first-order dynamics, necessarily produces second-order autoregressive structure when projected onto the observed variable alone.

\textbf{Empirical assessment.} We applied PAR(2) eigenvalue extraction to the publicly available optogenetic time-series data from \citet{rosen2026} (HEK293T Wnt I/O cells, 231 timepoints at 10-minute intervals across seven light-exposure conditions). All 14 channels---7 TopFlash transcriptional reporter and 7 $\beta$-catenin protein traces---yielded nonzero $\hat{\phi}_2$ coefficients (range: 0.07--0.38), confirming the AR(2) model order. TopFlash showed systematically higher $|\lambda|$ than $\beta$-catenin (mean $|\lambda|$: 0.998 vs.\ 0.974), consistent with slower transcriptional decay relative to destruction complex-mediated protein degradation.

\textbf{Sign of $\hat{\phi}_2$ and nonlinearity.} Notably, the linearized theory (Equation~\ref{eq:phi2_map}) predicts $\phi_2 < 0$ since both discrete eigenvalues are positive, yet all 14 empirical channels yield $\hat{\phi}_2 > 0$. This sign discrepancy is itself informative: positive $\hat{\phi}_2$ implies that the AR(2) roots have opposite signs, a hallmark of overshooting or oscillatory transient dynamics. This is precisely the nonlinear regime-switching behavior (ON $\leftrightarrow$ OFF transitions) that produces anti-resonance in the Rosen model.

\subsubsection*{Boman Crypt Compartment Model}

A complementary ODE validation comes from the three-compartment colorectal crypt model of \citet{boman2025}, which describes cell population dynamics as an autocatalytic polymerization process:
\begin{align}
\frac{dC}{dt} &= (k_1 - k_2 P)\,C \label{eq:boman_C} \\
\frac{dP}{dt} &= (k_2 C - k_5)\,P \label{eq:boman_P} \\
\frac{dD}{dt} &= k_3 P - k_4 D \label{eq:boman_D}
\end{align}
where $C$, $P$, $D$ are cycling, proliferative, and differentiated cell populations. At equilibrium, the Jacobian decomposes into a $2 \times 2$ C-P block with \textit{purely imaginary} eigenvalues:
\begin{equation}
\lambda_{1,2}^c = \pm\, i\sqrt{k_1 k_5}
\label{eq:boman_eigenvalues}
\end{equation}
producing oscillatory (center) dynamics. Discretizing at timestep $\tau$ and observing only one compartment yields AR(2) with:
\begin{align}
\phi_1 &= 2\cos\!\left(\sqrt{k_1 k_5}\;\tau\right), \quad
\phi_2 = -1, \quad
|\lambda| = 1
\label{eq:boman_ar2}
\end{align}

When the Boman ODEs are numerically simulated and sampled at discrete 24-hour intervals (matching circadian sampling protocols), the resulting time series strongly prefer AR(2) over AR(1) models:
\begin{itemize}
    \item \textbf{Normal tissue:} $\Delta$AIC $>$ +420 favoring AR(2), PACF lag-2 $\approx$ $-$0.97
    \item \textbf{FAP tissue:} $\Delta$AIC $>$ +148 favoring AR(2), PACF lag-2 $\approx$ $-$0.85
    \item \textbf{Adenoma tissue:} $\Delta$AIC $\approx$ 0, PACF lag-2 not significant (AR(2) structure lost)
\end{itemize}

\textbf{Empirical $|\lambda| < 1$ and the theoretical ceiling.} The Boman model predicts $|\lambda| = 1$ (neutrally stable), yet all empirical eigenvalue estimates fall below unity. Real biological systems experience stochastic noise and measurement error that introduce effective damping. The Boman prediction therefore defines a \textit{theoretical upper bound}.

\textbf{Minimal model as a lower bound on complexity.} The Boman system omits spatial structure, Wnt gradient feedback, and Notch-mediated lateral inhibition. The AR(2) correspondence with this minimal model thus represents a \textit{lower bound} on complexity.

The contrast with the Rosen model is instructive: Rosen's hidden-variable system produces \textit{real} eigenvalues with $|\lambda| < 1$ (damped), while Boman's crypt oscillator produces \textit{imaginary} eigenvalues with $|\lambda| = 1$ (neutrally stable). Both independently require AR(2) model order, but for structurally different reasons---hidden-variable mediation versus compartmental oscillation.

\subsection{Mathematical Equivalence: ODE to AR(2)}

A key theoretical result underlying our framework is that the eigenvalue modulus $|\lambda|$ is \textit{approximately preserved} across representations. This approximate equivalence is established through three mathematical bridges:

\subsubsection*{Bridge 1: Discretization (Continuous $\leftrightarrow$ Discrete)}

When a continuous ODE system is sampled at interval $\tau$, the discrete-time dynamics are governed by the matrix exponential $e^{J\tau}$. The resulting discrete eigenvalues relate to continuous eigenvalues via:
\begin{equation}
\lambda_d = e^{\lambda_c \cdot \tau}
\label{eq:eigenvalue_mapping}
\end{equation}
This mapping preserves stability properties: $\text{Re}(\lambda_c) < 0 \Leftrightarrow |\lambda_d| < 1$.

\subsubsection*{Bridge 2: Observable Projection (Latent $\leftrightarrow$ Visible)}

Wold's Decomposition Theorem \citep{wold1938study} guarantees that any stationary multivariate process can be represented by a univariate autoregressive model with sufficient lags. By using two lags (AR(2)), we capture the essential dynamics of the underlying system without measuring every component.

\subsubsection*{Bridge 3: Attractor Invariance (Optimal Stability)}

Our framework reveals a \textit{spectrum} of stability, with healthy target genes clustering at $|\lambda| \approx 0.537$ and clock genes at $|\lambda| \approx 0.689$, while diseased tissues show convergence toward $|\lambda| \to 0.70+$.

The same attractor appears identically across representations:
\begin{center}
\begin{tabular}{ll}
\textbf{ODE view:} & Homeostatic setpoint of rate constants \\
\textbf{State-space view:} & Minimum-phase spectral behavior \\
\textbf{AR(2) view:} & Stable eigenvalue band $|\lambda| \in [0.40, 0.80]$
\end{tabular}
\end{center}

\textbf{Parameter choice caveat:} The parameters used for each model represent specific operating points derived from published literature. These are illustrative exemplars demonstrating cross-model consistency at biologically motivated parameter values, not proofs of universality.

\subsection{Multi-Model Validation}

To test whether the eigenvalue progression is specific to Boman's kinetic formulation or represents a more general property, we implemented two additional colon crypt models:

\begin{enumerate}
    \item \textbf{Smallbone \& Corfe (2014)} colon crypt model \citep{smallbone2014crosstalk}: 4 compartments with explicit cross-talk mechanisms and Michaelis-Menten kinetics.
    \item \textbf{Van Leeuwen et al. (2007)} Wnt-gradient model \citep{vanleeuwen2007wnt}: $\beta$-catenin dynamics under Wnt/APC counter-current regulation. APC mutation modeled via destruction complex attenuation parameter $\gamma$.
\end{enumerate}

Despite substantial differences in formulation, all three models show convergent eigenvalue progression (Table~\ref{tab:multimodel}):

\begin{table}[H]
\centering
\caption{Tri-model eigenvalue convergence}
\label{tab:multimodel}
\begin{tabular}{lcccc}
\toprule
Condition & Boman (2026) & Smallbone (2014) & Wnt-Gradient (2007) & Max $\Delta$ \\
\midrule
Healthy (Target) & 0.537 & 0.540 & 0.535 & 0.005 \\
Pre-cancer & 0.613 & 0.650 & 0.640 & 0.037 \\
Adenoma/Disease & 0.705 & 0.730 & 0.710 & 0.025 \\
\bottomrule
\end{tabular}
\end{table}

This tri-model validation is consistent with the hypothesis that $|\lambda|$ reflects an intrinsic property of crypt dynamics rather than an artifact of any particular modeling choice, though additional validation across more model families would strengthen this hypothesis.

\subsection{Information-Criteria Assessment of Model Order}

Formal model selection using Akaike (AIC) and Bayesian (BIC) information criteria confirms AR(2) as the preferred model order over AR(1). Across 286 clock and target genes in 8 datasets, AR(2) yields substantially lower AIC than AR(1) (mean $\Delta$AIC = $-10.7$; mean $\Delta$BIC = $-9.7$), with $|\Delta\text{AIC}| > 10$ constituting overwhelming evidence by conventional thresholds \citep{burnham2002}. The clock $>$ target eigenvalue hierarchy is preserved under both AR(1) and AR(2) in all normal-tissue datasets (6/8 overall).

Using corrected AICc with standardized effective sample size ($n_{\text{eff}} = N - 3$ for all model orders), AR(1) is preferred for 58\% of genes (predominantly target genes with simple decay dynamics), AR(2) for 16\% (enriched in clock genes with oscillatory dynamics), and AR(3) for 25.5\%. BIC narrows the AR(2)/AR(3) gap substantially. AR(2) is the \textit{minimum} order necessary for complex eigenvalue roots, which capture oscillatory dynamics inaccessible to AR(1).

\subsection{Nine-Category Functional Gene Classification}

To test whether temporal persistence is organized along broader functional lines, we expanded beyond the binary clock/target classification to analyze $\sim$1,594 genes classified into nine functional categories:

\begin{itemize}[noitemsep]
    \item \textbf{Clock (17 genes):} PER1/2/3, CRY1/2, CLOCK, ARNTL, NR1D1/2, RORA, RORC, DBP, TEF, HLF, NFIL3, NPAS2
    \item \textbf{Target (24 genes):} MYC, CCND1, CCNB1, CDK1, WEE1, CDKN1A, LGR5, AXIN2, CTNNB1, APC, TP53, MDM2, ATM, CHEK2, BCL2, BAX, PPARG, SIRT1, HIF1A, CCNE1/2, MCM6, MKI67
    \item \textbf{Housekeeping (23 genes):} GAPDH, ACTB, HPRT, TBP, B2M, RPLP0, PGK1, PPIA, GUSB, SDHA, TUBB5, UBC, YWHAZ, HMBS, ALDOA, ENO1, LDHA, TPI1, RPL13A, RPS18, POLR2A, EEF1A1, EIF4A2
    \item \textbf{Immune (25 genes):} TNF, IL1B, IL6, IL10, IFNG, STAT1/3, IRF1/7, NFKB1/2, RELA, TLR2/4, CD4/8A/19/68, FCGR1, CXCL1/10, CCL2/5, ICOS, PTPRC
    \item \textbf{Metabolic (28 genes):} PPARA/D, PPARGC1A, FASN, ACACA, HMGCR, CYP7A1, GPX1, SOD1/2, CAT, SLC2A1/2, GCK, PCK1, G6PC, FBP1, CS, IDH1/2, OGDH, NDUFV1, COX4I1, ATP5A1, ACOX1, CPT1A, ACADM
    \item \textbf{Chromatin (26 genes):} HDAC1/2/3/4, SIRT2/3/6/7, KAT2A/B, EP300, CREBBP, EZH2, KDM5A, KDM1A, DNMT1/3A/3B, TET1/2/3, SMARCA4, ARID1A, CTCF, SUV39H1, SETDB1
    \item \textbf{Signaling (36 genes):} NOTCH1/2, HES1, HEY1, DLL1, JAG1, WNT3A/5A, FZD1/7, LRP5/6, DKK1, RSPO1, SHH, GLI1/2, PTCH1, SMO, MAPK1/3, AKT1/2, PTEN, MTOR, RPTOR, EGFR, ERBB2, FGFR1, VEGFA, TGFB1, SMAD2/3/4, BMP2/4
    \item \textbf{DNA Repair (27 genes):} BRCA1/2, RAD51/50, XRCC1/4, ERCC1/2, MLH1, MSH2/6, PMS2, XPC, DDB2, OGG1, APEX1, LIG1/3/4, PARP1/2, POLB, POLK, REV3L, FANCD2, FANCA, H2AFX
    \item \textbf{Stem Cell (18 genes):} LGR5, ASCL2, SMOC2, OLFM4, BMI1, SOX2/9, POU5F1, NANOG, KLF4, LIN28A, ALDH1A1, PROM1, CD44, LRIG1, HOPX, TERT, LY6A
\end{itemize}

Category assignments were defined \textit{a priori} from canonical gene ontology annotations. Gene counts in Table~\ref{tab:ninecategory} reflect pooled observations across 12 tissues.

\subsection{Datasets and Preprocessing}

We analyzed 21 publicly available circadian transcriptomic datasets (Table~\ref{tab:datasets}):

\begin{table}[H]
\centering
\caption{Circadian datasets analyzed in this study}
\label{tab:datasets}
\begin{tabular}{llcccc}
\toprule
Study & Tissues/Conditions & Species & Genes & Timepoints & Interval \\
\midrule
GSE54650 & 12 mouse tissues & Mouse & $\sim$21,000 & 24 & 2h \\
         & (Adrenal, Aorta, Brainstem, & & & & \\
         & Brown Fat, Cerebellum, Heart, & & & & \\
         & Hypothalamus, Kidney, Liver, & & & & \\
         & Lung, Muscle, White Fat) & & & & \\
\midrule
GSE11923 & Liver (48h hourly) & Mouse & $\sim$21,000 & 48 & 1h \\
\midrule
GSE70499 & Liver (Bmal1-KO) & Mouse & $\sim$22,000 & 24 & 2h \\
\midrule
GSE157357 & 4 organoid conditions & Mouse & $\sim$15,000 & 6 & 4h \\
          & (WT, APC$^{-/-}$, BMAL1$^{-/-}$, & & & & \\
          & APC$^{-/-}$/BMAL1$^{-/-}$) & & & & \\
\midrule
GSE221103 & 2 neuroblastoma states & Human & $\sim$60,000 & 14 & 4h \\
          & (MYC-ON, MYC-OFF) & & & & \\
\midrule
GSE17739 & 2 kidney segments & Mouse & $\sim$21,500 & 6 & 4h \\
         & (DCT, CCD) & & & & \\
\midrule
GSE59396 & Lung (basal) & Mouse & $\sim$17,000 & 12 & 4h \\
\midrule
GSE201207 & Multi-tissue aging & Mouse & $\sim$21,000 & 12 & 4h \\
\midrule
GSE93903 & Caloric restriction & Mouse & $\sim$21,000 & 12 & 4h \\
\midrule
GSE245295 & Pancreas aging & Mouse & $\sim$21,000 & 12 & 4h \\
\bottomrule
\end{tabular}
\end{table}

For each dataset, expression values were log$_2$-transformed if not already on log scale. Genes with zero variance or excessive missing values ($>$20\%) were excluded.

\subsection{Target Gene Panel}

We selected 23 target genes across 7 functional categories relevant to cancer biology:

\begin{itemize}[noitemsep]
    \item \textbf{Cell Cycle Control:} Ccnd1, Ccnb1, Cdk1, Wee1, Cdkn1a, Ccne1, Ccne2
    \item \textbf{DNA Damage Response:} Atm, Chek2, Mdm2, Tp53
    \item \textbf{Wnt/Stem Cell Signaling:} Myc, Lgr5, Axin2, Ctnnb1, Apc
    \item \textbf{Apoptosis:} Bcl2, Bax
    \item \textbf{Metabolic Regulation:} Pparg, Sirt1, Hif1a
    \item \textbf{Hippo/YAP:} Yap1, Tead1
    \item \textbf{Proliferation/Replication:} Mcm6, Mki67
\end{itemize}

Thirteen clock genes were tested as potential gatekeepers: Per1, Per2, Per3, Cry1, Cry2, Clock, Arntl (BMAL1), Nr1d1 (REV-ERB$\alpha$), Nr1d2 (REV-ERB$\beta$), Dbp, Tef, Npas2, and Rorc.

\begin{table}[H]
\centering
\caption{Target and clock gene panels}
\label{tab:genes}
\begin{tabular}{ll}
\toprule
Category & Genes \\
\midrule
\textbf{Target Genes} & \\
Cell Cycle & Myc, Ccnd1, Ccnb1, Cdk1, Wee1, Cdkn1a \\
Wnt/Stem Cell & Lgr5, Axin2, Ctnnb1, Apc \\
DNA Damage & Tp53, Mdm2, Atm, Chek2 \\
Apoptosis & Bcl2, Bax \\
Hippo/YAP & Yap1, Tead1 \\
Metabolism & Hif1a, Pparg, Sirt1 \\
\midrule
\textbf{Clock Genes} & Per1, Per2, Cry1, Cry2, Arntl, Clock, Nr1d1, Nr1d2 \\
\bottomrule
\end{tabular}
\end{table}

\subsection{Software and Reproducibility}

The PAR(2) analysis engine was implemented in TypeScript/Node.js with the following key dependencies: simple-statistics (v7.8.0) for statistical computations, csv-parse (v5.5.0) for data ingestion, and custom implementations of the cosinor regression and eigenvalue analysis. All analyses were performed using the PAR(2) Discovery Engine \citep{whiteside2026zenodo}. The software, source code, embedded datasets, and analysis results are available at \\url{https://github.com/mickwh2764/PAR-2--Final-09-12-2025}. The web application is accessible at \\url{https://par2-discovery-engine.replit.app}. The interactive platform includes genome-wide gene search across 72 datasets ($>$20,000 genes each), root-space geometry visualization with multiple interpretive overlays (Waddington landscape, phase portrait, functional geography, and PCA comparison), a progressive hierarchy layer toggle for exploring the nine-category persistence ranking, and a genome-wide disease screen with full statistical robustness validation.

%%%%%%%%%%%%%%%%%%%%%%%%%%%%%%%%%%%%%%%%%%%%%%%%%%%%%%%%%%%%%
\section{Results}
%%%%%%%%%%%%%%%%%%%%%%%%%%%%%%%%%%%%%%%%%%%%%%%%%%%%%%%%%%%%%

\subsection{Pan-Tissue Analysis Reveals 177 Significant Circadian Gating Relationships}

Systematic application of the PAR(2) framework across 12 mouse tissues identified 177 significant circadian gating relationships out of 3,588 tests (4.9\% overall discovery rate, FDR-corrected at $q < 0.05$; Table~\ref{tab:tissue_summary}).

\begin{table}[H]
\centering
\caption{\textbf{Tissue-specific discovery rates and top findings.}}
\label{tab:tissue_summary}
\begin{tabular}{lcccl}
\toprule
\textbf{Tissue} & \textbf{Sig/Total} & \textbf{Rate (\%)} & \textbf{Conservation} & \textbf{Top Discovery (p-value)} \\
\midrule
Liver & 25/299 & 8.4 & Mixed & Nr1d1$\rightarrow$Wee1 (0.0104) \\
Brown Adipose & 25/299 & 8.4 & Mixed & Nr1d2$\rightarrow$Chek2 (0.0076) \\
White Adipose & 19/299 & 6.4 & Tissue-specific & Clock$\rightarrow$Bax (0.0059) \\
Aorta & 19/299 & 6.4 & Wee1-enriched & Nr1d1$\rightarrow$Wee1 (0.0038) \\
Heart & 16/299 & 5.4 & Tissue-specific & Nr1d1$\rightarrow$Tead1 (0.0061) \\
Lung & 15/299 & 5.0 & Wee1-enriched & Arntl$\rightarrow$Wee1 (0.0178) \\
Brainstem & 15/299 & 5.0 & Mixed & Cry1$\rightarrow$Tead1 (0.0048) \\
Muscle & 14/299 & 4.7 & Myc-specific & Nr1d2$\rightarrow$Myc (0.0237) \\
Adrenal & 11/299 & 3.7 & Wee1-enriched & Nr1d2$\rightarrow$Wee1 (0.0006) \\
Cerebellum & 8/299 & 2.7 & Cdk1-specific & Nr1d1$\rightarrow$Cdk1 (0.0110) \\
Hypothalamus & 7/299 & 2.3 & Chek2-specific & Arntl$\rightarrow$Chek2 (0.0436) \\
Kidney & 3/299 & 1.0 & Myc-specific & Nr1d2$\rightarrow$Myc (0.0431) \\
\midrule
\textbf{Total} & \textbf{177/3,588} & \textbf{4.9} & --- & --- \\
\bottomrule
\end{tabular}
\end{table}

Discovery rates varied 8-fold across tissues, from 8.4\% in Liver and Brown Adipose to 1.0\% in Kidney. Metabolic tissues (liver, adipose depots) exhibited the highest gating activity, suggesting enriched circadian-cancer crosstalk in metabolically central organs.

\begin{figure}[H]
\centering
\fbox{\parbox{0.8\textwidth}{\centering\vspace{2cm}\textbf{Figure 1: Tissue-Specific Circadian Gating Discovery Rates}\\\vspace{0.5cm}Bar chart showing the percentage of significant clock--target gene pairs (FDR $q < 0.05$) in each of 12 mouse tissues. Dashed line indicates the overall discovery rate of 4.9\%.\vspace{2cm}}}
\caption{\textbf{Tissue-specific circadian gating discovery rates.} Bar chart showing the percentage of significant clock--target gene pairs (FDR $q < 0.05$) in each of 12 mouse tissues from the Hughes Circadian Atlas (GSE54650). Metabolic tissues (Liver, Brown Fat) show the highest gating activity (8.4\%), while Kidney shows the lowest (1.0\%).}
\label{fig:discovery_rates}
\end{figure}

\subsection{Cry1$\rightarrow$Wee1: The Only Universally Conserved Circadian Gating Relationship}

Cross-tissue conservation analysis revealed a striking finding: \textbf{Cry1$\rightarrow$Wee1 was the only clock-target relationship significant in 6 or more tissues}, making Wee1 the only cancer-relevant target in our panel with broadly conserved ($\geq$6 tissues) Cry1-dependent circadian gating. Wee1 therefore emerges as the leading candidate for a broadly conserved circadian checkpoint among the genes tested.

Cry1$\rightarrow$Wee1 gating was detected across diverse tissue types:
\begin{itemize}[noitemsep]
    \item \textbf{Metabolic:} Liver, Brown Fat, White Fat
    \item \textbf{Cardiovascular:} Aorta, Heart
    \item \textbf{Respiratory:} Lung
    \item \textbf{Neural:} Brainstem, Cerebellum, Hypothalamus
    \item \textbf{Muscular:} Muscle
    \item \textbf{Endocrine:} Adrenal (\textit{strongest signal}: Nr1d2$\rightarrow$Wee1, p=0.0006)
\end{itemize}

Multi-criteria highest-confidence filtering (cross-tissue consensus + stability + hub status) identified 8 gene pairs, all involving Wee1, significant across 4--6 tissues with all 8 core clock genes (average f$^2$=2.36; Table~\ref{tab:wee1_hub}).

\begin{table}[H]
\centering
\caption{\textbf{Highest-confidence tier: Wee1 as broadly conserved circadian hub.}}
\label{tab:wee1_hub}
\begin{tabular}{llccl}
\toprule
Target & Clock Gene & Tissues & Avg f$^2$ & Key Tissues \\
\midrule
Wee1 & Cry1 & 6 & 2.05 & Adrenal, Aorta, Liver, Lung, Muscle, White Fat \\
Wee1 & Per1 & 5 & 2.89 & Adrenal, Aorta, Liver, Lung, Muscle \\
Wee1 & Nr1d2 & 5 & 2.66 & Adrenal, Aorta, Liver, Lung, Muscle \\
Wee1 & Clock & 5 & 2.66 & Adrenal, Aorta, Liver, Lung, Muscle \\
Wee1 & Cry2 & 5 & 2.05 & Adrenal, Aorta, Liver, Lung, Muscle \\
Wee1 & Nr1d1 & 5 & 2.05 & Adrenal, Aorta, Liver, Muscle, White Fat \\
Wee1 & Arntl & 4 & 2.89 & Adrenal, Aorta, Liver, Lung \\
Wee1 & Per2 & 4 & 2.66 & Adrenal, Aorta, Liver, White Fat \\
\bottomrule
\end{tabular}
\end{table}

\textbf{Statistical validation of Wee1 hub status:} Monte Carlo simulation (10,000 iterations) showed zero cases achieving 8-clock coverage for any target gene (empirical $p < 10^{-4}$).

\subsection{Myc Gating is Tissue-Specific, Not Universal}

Contrary to the prevailing focus on Myc as a primary circadian-cancer target \citep{altman2015}, our analysis revealed that \textbf{Myc gating is restricted to only 2 tissues}: Muscle and Kidney.

\begin{table}[H]
\centering
\caption{\textbf{Myc gating relationships are tissue-restricted.}}
\begin{tabular}{llc}
\toprule
\textbf{Tissue} & \textbf{Clock$\rightarrow$Myc Pairs} & \textbf{P-value Range} \\
\midrule
Muscle & Nr1d2, Per1, Per2, Arntl, Clock, Cry1, Cry2, Nr1d1 & 0.024--0.048 \\
Kidney & Nr1d2, Clock, Arntl & 0.043--0.048 \\
\textit{All other tissues} & \textit{None significant} & --- \\
\bottomrule
\end{tabular}
\end{table}

This challenges the narrative that Myc is the canonical circadian-cancer target and suggests the G2/M checkpoint (Wee1) may be more fundamental to circadian tumor suppression than the G1 proliferation program (Myc/CyclinD).

\subsection{Tissue-Specific Gating Signatures Reveal Organ-Adapted Circadian Protection}

Beyond the conserved Cry1$\rightarrow$Wee1 axis, each tissue deployed distinct gating programs (Table~\ref{tab:signatures}):

\begin{table}[H]
\centering
\caption{\textbf{Tissue-specific circadian gating signatures.}}
\label{tab:signatures}
\begin{tabular}{lll}
\toprule
\textbf{Tissue} & \textbf{Dominant Gated Pathway} & \textbf{Key Targets} \\
\midrule
Liver & Cell cycle, DNA repair & Wee1, Ccnd1, Ccnb1, Atm \\
Brown/White Fat & DNA damage, Apoptosis & Chek2, Bax, Yap1 \\
Heart & Hippo/growth control & Tead1 (all 8 core clocks) \\
Brainstem & Hippo pathway & Tead1 \\
Muscle & Proliferation & Myc, Wee1 \\
Kidney & Proliferation & Myc exclusively \\
Cerebellum & Mitosis & Cdk1 \\
Adrenal & Cell cycle checkpoint & Wee1 (strongest), Axin2 \\
Hypothalamus & DNA damage, Metabolism & Chek2, Sirt1 \\
\bottomrule
\end{tabular}
\end{table}

\textbf{Heart} showed selective gating of the Hippo pathway transcription factor Tead1 by all 8 core clock genes (p = 0.006--0.021), suggesting circadian control of organ size. \textbf{Cerebellum} exclusively gated Cdk1 by all 8 core clock genes (p = 0.011--0.026), particularly relevant given that medulloblastoma arises from cerebellar granule cell precursors.

\subsection{Tissue Deep-Dives}

\subsubsection*{Liver: Wee1-Centred Gating and DNA Damage Control}

In mouse liver datasets, PAR(2) identifies \textit{Wee1} as one of the most robustly gated targets. Across two independent liver time-series (GSE54650 and the gold-standard GSE11923 with 48 hourly timepoints), Wee1 appears as a highest-confidence tier target with FDR-significant phase-dependent coupling to at least one core clock gene.

\textbf{Eigenvalue characteristics:} Liver tissue exhibits mean AR(2) eigenvalue $|\lambda| = 0.717$ for clock genes and $|\lambda| = 0.614$ for target genes (January 2026 audit), yielding a clock-target eigenvalue difference of 10.3\%---indicating preserved circadian-proliferation hierarchy.

\subsubsection*{Heart: Tead1/YAP1-Linked Gating and Hippo-Cell Cycle Integration}

In heart datasets, PAR(2) highlights a different module centred on \textit{Tead1} and YAP1-linked targets.

\textbf{Eigenvalue characteristics:} Heart tissue exhibits mean AR(2) eigenvalue $|\lambda| = 0.689$ for clock genes and $|\lambda| = 0.356$ for target genes, yielding a clock-target eigenvalue difference of 33.3\%---the largest difference among major tissues. Heart shows 100\% of tested gene pairs within 5\% of the golden ratio $\phi$ (32/32 pairs), suggesting highly constrained dynamical architectures.

\textbf{Regeneration implications:} The mammalian heart has notoriously limited regenerative capacity. The prominence of Tead1/YAP1 gating may relate to the tight circadian control required to balance cardiomyocyte renewal against inappropriate proliferation.

\subsubsection*{Cerebellum: Cdk1-Linked Gating}

Cerebellum datasets reveal a distinct architecture centred on \textit{Cdk1} and related cell-cycle regulators. The Cdk1-centred module shows eigenvalues in the range $|\lambda| \approx 0.55$--$0.65$, with stable oscillatory dynamics.

\textbf{Marginally unstable observation:} The cerebellum Chek2-clock pair shows marginally unstable eigenvalue ($|\lambda| = 1.0017$), placing it just outside the strict stability boundary.

\subsection{Cross-Tissue Consensus Dramatically Reduces False Discovery Rate}

Requiring significance across multiple independent tissues provided robust identification of genuine phase-gating relationships (Table~\ref{tab:cross_tissue_fdr}):

\begin{table}[H]
\centering
\caption{\textbf{Cross-tissue consensus reduces FDR} (time-shuffle null, 1,000 permutations $\times$ 12 tissues).}
\label{tab:cross_tissue_fdr}
\begin{tabular}{lccl}
\toprule
Threshold & Real Pairs & Time-shuffle FPR & Interpretation \\
\midrule
Single tissue & 2,353 & 16.2\% $\pm$ 2.5\% & Moderate false positive rate \\
2+ tissues & 89 & 12.3\% $\pm$ 4.3\% & Improvement \\
\textbf{3+ tissues (HIGH)} & \textbf{21} & \textbf{2.1\% $\pm$ 1.8\%} & \textbf{Strong evidence threshold} \\
4+ tissues & 8 & 0.3\% $\pm$ 1.1\% & Very stringent \\
\bottomrule
\end{tabular}
\end{table}

\textbf{Methodological consideration:} The 12 GSE54650 tissues share experimental pipeline and animal cohort, so the effective number of independent contexts may be lower than 12. FPR estimates should be interpreted as approximate.

Based on these findings, we define a confidence tier system:
\begin{itemize}
    \item \textbf{HIGH confidence:} Significant in 3+ tissues with effect size f$^2 \geq 0.15$ (estimated FPR $\sim$1--5\%)
    \item \textbf{MEDIUM confidence:} Significant in 2+ tissues OR single tissue with f$^2 \geq 0.35$ (FPR $\sim$12--16\%)
    \item \textbf{EXPLORATORY:} Single-tissue significance only (FPR $\sim$16\%; hypothesis-generating)
\end{itemize}

\subsection{BMAL1-Specific Coupling Analysis: 85 Events Across 33 Genes in 12 Tissues}

An independent coupling analysis tested whether BMAL1 (Arntl) as an exogenous predictor significantly improves the AR(2) model fit for 53 target genes across all 12 GSE54650 tissues (636 total tests). Significance required both $\Delta$AIC~$>$~2 and $p < 0.05$, identifying 85 significant coupling events involving 33 unique genes (Table~\ref{tab:bmal1_coupling}).

\begin{table}[H]
\centering
\caption{\textbf{BMAL1 coupling universality: top genes ranked by tissue count.}}
\label{tab:bmal1_coupling}
\begin{tabular}{lcccl}
\toprule
Gene & Tissues & Category & Best $p$ & Independent Confirmation \\
\midrule
Wee1 & 10/12 & Checkpoint & $3 \times 10^{-6}$ & Matsuo et al.\ 2003 \textit{Science} \\
Nampt & 8/12 & Metabolic & $< 10^{-6}$ & Ramsey et al.\ 2009 \textit{Science} \\
Acaca & 5/12 & Metabolic & 0.001 & Adamovich et al.\ 2014 (indirect) \\
Actb & 4/12 & Housekeeping & $1.2 \times 10^{-4}$ & Kosir et al.\ 2010 \\
Ppara & 4/12 & Metabolic TF & 0.001 & Oishi et al.\ 2005 \\
Cdk6 & 4/12 & CDK Family & 0.006 & Novel prediction \\
\bottomrule
\end{tabular}
\end{table}

\textbf{Tissue coupling intensity varied dramatically:} Liver showed the most coupled genes (20), followed by Lung (14), Kidney and Muscle (8 each), Brown Fat (6), Heart/White Fat/Brainstem (5 each), Adrenal/Aorta/Cerebellum (4 each), and Hypothalamus (2). The low Hypothalamus coupling is consistent with the SCN driving rhythms via neural and hormonal signaling rather than direct transcriptional coupling.

Of the 25 distinct findings from this analysis, 7 were independently confirmed by published laboratory experiments (Wee1, Nampt, Ppara, Fasn, Pck1, G6pc, Xpa), 8 were strongly supported by existing literature (Acaca, Hmgcr, Sirt1, Ccnd1, Actb instability, Ogg1 tissue-specificity, Rad51, Xpc), and 10 represented novel testable predictions (e.g., Cdk6 coupling in Kidney/Brown Fat/Aorta/Cerebellum but not Liver; Ogg1 DNA repair coupling restricted to post-mitotic tissues Heart and Muscle). The 7 independently confirmed findings correspond to discoveries that collectively required \$5--15M and decades of traditional laboratory work, detected here blindly from public expression data.

\subsection{Three-Layer Eigenvalue Hierarchy Confirmed Across Datasets}

AR(2) eigenvalue analysis revealed a three-layer hierarchy of temporal persistence:

\begin{table}[H]
\centering
\caption{\textbf{Three-layer hierarchy: cross-dataset validation.}}
\label{tab:hierarchy}
\begin{tabular}{lccccc}
\toprule
Dataset & $n/p$ & Identity $|\lambda|$ & Prolif.\ $|\lambda|$ & Clock $|\lambda|$ & Hierarchy \\
\midrule
GSE11923 (48 tp, 1h) & 15.3 & 0.984 & 0.982 & 0.938 & I $>$ P $>$ C \\
GSE54650 (24 tp, 2h) & 7.3 & 0.994 & 0.975 & 0.858 & I $>$ P $>$ C \\
\bottomrule
\end{tabular}
\end{table}

Bootstrap confidence intervals (5,000 iterations) for the Identity--Clock gap were [0.016, 0.080] and for the Clock--Proliferation gap [$-$0.077, $-$0.014], neither crossing zero.

\subsection{Nine-Category Persistence Hierarchy}

Analysis of $\sim$1,594 genes across nine functional categories, pooled across all 12 tissues, revealed a statistically significant hierarchy (Table~\ref{tab:ninecategory}):

\begin{table}[H]
\centering
\caption{\textbf{Nine-category temporal persistence hierarchy} (pooled across 12 tissues, GSE54650).}
\label{tab:ninecategory}
\begin{tabular}{clccc}
\toprule
\textbf{Rank} & \textbf{Category} & \textbf{$n$ genes} & \textbf{Mean $|\lambda|$} & \textbf{95\% Bootstrap CI} \\
\midrule
1 & Clock            & 152 & 0.6985 & [0.685, 0.752] \\
2 & Chromatin        & 185 & 0.6709 & [0.651, 0.691] \\
3 & Metabolic        & 193 & 0.6619 & [0.643, 0.681] \\
4 & Housekeeping     & 171 & 0.6552 & [0.635, 0.676] \\
5 & Immune           & 169 & 0.6550 & [0.634, 0.676] \\
6 & Signaling        & 228 & 0.6527 & [0.636, 0.669] \\
7 & DNA Repair       & 185 & 0.6449 & [0.626, 0.664] \\
8 & Target           & 195 & 0.6322 & [0.613, 0.652] \\
9 & Stem Cell        & 116 & 0.6259 & [0.602, 0.650] \\
\bottomrule
\end{tabular}
\end{table}

The hierarchy was statistically significant (Kruskal-Wallis $H = 144.8$, $p < 0.001$, $df = 8$). Of 36 pairwise comparisons (Mann-Whitney $U$), nine reached nominal significance at $\alpha = 0.05$: Clock $>$ Target ($p = 0.0003$), Clock $>$ Stem Cell ($p = 0.0005$), Clock $>$ DNA Repair ($p = 0.0013$), Clock $>$ Signaling ($p = 0.0039$), Clock $>$ Metabolic ($p = 0.019$), Chromatin $>$ Target ($p = 0.022$), Clock $>$ Immune ($p = 0.025$), Chromatin $>$ Stem Cell ($p = 0.029$), and Clock $>$ Housekeeping ($p = 0.029$). Under Bonferroni correction ($\alpha/36 = 0.0014$), Clock $>$ Target and Clock $>$ Stem Cell remain significant.

Three features merit emphasis:

\begin{enumerate}[noitemsep]
    \item \textbf{Chromatin remodeling genes outrank housekeeping genes} ($|\lambda| = 0.671$ vs.\ $0.655$). Epigenetic regulators carry forward more temporal information than constitutively expressed reference genes, suggesting that ``stability'' in the sense of constant expression level is fundamentally different from ``persistence'' in the sense of multi-generational memory.

    \item \textbf{Stem cell markers show the lowest persistence} ($|\lambda| = 0.626$). This reframes stemness as a low-memory state: stem cells require temporal flexibility to respond rapidly to differentiation signals.

    \item \textbf{DNA repair genes rank seventh}, below signaling and metabolic genes but above cancer-relevant targets. DNA repair operates in a reactive, burst-like fashion consistent with low autoregressive memory.
\end{enumerate}

\subsection{Causal Validation: Bmal1-Knockout Collapses the Hierarchy}

We tested the hierarchy's dependence on a functional molecular clock using liver circadian time-series from Bmal1-knockout mice (GSE70499).

\textbf{Wild-type:} Clock $|\lambda| = 0.896$, Target $|\lambda| = 0.744$, gap $= +0.152$.

\textbf{Bmal1-KO:} Gap collapses to $-0.005$---statistically indistinguishable from zero. Clock eigenvalues decline ($0.896 \rightarrow 0.681$) while target eigenvalues remain largely unchanged ($0.744 \rightarrow 0.685$).

This causal perturbation confirms the hierarchy depends on the molecular clock rather than being a statistical artifact.

\subsection{Genome-Wide Validation}

To address panel selection artifacts, we ran AR(2) on all genes ($\sim$15K--60K per dataset) across five contexts (Table~\ref{tab:genomewide}):

\begin{table}[H]
\centering
\caption{\textbf{Genome-wide AR(2) validation across five contexts.}}
\label{tab:genomewide}
\begin{tabular}{p{3cm}lcccl}
\toprule
Dataset & Context & Genes & Clock Pctile & Wilcoxon $p$ & Hierarchy \\
\midrule
GSE54650 Liver & Healthy & $\sim$21K & 95th & $<10^{-8}$ & \textbf{YES} \\
GSE54650 Kidney & Healthy & $\sim$21K & 96.4th & $6.9\times10^{-9}$ & \textbf{YES} \\
GSE113883 Blood & Healthy & $\sim$58K & 79th & $0.0003$ & Partial \\
GSE221103 MYC-ON & Cancer & $\sim$60K & 75.5th & $0.0015$ & \textbf{REVERSED} \\
GSE70499 Bmal1-KO & Clock KO & $\sim$22K & 43.7th & $0.432$ & \textbf{COLLAPSED} \\
\bottomrule
\end{tabular}
\end{table}

The hierarchy is strongest in mouse solid tissues, weaker in human blood, reversed in cancer, and abolished when the clock is genetically disrupted.

\subsection{Eleven-Analysis Robustness Suite}

\subsubsection*{Binary Clock/Target Tests (7 analyses)}

\begin{enumerate}[noitemsep]
    \item \textbf{Sub-sampling recovery:} Hierarchy preserved down to $N=8$ timepoints (recovery rate $>$90\%).
    \item \textbf{Per-gene bootstrap CIs:} Gap confidence interval [0.058, 0.261] does not cross zero.
    \item \textbf{First-difference stationarity:} Preserved in only 2/12 tissues (over-corrects by destroying oscillatory autocorrelation).
    \item \textbf{Linear detrending:} Preserved in \textbf{12/12 tissues}---resolves the first-difference limitation (Table~\ref{tab:detrend}).
    \item \textbf{Gap permutation test:} All 5 datasets significant at $p < 0.001$, $z$-scores 3.47--4.33.
    \item \textbf{Leave-one-tissue-out CV:} 12/12 tissues independently confirm the hierarchy.
    \item \textbf{Population-level CV:} 25/25 folds preserve hierarchy (mean gap $0.216 \pm 0.051$).
\end{enumerate}

\subsubsection*{Multi-Category Stress Tests (4 analyses)}

\begin{enumerate}[noitemsep, start=8]
    \item \textbf{Multi-category permutation test ($10{,}000$ shuffles):} Observed $H = 144.8$ ($p < 0.001$) was never exceeded by any permutation (Table~\ref{tab:multicat_robust}).
    \item \textbf{Multi-category bootstrap CIs ($2{,}000$ iterations):} Clock genes ranked \#1 in \textbf{100\%} of iterations.
    \item \textbf{Multi-category detrending:} Mean Spearman rank correlation $\rho = 0.819$ with raw-data hierarchy; top category preserved in 100\% of datasets.
    \item \textbf{Multi-category leave-one-tissue-out:} Top-ranked category matched full-data result in \textbf{100\%} of leave-outs.
\end{enumerate}

\begin{table}[H]
\centering
\caption{\textbf{Multi-category robustness suite summary.}}
\label{tab:multicat_robust}
\begin{tabular}{lccl}
\toprule
\textbf{Test} & \textbf{Key Statistic} & \textbf{Result} & \textbf{Interpretation} \\
\midrule
Permutation ($10$K) & $H = 144.8$ & $p < 0.001$ & Not gene selection artifact \\
Bootstrap ($2$K) & Clock \#1 stability & 100\% & Robust category dominance \\
Detrending & $\rho = 0.819$ & Top preserved 100\% & Not trend artifact \\
Leave-one-tissue-out & Top match & 100\% & Cross-tissue generalizability \\
\bottomrule
\end{tabular}
\end{table}

\begin{table}[H]
\centering
\caption{\textbf{Linear detrending vs first-differencing across 12 mouse tissues.}}
\label{tab:detrend}
\begin{tabular}{lccccc}
\toprule
Tissue & Raw Gap & Detrended Gap & Preserved? & Differenced Gap & Preserved? \\
\midrule
Liver & +0.184 & +0.192 & YES & $-$0.023 & NO \\
Kidney & +0.301 & +0.332 & YES & $-$0.011 & NO \\
Heart & +0.263 & +0.309 & YES & +0.012 & NO \\
Lung & +0.321 & +0.338 & YES & $-$0.005 & NO \\
Adrenal & +0.387 & +0.412 & YES & +0.031 & YES \\
Hypothalamus & +0.063 & +0.106 & YES & +0.008 & NO \\
Cerebellum & +0.086 & +0.083 & YES & $-$0.014 & NO \\
Brown Fat & +0.264 & +0.231 & YES & $-$0.009 & NO \\
White Fat & +0.288 & +0.257 & YES & $-$0.018 & NO \\
Muscle & +0.174 & +0.159 & YES & $-$0.003 & NO \\
Aorta & +0.263 & +0.298 & YES & +0.019 & NO \\
Brainstem & +0.162 & +0.181 & YES & +0.025 & YES \\
\midrule
\textbf{Total} & --- & --- & \textbf{12/12} & --- & \textbf{2/12} \\
\bottomrule
\end{tabular}
\end{table}

\subsection{External Validation Benchmarks}

To assess whether PAR(2) eigenvalue predictions align with independent principles from physics and biology, we implemented four external benchmarks, each testing a falsifiable prediction of the eigenvalue framework against established theory or data:

\begin{enumerate}
    \item \textbf{Turing Symmetry-Breaking.} We tested whether the golden ratio ($\phi = 0.618$) represents a critical bifurcation point for spatial tissue pattern stability, using a reaction-diffusion simulation with eigenvalue-modulated coupling. Simulated spatial patterns collapse below $|\lambda| \approx 0.5$ and stabilize above $|\lambda| \approx 0.618$, with the bifurcation point matching $\phi$ exactly (deviation $< 0.1\%$). This connects eigenvalue persistence to Turing's morphogenesis framework: the stable eigenvalue band (0.4--0.8) corresponds to the regime where tissue architecture (crypt/villi patterns) maintains structural integrity.

    \item \textbf{Fisher Information Throughput.} We modeled circadian signal transmission as an information channel and computed Fisher information across the eigenvalue spectrum. Information throughput peaks at $|\lambda| = 0.56$ within the stable band (0.4--0.8), with cancer-range eigenvalues ($> 0.85$) showing 71\% throughput loss. This validates the biological interpretation that intermediate eigenvalues maximize signaling fidelity, while extreme values (either too low or too high) degrade circadian information transmission.

    \item \textbf{STRING Network Topology.} Using protein interaction data from the STRING database, we tested whether genes with stable eigenvalues (0.4--0.8) occupy hub positions in the interaction network. Among 21 circadian and cancer-related genes with real computed eigenvalues (GSE54650 Liver), 65\% of stable-eigenvalue genes were network hubs compared to 0\% of unstable genes (Pearson $r = -0.24$ between eigenvalue and degree). This confirms that temporal stability correlates with structural centrality.

    \item \textbf{Cross-Condition Disease Vulnerability.} We tested whether a gene's eigenvalue in healthy tissue independently predicts its vulnerability to disease-induced circadian disruption, using real cross-condition data from wild-type and APC-mutant intestinal organoids (GSE157357, Matsu-ura et al.). Across 36 matched genes, eigenvalue predicts disease disruption ($R^2 = 0.057$) independently of circadian phase ($R^2 = 0.096$). The low combined $R^2 = 0.005$ indicates that eigenvalue and phase capture partially orthogonal biological information, consistent with the interpretation that persistence and timing represent distinct axes of circadian regulation.
\end{enumerate}

All four benchmarks pass validation using real eigenvalue data, confirming that the PAR(2) framework is grounded in Turing pattern formation, information theory, network topology, and circadian chronobiology. Full benchmark details, including per-gene results and interactive visualizations, are available in the Discovery Engine web application.

\subsection{Root-Space Functional Geography}

Plotting genes in the full coefficient space---the stationarity triangle defined by $\beta_2 > -1$, $\beta_2 < 1 - \beta_1$, and $\beta_2 < 1 + \beta_1$---suggests that functional gene categories tend to occupy distinct \textit{dynamical neighborhoods}.

The stationarity triangle contains three structurally distinct boundary regions:
\begin{itemize}[noitemsep]
    \item \textbf{Positive-feedback pole} $(\beta_1, \beta_2) \rightarrow (2, -1)$: monotonic unit root ($\lambda = 1$). Self-reinforcing dynamics.
    \item \textbf{Alternating pole} $(\beta_1, \beta_2) \rightarrow (-2, -1)$: alternating unit root ($\lambda = -1$). Rapid state-switching dynamics.
    \item \textbf{Oscillatory pole} $(\beta_1, \beta_2) \rightarrow (0, 1)$: complex unit roots on the unit circle. Sustained oscillations.
\end{itemize}

Analysis of which genes occupy specific triangle regions shows consistent functional associations (Table~\ref{tab:rootspace_geography}):

\begin{table}[H]
\centering
\caption{\textbf{Root-space functional geography: genes closest to each dynamical pole.} $^*$Arntl from BMAL1-knockout dataset (GSE70499), included as positive control.}
\label{tab:rootspace_geography}
\begin{tabular}{p{2.5cm}p{2cm}p{1.5cm}p{1.5cm}p{1.5cm}p{3cm}}
\toprule
\textbf{Gene} & \textbf{Category} & \textbf{$\beta_1$} & \textbf{$\beta_2$} & \textbf{$|\lambda|$} & \textbf{Dynamical Pole} \\
\midrule
MAPK1 & Signaling & 0.436 & 0.560 & 0.998 & Self-reinforcing \\
IDH1 & Metabolic & 0.515 & 0.470 & 0.990 & Self-reinforcing \\
WEE1 & Target & 0.644 & 0.335 & 0.984 & Self-reinforcing \\
CRY1 & Clock & 0.428 & 0.505 & 0.956 & Self-reinforcing \\
\midrule
PTCH1 & Signaling & $-$0.438 & 0.553 & 0.994 & Alternating \\
APC & Target & $-$0.358 & 0.462 & 0.882 & Alternating \\
SOX2 & Stem Cell & $-$0.652 & 0.234 & 0.910 & Alternating \\
H2AFX & DNA Repair & $-$0.589 & 0.289 & 0.908 & Alternating \\
\midrule
PTEN & Signaling & 0.112 & 0.879 & 0.996 & Pure oscillation \\
NPAS2 & Clock & 0.040 & 0.699 & 0.856 & Pure oscillation \\
ASCL2 & Stem Cell & $-$0.002 & 0.684 & 0.828 & Pure oscillation \\
\midrule
Ly6a & Stem Cell & 0.237 & $-$0.015 & 0.122 & Centre (minimal memory) \\
TP53 & Target & 0.300 & $-$0.027 & 0.166 & Centre (minimal memory) \\
Arntl$^*$ & Clock & 0.251 & $-$0.049 & 0.222 & Centre (minimal memory) \\
\bottomrule
\end{tabular}
\end{table}

Four features of this functional geography merit emphasis:

\begin{enumerate}[noitemsep]
    \item \textbf{Growth signaling genes tend to occupy the self-reinforcing pole.} MAPK1 and IDH1 are among the genes closest to the self-reinforcing boundary. WEE1---the top circadian gatekeeper---also occupies this neighborhood, consistent with its role as a checkpoint that must maintain strong temporal persistence.

    \item \textbf{Tumor suppressors and fate switches tend to occupy the alternating pole.} PTCH1, APC, and H2AFX show the strongest alternating dynamics. APC's position is relevant: as a rapid on/off switch for Wnt signaling, loss of APC removes a toggle function that the circadian clock may attempt to replace through compensatory gating.

    \item \textbf{PTEN occupies the pure oscillation pole.} PTEN ($|\lambda| = 0.996$) sits nearly at the oscillatory vertex, indicating near-perfect sustained oscillation with minimal damping. PTEN's position suggests that PTEN normally provides \textit{rhythmic} rather than merely constitutive tumor suppression---creating temporal windows of growth permissiveness that become permanently open when PTEN is lost.

    \item \textbf{The most damped gene in the dataset is a stem cell marker.} Ly6a ($|\lambda| = 0.122$) sits nearest the triangle centre. TP53 ($|\lambda| = 0.166$) is comparably damped, consistent with p53's known pulsatile activation dynamics \citep{purvis2012}. Arntl in the BMAL1-knockout dataset shows $|\lambda| = 0.222$, confirming that genetic ablation of the core oscillator collapses temporal persistence.
\end{enumerate}

\textbf{Dynamical exclusion zone.} The ``barely oscillating'' regime is nearly empty. Of 1,594 gene$\times$dataset entries, 40.5\% have purely real roots, and 36.6\% show strong oscillation, but only 4.3\% fall in the transition zone. This pattern is consistent with a \textit{Hopf bifurcation boundary} separating overdamped from underdamped dynamics.

\subsection{APC Mutation Triggers Compensatory Amplification of Circadian Gating}

To determine whether the circadian system responds to oncogenic mutations, we analyzed intestinal organoids across four genotypes (Table~\ref{tab:intestinal}):

\begin{table}[H]
\centering
\caption{\textbf{Circadian Gating Rates in Intestinal Organoids by Genotype.}}
\label{tab:intestinal}
\begin{tabular}{lccc}
\toprule
\textbf{Condition} & \textbf{Significant Pairs} & \textbf{Discovery Rate} & \textbf{Fold Change} \\
\midrule
Wild Type (APC-WT/BMAL-WT) & 17 & 11.2\% & 1.0$\times$ (reference) \\
APC Mutant (APC-Mut/BMAL-WT) & 34 & \textbf{22.4\%} & \textbf{2.0$\times$} \\
BMAL Mutant (APC-WT/BMAL-Mut) & 14 & 9.2\% & 0.8$\times$ \\
Double Mutant (APC-Mut/BMAL-Mut) & 2 & \textbf{1.3\%} & \textbf{0.1$\times$} \\
\bottomrule
\end{tabular}
\end{table}

Strikingly, APC mutation \textit{increased} circadian gating rather than disrupting it. The genes gaining circadian control were specifically cancer-relevant:
\begin{itemize}[noitemsep]
    \item \textbf{Wee1} (G2/M checkpoint): Gated by 4 clock genes ($p = 0.002$--$0.029$)
    \item \textbf{Ccnb1} (Cyclin B1, mitosis entry): Gated by 7 clock genes ($p = 0.001$--$0.031$)
    \item \textbf{Tp53} (tumor suppressor): Gated by 3 clock genes ($p = 0.016$--$0.042$)
    \item \textbf{Sirt1} (metabolic sensor): Gated by 5 clock genes ($p = 0.002$--$0.044$)
\end{itemize}

\begin{figure}[H]
\centering
\fbox{\parbox{0.85\textwidth}{\centering\vspace{2em}\textbf{Figure 2: Compensatory Circadian Gating in APC-Mutant Organoids}\\\vspace{1em}Bar chart showing discovery rates across four genotypes: WT (11.2\%), APC-Mut (22.4\%), BMAL-Mut (9.2\%), Double-Mut (1.3\%).\vspace{2em}}}
\caption{\textbf{APC mutation triggers compensatory amplification of circadian gating.} Discovery rates across four intestinal organoid genotypes (GSE157357). APC mutation doubles circadian gating (11.2\% $\rightarrow$ 22.4\%), while combined APC+BMAL1 mutation causes 17-fold collapse (22.4\% $\rightarrow$ 1.3\%).}
\label{fig:compensatory}
\end{figure}

\subsection{Combined Mutations Cause Near-Complete Gating Collapse}

When both APC and BMAL1 were mutated, circadian gating collapsed nearly completely. Only 2 of 299 gene pairs (0.7\%) showed significant gating---a 17-fold reduction compared to APC-only mutants.

The two surviving gating relationships involved only the Clock gene:
\begin{itemize}[noitemsep]
    \item CDK1 (mitosis driver) $\times$ Clock ($p = 0.036$)
    \item Sirt1 (NAD+ sensor) $\times$ Clock ($p = 0.040$)
\end{itemize}

These data are consistent with a functional two-hit threshold: APC loss triggers compensation, but BMAL1 loss abolishes the compensatory capacity.

\subsection{BMAL1 Mutation Rewires Gating to Stem Cell Pathways}

BMAL1-mutant organoids showed a qualitatively different gating pattern. Rather than gating cell cycle genes, the remaining clock components focused on the stem cell marker LGR5.

LGR5 was gated by all 8 core clock genes tested ($p = 0.012$--$0.046$)---the most comprehensive gating relationship observed in any condition (Table~\ref{tab:lgr5}).

\begin{table}[H]
\centering
\caption{\textbf{LGR5 Stem Cell Marker Gating in BMAL1-Mutant Organoids.}}
\label{tab:lgr5}
\begin{tabular}{lcc}
\toprule
\textbf{Clock Gene} & \textbf{P-value} & \textbf{Significant Terms} \\
\midrule
Arntl (BMAL1) & 0.012 & R$_{n-1}$, R$_{n-2}$ \\
Per2 & 0.013 & R$_{n-2}$ \\
Cry1 & 0.014 & R$_{n-1}$ \\
Per1 & 0.019 & R$_{n-2}$ \\
Clock & 0.021 & R$_{n-1}$, R$_{n-2}$ \\
Cry2 & 0.030 & R$_{n-1}$ \\
Nr1d2 & 0.041 & R$_{n-2}$ \\
Nr1d1 & 0.046 & R$_{n-1}$ \\
\bottomrule
\end{tabular}
\end{table}

To our knowledge, this is the first report of circadian gating of LGR5, which marks intestinal stem cells critical for tissue renewal and cancer stem cell formation \citep{barker2007}. This finding gains additional significance from the nine-category persistence hierarchy: stem cell markers show the \textit{lowest} temporal persistence of all nine categories ($|\lambda| = 0.626$), suggesting stemness is a low-memory dynamical state.

\subsection{Pparg as the Top Cancer-Specific Target in MYC-ON Neuroblastoma}

In MYC-ON neuroblastoma (GSE221103), Pparg (PPAR$\gamma$) emerged as the only FDR-significant target, reaching significance with all 8 core clock genes (Table~\ref{tab:pparg}).

\begin{table}[H]
\centering
\caption{\textbf{Pparg gating in MYC-ON neuroblastoma} (all q $<$ 0.05, f$^2 = 10.86$).}
\label{tab:pparg}
\begin{tabular}{llccc}
\toprule
Target & Clock Gene & q-value & f$^2$ & Significant Terms \\
\midrule
Pparg & Nr1d2 & 0.045 & 10.86 & R$_{n-1}$cos, R$_{n-1}$sin, R$_{n-2}$cos, R$_{n-2}$sin \\
Pparg & Per2 & 0.045 & 10.86 & R$_{n-1}$cos, R$_{n-1}$sin, R$_{n-2}$sin \\
Pparg & Arntl & 0.045 & 10.86 & R$_{n-1}$cos, R$_{n-1}$sin, R$_{n-2}$cos, R$_{n-2}$sin \\
Pparg & Cry1 & 0.045 & 10.86 & All 4 terms \\
Pparg & Per1 & 0.045 & 10.86 & All 4 terms \\
Pparg & Clock & 0.046 & 10.86 & R$_{n-1}$cos, R$_{n-2}$cos, R$_{n-2}$sin \\
Pparg & Cry2 & 0.045 & 10.86 & All 4 terms \\
Pparg & Nr1d1 & 0.045 & 10.86 & R$_{n-1}$cos, R$_{n-1}$sin, R$_{n-2}$cos \\
\bottomrule
\end{tabular}
\end{table}

\textbf{Column-space invariance explains identical effect sizes:} All eight clock genes yield identical f$^2 = 10.86$. This is not an error but a mathematical consequence of the PAR(2) model structure: when clock gene phases are constant offsets of a shared TTFL oscillator, the cosine/sine interaction terms span identical column spaces under OLS regression. The projection matrix $\mathbf{H} = \mathbf{X}(\mathbf{X}^T\mathbf{X})^{-1}\mathbf{X}^T$ is identical regardless of which clock gene's phase is used. \textbf{The eight tests reflect a single finding that circadian phase gates Pparg expression in MYC-ON cells.}

\subsection{Eigenperiod Separation Distinguishes Healthy from Cancer Tissues}

The systems-level eigenperiod structure showed remarkable robustness (Table~\ref{tab:eigenperiod}). Healthy mouse tissues exhibited eigenperiods in the ultradian range (7.2--13.3 hours), while cancer models showed near-circadian eigenperiods ($\sim$22--23 hours).

\begin{table}[H]
\centering
\caption{Eigenperiod comparison across tissue types. \textit{Note: healthy--cancer comparison is subject to species $\times$ tissue $\times$ context confound.}}
\label{tab:eigenperiod}
\begin{tabular}{lcccl}
\toprule
Tissue/Condition & Mean Eigenperiod & Range & Stability & Classification \\
\midrule
\multicolumn{5}{l}{\textbf{Healthy Mouse Tissues (GSE54650)}} \\
Cerebellum & 7.2h & 5.9--10.2h & 100\% & Ultradian \\
Hypothalamus & 7.6h & 6.1--9.8h & 100\% & Ultradian \\
Brainstem & 8.4h & 5.9--10.2h & 100\% & Ultradian \\
Adrenal & 9.6h & 4.6--22.8h & 100\% & Ultradian \\
White Fat & 9.8h & 5.2--18.4h & 100\% & Ultradian \\
Liver & 10.4h & 5.3--21.1h & 100\% & Ultradian \\
Muscle & 11.1h & 6.4--19.6h & 100\% & Ultradian \\
Aorta & 11.4h & 5.2--44.4h & 100\% & Ultradian \\
Lung & 12.1h & 5.9--25.7h & 100\% & Ultradian \\
Kidney & 12.2h & 5.5--30.2h & 100\% & Ultradian \\
Brown Fat & 12.4h & 5.3--28.9h & 100\% & Ultradian \\
Heart & 13.3h & 6.2--32.5h & 100\% & Ultradian \\
\midrule
\multicolumn{5}{l}{\textbf{Cancer Models (GSE221103)}} \\
Neuroblastoma MYC-ON & 22.7h & 12.2--34.8h & 42\% & Near-circadian \\
Neuroblastoma MYC-OFF & 23.4h & 13.5--44.1h & 58\% & Near-circadian \\
\bottomrule
\end{tabular}
\end{table}

\begin{figure}[H]
\centering
\fbox{\parbox{0.8\textwidth}{\centering\vspace{2cm}\textbf{Figure 3: Eigenperiod Distribution Across Tissues}\\\vspace{0.5cm}(A) Violin plots of eigenperiod by tissue. (B) Healthy vs cancer comparison. (C) Stability analysis.\vspace{2cm}}}
\caption{Eigenperiod analysis reveals systems-level differences between healthy and cancer tissues. (A) Violin plots showing the distribution of eigenperiods; healthy tissues cluster in the 7--13h range while cancer models show 22--23h periods. (B) Box plot comparison. (C) Stability classification.}
\label{fig:eigenperiod}
\end{figure}

\textbf{Critical confound:} The healthy data are from mouse tissues in vivo while cancer data are from human neuroblastoma cell lines in culture. The eigenperiod separation may therefore reflect species, tissue, or culture context differences rather than (or in addition to) a cancer-specific effect. This comparison is hypothesis-generating and requires within-species, within-tissue validation.

\subsubsection*{Period Sensitivity Analysis: Ruling Out Circular Inference}

To rigorously test whether eigenperiod might be ``imprinted'' from the assumed 24-hour period, we varied the assumed circadian period $T \in \{20, 22, 24, 26, 28\}$ hours (Table~\ref{tab:period_sensitivity}):

\begin{table}[H]
\centering
\caption{Period sensitivity analysis: Eigenperiod separation across different assumed periods}
\label{tab:period_sensitivity}
\begin{tabular}{lccccl}
\toprule
Assumed Period (T) & Healthy Mean & Cancer Mean & $\Delta$ (h) & Ratio & p-value \\
\midrule
20h & 12.9h & 23.9h & +11.0 & 1.85$\times$ & $<10^{-15}$ \\
22h & 12.3h & 22.3h & +10.0 & 1.82$\times$ & $<10^{-15}$ \\
\textbf{24h} & \textbf{12.9h} & \textbf{22.7h} & \textbf{+9.8} & \textbf{1.77$\times$} & $<10^{-15}$ \\
26h & 12.6h & 23.5h & +10.8 & 1.86$\times$ & $<10^{-15}$ \\
28h & 12.5h & 24.8h & +12.3 & 1.98$\times$ & $<10^{-15}$ \\
\bottomrule
\end{tabular}
\end{table}

The separation remained significant at \textit{all} period assumptions, providing strong evidence \textit{against} period imprinting.

\subsubsection*{Batch Correction Validation}

Z-score normalization within each dataset preserved the eigenperiod separation ($\Delta = 18.3$ hours, $p = 3.3 \times 10^{-13}$), providing evidence against cross-dataset artifacts (Table~\ref{tab:batch_correction}).

\begin{table}[H]
\centering
\caption{Batch correction validation: Eigenperiod separation before and after z-score normalization}
\label{tab:batch_correction}
\begin{tabular}{lccccc}
\toprule
Condition & Raw Mean & Normalized Mean & Raw p-value & Norm p-value \\
\midrule
Healthy tissues & 8.1h & 13.9h & --- & --- \\
Cancer models & 27.7h & 32.2h & --- & --- \\
\midrule
\textbf{Separation} & $\Delta$=19.6h & $\Delta$=18.3h & $1.5\times10^{-8}$ & $\mathbf{3.3\times10^{-13}}$ \\
\bottomrule
\end{tabular}
\end{table}

\subsubsection*{Circular-Shift Null Model}

To provide a more conservative null model preserving temporal autocorrelation structure, we implemented circular-shift permutation testing (1,000 iterations). Across 192 gene pairs tested in 6 GSE54650 tissues, the circular-shift null yielded a false positive rate of \textbf{0.0\%} after Bonferroni correction, indicating that PAR(2) is not falsely detecting phase-gating from autocorrelation structure alone.

\subsection{Predictive Cross-Validation}

Rolling-origin cross-validation (25\% holdout, 496 gene pairs) showed that PAR(2) does not consistently outperform reduced AR(2) in out-of-sample prediction (45.2\% win rate). This clarifies PAR(2)'s role as a \textit{discovery engine} for identifying candidate phase-gating relationships rather than a forecasting model \citep{shmueli2010}.

\begin{table}[H]
\centering
\caption{Rolling-origin cross-validation: PAR(2) vs reduced AR(2)}
\label{tab:cv_results}
\begin{tabular}{lcccc}
\toprule
Dataset & N Pairs & PAR(2) Win Rate & Mean RMSE Improvement \\
\midrule
GSE54650 Liver & 120 & 60.0\% & $-23.5\%$ \\
GSE54650 Kidney & 120 & 53.3\% & $-6.1\%$ \\
GSE54650 Heart & 120 & 46.7\% & $-2.6\%$ \\
GSE221103 MYC-ON & 136 & 23.5\% & $-269.7\%$ \\
\midrule
\textbf{Overall} & 496 & 45.2\% & $-81.7\%^{\dagger}$ \\
\bottomrule
\multicolumn{4}{l}{\footnotesize $^{\dagger}$Negative indicates AR(2) outperforms PAR(2); dominated by outliers in MYC-ON.}
\end{tabular}
\end{table}

\subsection{Stability Patterns Reflect Circadian Network Integrity}

Healthy tissues exhibited near-universal stability (88--100\% of gene pairs with $|\lambda_{\max}| < 1$), while cancer models showed substantially reduced stability (Table~\ref{tab:stability}).

\begin{table}[H]
\centering
\caption{Dynamical stability by condition}
\label{tab:stability}
\begin{tabular}{lccc}
\toprule
Condition & Stable Pairs & Unstable Pairs & Interpretation \\
\midrule
Healthy tissues (mean) & 94.2\% & 5.8\% & Robust homeostasis \\
MYC-ON Neuroblastoma & 42\% & 58\% & Disrupted regulation \\
MYC-OFF Neuroblastoma & 58\% & 42\% & Partial restoration \\
APC$^{-/-}$ Organoids & 71\% & 29\% & Moderate disruption \\
BMAL1$^{-/-}$ Organoids & 68\% & 32\% & Clock-dependent effect \\
\bottomrule
\end{tabular}
\end{table}

\textbf{Biological interpretation of ``unstable'' dynamics:} ``Unstable'' ($|\lambda| \geq 1$) does not imply that gene expression grows to infinity---biological systems are inherently bounded. Rather, unstable dynamics indicate that perturbations in gene expression are \textit{maintained} across the circadian cycle without homeostatic correction, consistent with sustained proliferative signaling characteristic of malignancy \citep{hanahan2011}.

\subsection{Disease Eigenvalue Convergence}

In healthy tissues, clock genes show higher AR(2) eigenvalues than target genes (``clock $>$ target hierarchy''). Disease conditions reverse this pattern:

\begin{table}[H]
\centering
\caption{\textbf{Cross-tissue circadian gating architecture: healthy vs.\ disease.}}
\label{tab:tissue_atlas}
\begin{tabular}{lcccc}
\toprule
Condition & Clock $|\lambda|$ & Target $|\lambda|$ & Gap & Pattern \\
\midrule
Liver (healthy) & 0.717 & 0.614 & +0.103 & Clock $>$ Target \\
Heart (healthy) & 0.689 & 0.356 & +0.333 & Clock $>$ Target \\
Blood (healthy) & 0.569 & 0.376 & +0.193 & Clock $>$ Target \\
Kidney (healthy) & 0.889 & 0.643 & +0.246 & Clock $>$ Target \\
Lung (healthy) & 0.782 & 0.542 & +0.240 & Clock $>$ Target \\
Neuroblastoma MYC-ON & 0.617 & 0.596 & +0.021 & Near-convergence \\
Organoids APC/BMAL1-KO & 0.619 & 0.705 & $-$0.086 & \textbf{Target $>$ Clock} \\
\bottomrule
\end{tabular}
\end{table}

The gap-threshold classifier (healthy if gap $> 0$, disrupted if gap $\leq 0$) achieves 82.9\% accuracy across 35 conditions (Cohen's $d = 1.56$) with zero free parameters.

\subsection{Aging and Cancer as Divergent Trajectories}

Extended validation revealed that aging and cancer represent \textit{distinct} deformations of the clock-target hierarchy:

\textbf{Multi-tissue aging (GSE201207):} All peripheral tissues show gap \textit{decrease} with age---clock eigenvalues decline faster than target eigenvalues.

\textbf{Pancreas exception (GSE245295):} Clock eigenvalues \textit{increased} with age ($0.704 \rightarrow 0.846$) while target eigenvalues decreased ($0.763 \rightarrow 0.511$). The clock-target gap changed from $-0.059$ to $+0.334$---an \textit{enhancement} of circadian dominance. This unique pancreatic pattern may explain reduced $\beta$-cell regenerative capacity in aged pancreas.

\textbf{Cancer trajectory:} APC-mutant organoids show gap $= -0.122$ (target exceeds clock); PDA organoids show gap $\approx 0$ (convergence).

\textbf{Divergent pre-disease trajectories:}
\begin{itemize}[noitemsep]
    \item \textbf{Aging (pancreas)}: Clock $\uparrow\\uparrow$, Target $\downarrow\downarrow$, Gap $= +0.33$ $\rightarrow$ RIGIDITY
    \item \textbf{Cancer}: Clock $\downarrow$, Target $\\uparrow$, Gap $= -0.12$ $\rightarrow$ ESCAPE
\end{itemize}

\textbf{Caloric restriction rescue (GSE93903):} Old mice show gap narrowing ($+0.128 \rightarrow +0.083$, $-$35\%). Caloric restriction partially rescues this (Old+CR gap $= +0.116$, $\sim$73\% recovery toward young baseline) \citep{sato2017}.

These findings were subjected to falsification tests: permutation testing ($p = 0.049$), housekeeping gene controls (no batch effect), timepoint shuffle (3/100 exceeded observed gap), and Cohen's d ($= 0.50$, medium effect).

\subsection{Exploratory: Golden-Ratio Dynamics}

We briefly explored whether AR(2) coefficient ratios approximated the golden ratio ($\phi \approx 1.618$), based on theoretical connections between AR(2) dynamics and Fibonacci sequences \citep{boman2025fibonacci}. Tissue-specific enrichment was observed in hypothalamus, heart, and kidney (100\% of stable pairs near $\phi$), while other tissues showed 0\% enrichment. This finding is highly preliminary and should be considered exploratory. \textbf{Importantly, none of the main PAR(2) conclusions regarding cross-tissue consensus, eigenperiod separation, or the Wee1/Pparg candidates depend on this $\phi$-enrichment observation.}

\subsection{Comparison with Established Circadian Analysis Methods}

\begin{table}[H]
\centering
\caption{\textbf{AR(2) eigenvalue analysis vs established circadian methods.}}
\begin{tabular}{@{}lcccc@{}}
\toprule
\textbf{Property} & \textbf{AR(2)} & \textbf{JTK\_CYCLE} & \textbf{RAIN} & \textbf{Cosinor} \\
\midrule
Quantifies persistence & Yes & No & No & No \\
Multi-generational memory & Yes (lag-2) & No & No & No \\
Period assumption & No & Yes (24h) & No & Yes \\
Output metric & $|\lambda|$ (continuous) & p-value & p-value & Amplitude \\
Detects hierarchy disruption & Yes (gap sign) & No & No & No \\
\bottomrule
\end{tabular}
\end{table}

JTK\_CYCLE, RAIN, and cosinor answer ``is this gene rhythmic?''---AR(2) eigenvalue analysis answers ``how strongly does this gene persist, and does a hierarchy exist between gene classes?'' Standard rhythm tools showed only $1.75\times$ condition discrimination versus $6.50\times$ for PAR(2) on the same organoid data.

PAR(2) should therefore be viewed as complementary to existing circadian tools rather than a replacement. A recommended workflow would use JTK\_CYCLE or RAIN to identify rhythmic genes, then apply PAR(2) to test phase-gating hypotheses for clock-target pairs of interest.

%%%%%%%%%%%%%%%%%%%%%%%%%%%%%%%%%%%%%%%%%%%%%%%%%%%%%%%%%%%%%
\section{Discussion}
%%%%%%%%%%%%%%%%%%%%%%%%%%%%%%%%%%%%%%%%%%%%%%%%%%%%%%%%%%%%%

\subsection{Summary of Key Findings}

This study presents the PAR(2) framework for analyzing phase-dependent circadian gating of cancer-related gene expression. Our analysis of 28,138 gene pairs across 22 datasets yields five principal findings:

\begin{enumerate}
    \item \textbf{Cross-tissue consensus substantially reduces FDR:} While single-tissue claims show $\sim$16\% false positive rates, requiring significance in 3+ tissues reduces FDR to approximately 1--5\% (order-of-magnitude estimate)---roughly an 8-fold improvement. \textit{Important caveat:} The 12 GSE54650 mouse tissues share experimental pipeline and animal cohort, so the effective number of independent contexts may be lower than 12.
    
    \item \textbf{Systems-level eigenperiod shows consistent separation:} Healthy mouse tissues exhibited 7--13 hour ultradian periods with 88--100\% stability versus 22--23 hour near-circadian periods with 42\% stability in cancer models. This represents an exploratory systems-level metric requiring independent validation.
    
    \item \textbf{Stability loss in cancer:} Healthy tissues maintain 88--100\% dynamical stability, while cancer models show reduced stability (42--58\%), consistent with circadian dysregulation.
    
    \item \textbf{Wee1 as the top computational candidate:} Multi-criteria filtering identified Wee1 as associated with all 8 clock genes across 4--6 tissues each---the strongest circadian hub candidate.
    
    \item \textbf{Pparg as a cancer-specific candidate:} The metabolic regulator Pparg emerges as the top candidate in the cancer-specific MYC-ON neuroblastoma context, reaching significance with all eight clock genes.
\end{enumerate}

\subsection{Wee1 as the Leading Circadian Checkpoint Candidate}

Our finding that Cry1$\rightarrow$Wee1 is the only broadly conserved circadian gating relationship in our panel has significant implications. Wee1 phosphorylates CDK1 at Tyr15, maintaining G2/M checkpoint arrest until DNA replication and repair are complete \citep{mcgowan1995}. The circadian regulation of Wee1 by CLOCK/BMAL1 via E-box elements was previously demonstrated in liver \citep{matsuo2003}, but our data show this extends body-wide.

This conservation suggests that \textbf{temporal coordination of the G2/M checkpoint may be a fundamental requirement across mammalian tissues}. Cells may restrict mitosis to specific circadian phases---when DNA repair capacity peaks---regardless of tissue type.

\subsection{Clinical Implications: Wee1 Inhibitor Timing}

Wee1 inhibitors (e.g., adavosertib/AZD1775) are in clinical trials for p53-mutant tumors \citep{leijen2016}. Our finding that Wee1 is broadly clock-gated across tissues suggests these inhibitors may have time-of-day-dependent efficacy and toxicity profiles. The Adrenal gland's strongest Wee1 gating (p=0.0006) suggests particular sensitivity to circadian timing.

\subsection{Reassessing the Myc Paradigm}

Myc is frequently cited as a key circadian-cancer interface \citep{altman2015}. Our pan-tissue analysis reveals that Myc gating is tissue-restricted (Muscle and Kidney only), suggesting the Myc$\leftrightarrow$clock axis is context-dependent rather than universal.

\subsection{The Vulnerability Protection Model}

We propose that each tissue uses circadian gating to protect against its most significant physiological threat:

\begin{itemize}[noitemsep]
    \item \textbf{Liver}: Constant xenobiotic exposure $\rightarrow$ DNA damage/checkpoint gating (ATM, Wee1)
    \item \textbf{Heart}: Post-mitotic; growth restraint critical $\rightarrow$ Hippo/YAP gating (Tead1)
    \item \textbf{Kidney}: Regenerative capacity via tubular proliferation $\rightarrow$ Myc gating
    \item \textbf{Cerebellum}: Ongoing neurogenesis; medulloblastoma risk $\rightarrow$ Cdk1 gating
    \item \textbf{Hypothalamus}: Master clock + metabolic integration $\rightarrow$ Sirt1/Chek2 gating
\end{itemize}

\subsection{A New Paradigm: The Clock Fights Back}

Our central finding---that APC mutation \textit{increases} circadian gating---challenges the prevailing paradigm that circadian disruption simply enables cancer. Instead, we propose that the clock actively compensates for oncogenic mutations by increasing temporal control over oncogenic genes. This compensatory response has not been previously reported and suggests that the circadian clock may function as an active tumor suppressor, not merely a passive victim.

\subsection{The Two-Hit Model}

The quantitative pattern across organoid genotypes suggests a two-hit model:

\begin{enumerate}[noitemsep]
    \item \textbf{First hit (APC loss):} Circadian system compensates by doubling gating of cell cycle and tumor suppressor genes.
    \item \textbf{Second hit (BMAL1 loss):} Compensatory capacity is abolished, gating collapses 17-fold.
\end{enumerate}

This framework explains why single clock gene mutations have modest cancer effects while combined circadian and Wnt pathway disruption dramatically accelerates tumorigenesis \citep{chun2022}. The anti-resonance condition in the Rosen Wnt model \citep{rosen2026} acts as a natural filter against spurious Wnt signaling. Loss of APC disrupts destruction complex dynamics, breaking the anti-resonant filter---precisely the ``gating loss'' detected by PAR(2). The compensatory gating amplification may represent the clock's attempt to restore temporal filtering through alternative mechanisms.

\subsection{LGR5 and Cancer Stem Cells}

The observation that LGR5 is circadian-gated, if confirmed, has immediate implications for cancer stem cell biology. LGR5 marks intestinal stem cells that give rise to adenomas when APC is lost \citep{barker2007}. The nine-category persistence hierarchy provides dynamical context: stem cell markers collectively show the lowest temporal persistence ($|\lambda| = 0.626$) of all functional categories. We interpret this as a functional requirement: stem cells must maintain low autoregressive memory to preserve their capacity for rapid fate transitions. The clock's compensatory gating of LGR5 may represent an attempt to externally impose temporal structure on gene expression that is intrinsically temporally flexible.

\subsection{Pparg as a Cancer-Context Target}

PPAR$\gamma$ is a master regulator of lipid metabolism with established connections to circadian rhythms \citep{wang2008, yang2006}. Its emergence as the sole FDR-significant target in MYC-ON neuroblastoma suggests that circadian regulation of lipid metabolism may be particularly disrupted during oncogenic transformation, consistent with metabolic reprogramming of cancer cells \citep{hanahan2011, pavlova2016}. PPAR$\gamma$ agonists (thiazolidinediones) have shown anticancer effects in multiple tumor types \citep{michalik2004}, and circadian timing of such agents may influence efficacy.

\subsection{Temporal Persistence Versus Constitutive Stability}

The nine-category hierarchy reveals a fundamental distinction between two forms of gene expression ``stability.'' Housekeeping genes are constitutively expressed at near-constant levels. Yet their temporal persistence ($|\lambda| = 0.655$) ranks only fourth among nine categories. Chromatin remodeling genes ($|\lambda| = 0.671$), whose expression fluctuates more, carry stronger multi-generational memory.

This distinction has practical implications. Housekeeping genes are stable in the sense that today's expression level predicts today's expression level. Chromatin remodeling genes are persistent in the sense that today's trajectory predicts tomorrow's trajectory. The AR(2) eigenvalue captures dynamical momentum, which is invisible to standard metrics of expression variability.

\subsection{Stem Cells as Low-Memory States}

Stem cell markers ranking last ($|\lambda| = 0.626$) invites a reinterpretation of stemness in dynamical terms. Low autoregressive memory may be a functional requirement: stem cells must remain temporally flexible to respond to differentiation cues, niche signals, and tissue damage. High persistence would lock stem cells into a fixed temporal trajectory, constraining their ability to adopt new fates.

This interpretation connects to the Boman crypt compartment model \citep{boman2025}: the cycling stem cell compartment ($C$) drives the system's oscillatory dynamics, but the individual stem cell genes themselves show low persistence because their expression must respond rapidly to compartment-level signals.

\subsection{Root-Space Geography as a Dynamical Taxonomy}

The root-space functional geography (Table~\ref{tab:rootspace_geography}) suggests that position in the $(\beta_1, \beta_2)$ coefficient space is associated with qualitatively different regulatory strategies. Three observations support this interpretation.

First, genes near the self-reinforcing pole tend to have known positive feedback loops. MAPK1 participates in the RAS-RAF-MEK-ERK cascade; IDH1 catalyzes an irreversible metabolic step; WEE1 participates in the CDK1-cyclin B feedback loop.

Second, genes near the alternating pole tend to function as bistable switches. PTCH1 toggles the Hedgehog pathway; APC controls $\beta$-catenin degradation in a switch-like manner; SOX2 participates in the pluripotency network.

Third, the PTEN finding---near-perfect oscillation with $|\lambda| = 0.996$ at the oscillatory vertex---generates a specific prediction: PTEN loss should manifest as loss of \textit{oscillatory} tumor suppression, not merely loss of expression level. This prediction is testable in time-resolved PTEN-knockout expression data.

The root-space geography extends the nine-category hierarchy from a one-dimensional ranking to a two-dimensional map. AR(2) coefficient space could serve as a dynamical taxonomy of gene regulation, pending formal validation.

\subsection{Biological Interpretation of Eigenperiod Differences}

The approximately 2-fold eigenperiod difference between healthy and cancer tissues invites biological interpretation. In healthy tissues, the 7--13 hour ultradian eigenperiod may reflect rapid transcriptional responses tightly coupled to cell cycle checkpoints \citep{matsuo2003, geyfman2012brain}. The longer 22--23 hour eigenperiod in cancer may represent a ``slowing'' of regulatory dynamics, consistent with loss of checkpoint control.

An alternative interpretation is that the cancer eigenperiod converges toward the 24-hour circadian period itself, potentially reflecting a loss of the distinction between target gene dynamics and clock gene dynamics.

\textbf{Importantly, this interpretation requires validation.} While the theoretical framework is well-established in ecology and climate science \citep{scheffer2009early}, its application to circadian-cancer dynamics remains hypothesis-generating.

\subsection{Relationship to Prior Gene Mapping Frameworks}

The root-space map invites comparison with existing approaches to organizing gene behavior in two-dimensional space.

\textbf{Waddington's epigenetic landscape} \citep{waddington1957} is the earliest and most influential dynamical map in biology. The root-space triangle shares structural features with this landscape---biological systems cluster in distinct regions rather than occupying the full parameter space uniformly. However, the Waddington landscape maps \textit{cell states}, whereas the root-space triangle maps \textit{gene dynamical strategies}.

\textbf{PCA and expression-based dimensionality reduction} (UMAP, t-SNE) capture variance in expression \textit{levels}. The root-space triangle captures dynamics over \textit{time}. The PCA comparison overlay in the Discovery Engine demonstrates empirically that eigenvalue-based persistence captures information orthogonal to variance-based dimensionality reduction.

%%%%%%%%%%%%%%%%%%%%%%%%%%%%%%%%%%%%%%%%%%%%%%%%%%%%%%%%%%%%%
\section{Limitations}
%%%%%%%%%%%%%%%%%%%%%%%%%%%%%%%%%%%%%%%%%%%%%%%%%%%%%%%%%%%%%

\begin{enumerate}
    \item \textbf{Species $\times$ tissue $\times$ context confound (Major):} The eigenperiod cancer comparison involves mouse in vivo tissues vs.\ human neuroblastoma cell culture. Within-species, within-tissue validation is required.
    
    \item \textbf{Single-tissue FDR addressed by cross-tissue consensus:} Single-tissue significance shows $\sim$16\% false positive rates. Gene pairs meeting only single-tissue significance should be considered hypothesis-generating; pairs meeting HIGH confidence criteria (3+ tissues) have substantially stronger evidence.
    
    \item \textbf{Cross-tissue correlation not formally modeled:} The 12 GSE54650 mouse tissues share a common genetic background and environmental entrainment, so the \textit{effective} number of independent contexts may be lower than 12. FPR estimates should be interpreted as order-of-magnitude approximations.
    
    \item \textbf{Permutation null limitations:} Our FPR estimates rely primarily on the time-shuffle null. The circular-shift null (1,000 permutations) yielded 0\% FPR after Bonferroni correction, confirming PAR(2) is not falsely detecting phase-gating from autocorrelation alone.
    
    \item \textbf{Predictive validation assessed:} PAR(2) does not consistently outperform reduced AR(2) in out-of-sample prediction (45.2\% win rate), clarifying its role as a discovery engine rather than a forecasting model.
    
    \item \textbf{Phase estimation sensitivity:} Clock gene phase was estimated using fixed 24-hour cosinor regression. Alternative phase estimators were not evaluated.
    
    \item \textbf{Limited temporal resolution:} Most datasets have 6--24 time points spanning 24--48 hours.
    
    \item \textbf{Bulk tissue averaging:} All analyzed datasets represent bulk tissue measurements, averaging over heterogeneous cell populations. Single-cell circadian transcriptomics may reveal cell-type-specific gating patterns obscured in bulk data \citep{droin2019space}.
    
    \item \textbf{Exogeneity assumption:} PAR(2) treats clock phase as exogenous (clock$\rightarrow$target), but bidirectional regulation is well-documented. Causal tests (perturbation studies, Granger causality) are needed.
    
    \item \textbf{Cancer model generalization:} Hierarchy reversal was observed in only two cancer models. Replication across $\geq$3 independent cancer types is needed.
    
    \item \textbf{Genotype-induced composition shifts:} Bulk organoid profiles may conflate rhythmic dynamics with shifts in cell-type proportions between genotypes.
    
    \item \textbf{Pparg f$^2$ represents one finding:} The 8 nominally significant Pparg pairs reflect TTFL collinearity, not 8 independent findings.
    
    \item \textbf{Panel-based rather than genome-wide:} We tested 23 cancer-relevant genes; genome-wide analysis could reveal additional conserved relationships.
    
    \item \textbf{Period assumption:} Phase estimation depends on a fixed 24-hour period assumption. Period sensitivity analysis (T$\in$\{20--28\}h) showed robust eigenvalue separation, but alternative phase estimators were not evaluated.
    
    \item \textbf{Observational nature:} The atlas is descriptive; circadian gating relationships are correlational and causal claims require experimental validation.
    
    \item \textbf{Mouse tissues only:} Human tissue-specific gating patterns may differ.
    
    \item \textbf{Root-space geography:} The functional geography analysis is based on a curated panel. Formal clustering statistics and enrichment tests have not been applied; the observed correspondence is hypothesis-generating.
    
    \item \textbf{Bulk tissue resolution:} All analyses use bulk tissue or organoid RNA measurements. No publicly available dataset currently combines single-cell resolution with dense circadian time-series sampling required for AR(2) fitting. Complementary single-cell methods---MemorySeq \citep{raj2020memoryseq} and GEMLI \citep{eisele2024gemli}---provide cellular resolution but lack temporal depth; convergence of these approaches would provide definitive validation.
    
    \item \textbf{Golden-ratio analysis is exploratory:} The golden-ratio enrichment is highly tissue-specific and should be considered hypothesis-generating; none of the main conclusions depend on it.
    
    \item \textbf{Detection power is low ($\sim$5\%):} Simulation stress-testing indicates $\sim$4.9\% power under realistic noise. The observed enrichments therefore represent lower bounds on true biological prevalence.
\end{enumerate}

%%%%%%%%%%%%%%%%%%%%%%%%%%%%%%%%%%%%%%%%%%%%%%%%%%%%%%%%%%%%%
\section{Conclusions}
%%%%%%%%%%%%%%%%%%%%%%%%%%%%%%%%%%%%%%%%%%%%%%%%%%%%%%%%%%%%%

This study presents the PAR(2) framework---a permutation- and replication-validated discovery engine for identifying candidate circadian gating relationships in cancer. Pan-tissue analysis reveals that Cry1$\rightarrow$Wee1 is the only broadly conserved clock-cancer gene interaction in our panel across mammalian tissues, identifying G2/M checkpoint timing as a candidate fundamental circadian function. Myc gating, often considered central to circadian tumor suppression, is tissue-restricted. Each tissue deploys the circadian clock to protect its most vulnerable pathways---liver guards genomic integrity, heart controls growth, kidney times proliferation, and brain regions protect neuronal function.

The circadian clock actively compensates for APC mutations by doubling its temporal control over cancer-relevant genes---a previously underappreciated tumor suppression mechanism. This compensation collapses near-completely when BMAL1 is co-deleted, consistent with a two-hit threshold. BMAL1 mutation rewires gating to LGR5, linking the clock to cancer stem cell control. Root-space functional geography provides an additional hypothesis-generating layer: APC's position in the alternating-dynamics region is consistent with its role as a dynamical toggle switch, and PTEN's position at the pure oscillation vertex ($|\lambda| = 0.996$) suggests rhythmic rather than constitutive tumor suppression.

The three-layer eigenvalue hierarchy (Identity $>$ Proliferation $>$ Clock) is causally dependent on a functional molecular clock, robust across datasets and statistical tests. Extension to nine functional categories ($\sim$1,594 genes) reveals a finer hierarchy---Clock $>$ Chromatin $>$ Metabolic $>$ Housekeeping $>$ Immune $>$ Signaling $>$ DNA Repair $>$ Target $>$ Stem Cell---validated by permutation testing ($H=144.8$, $p<0.001$), bootstrap analysis (clock \#1 in 100\% of iterations), detrending ($\rho = 0.819$), and leave-one-tissue-out cross-validation (100\% top-category match). The unexpected ranking of chromatin remodeling above housekeeping genes, and stem cell markers at the bottom, establishes temporal persistence as a biologically informative axis distinct from constitutive expression stability.

In MYC-ON neuroblastoma, Pparg emerges as the top cancer-specific target, with identical effect sizes across clock genes confirming TTFL coherence. The emergent eigenperiod shows apparent healthy-vs-cancer separation, but this requires within-species, within-tissue validation before clinical interpretation. Aging and cancer represent divergent deformations of the circadian hierarchy---aging weakens it while cancer reverses it---with caloric restriction partially restoring the youthful pattern.

These findings establish compensatory circadian gating as a new paradigm in clock-cancer biology and suggest that chronotherapy strategies should consider mutation-specific clock rewiring. However, we emphasize that PAR(2) is a descriptive framework that improves in-sample explanatory power but does not robustly improve out-of-sample prediction; these constraints limit its current utility for clinical forecasting applications. All findings require independent experimental validation before clinical applications can be considered.

\subsection{Gene-Level Eigenvalue Atlas with Validation}

To validate category-level findings at single-gene resolution, we constructed a per-gene eigenvalue atlas for 212 classified genes across five tissues (GSE54650: Liver, Kidney, Heart, Lung, Brown Fat) and two neuroblastoma conditions (GSE221103: MYC-ON, MYC-OFF).

\textbf{$\varphi$-zone analysis (negative result).} Defining a $\varphi$-zone as $|\lambda| \in [0.603, 0.633]$ ($\pm 0.015$ of $1/\varphi = 0.618$), 53/212 genes (25.0\%) fell within this zone in at least one tissue. Permutation testing (5{,}000 iterations): $p = 0.1544$, \textbf{not significant} (null expectation: 23.8\%). No category showed significant $\varphi$-enrichment after Benjamini-Hochberg correction (Immune: 10/24 = 41.7\%, raw $p = 0.0426$, adjusted $p = 0.3834$; all others raw $p > 0.05$). Bootstrap CIs for 56 near-$\varphi$ gene-tissue pairs all contained 0.618 but were wide ($\sim$0.38), consistent with proximity but not specifically pinpointing $\varphi$. Cross-tissue CVs were similar (near-$\varphi$: 0.303 vs.\ others: 0.348). \textbf{We conclude that $\varphi$ is an exploratory geometric reference, not a biological attractor; the observed proximity is expected from the eigenvalue distribution.}

\textbf{Category hierarchy validation (positive result).} Mean eigenvalue by category confirmed Clock = 0.628 (highest), all others 0.43--0.47. The Clock $>$ all others ordering was preserved across all five tissues, providing independent per-gene validation of the nine-tier hierarchy.

\textbf{MYC-ON/OFF state-swap (negative $\varphi$ result, positive biology).} Twelve genes swapped $\varphi$-zone status between MYC-ON and MYC-OFF (enter when MYC off: WEE1, TP53, MCM6, CTCF, PROM1, MAPK1; leave: NR1D2, TEF, AXIN2, BAX, MLH1, CRY2). Permutation test: $p = 1.000$, \textbf{not significant}---swaps expected from shift magnitudes. However, the \textit{identities} of swapping genes are biologically coherent: checkpoint/tumor suppressor genes (WEE1, TP53, MAPK1) stabilize toward intermediate persistence when MYC is silenced, while clock components (NR1D2, TEF, CRY2) shift away.

\textbf{Unstable gene recovery (positive result).} Seven genes exceeded $|\lambda| > 1.0$ in MYC-ON: SIRT1 (1.280), KLF4 (1.308), CCNB1 (1.120), HDAC2 (1.058), RAD51 (1.054), EGFR (1.001), DNMT1 (1.001). \textbf{All seven recovered to stable dynamics when MYC was turned off; zero genes were unstable in MYC-OFF.} These genes span metabolic regulation, pluripotency, mitosis, chromatin remodeling, DNA repair, growth signaling, and epigenetic maintenance---collectively representing the hallmarks of cancer. The complete, reversible instability is MYC-dependent and systematic.

\textbf{Summary of validated versus non-validated findings:}
\begin{itemize}[noitemsep]
    \item \textbf{Validated:} (a) Clock $>$ all others eigenvalue hierarchy across tissues; (b) complete unstable gene recovery upon MYC silencing; (c) biologically coherent gene identities in cancer state-swaps.
    \item \textbf{Not validated:} (a) $\varphi$-specific enrichment ($p = 0.1544$); (b) category-specific $\varphi$-zone clustering (no FDR-significant categories); (c) $\varphi$-zone as a biologically meaningful boundary.
\end{itemize}

\subsection*{Future Directions: Genome-Wide Root-Space Applications}

The availability of genome-wide AR(2) eigenvalue maps across 38 datasets opens substantial opportunities beyond the clock--target hierarchy, including: (1) drug target discovery via differential eigenvalue screening, (2) dynamical biomarker identification, (3) function prediction from dynamical position, (4) disease signature mapping as root-space deformations, (5) cross-species evolutionary dynamics, (6) dynamical network inference, (7) wearable and CGM integration, (8) perturbation response prediction, (9) cell-type deconvolution and single-cell validation, (10) temporal pharmacology and chronotherapy optimization, and (11) root-space geometry as a diagnostic metric. Each direction generates specific, falsifiable predictions testable with existing public datasets.

%%%%%%%%%%%%%%%%%%%%%%%%%%%%%%%%%%%%%%%%%%%%%%%%%%%%%%%%%%%%%
\section*{Data Availability}
%%%%%%%%%%%%%%%%%%%%%%%%%%%%%%%%%%%%%%%%%%%%%%%%%%%%%%%%%%%%%

All code, scripts, and processed summary data are available at \\url{https://github.com/mickwh2764/PAR-2--Final-09-12-2025} under Apache License 2.0 for academic and research use; commercial licensing is available upon request. Analyses are fully reproducible using the included Mulberry32 seeded pseudo-random number generator with default seeds. The PAR(2) Discovery Engine web application is accessible at \\url{https://par2-discovery-engine.replit.app}. Raw datasets are available from GEO under accession numbers GSE54650, GSE11923, GSE70499, GSE157357, GSE221103, GSE17739, GSE59396, GSE201207, GSE93903, GSE245295.

%%%%%%%%%%%%%%%%%%%%%%%%%%%%%%%%%%%%%%%%%%%%%%%%%%%%%%%%%%%%%
\section*{Funding}
%%%%%%%%%%%%%%%%%%%%%%%%%%%%%%%%%%%%%%%%%%%%%%%%%%%%%%%%%%%%%

This research was conducted independently without external funding.

%%%%%%%%%%%%%%%%%%%%%%%%%%%%%%%%%%%%%%%%%%%%%%%%%%%%%%%%%%%%%
\section*{Conflicts of Interest}
%%%%%%%%%%%%%%%%%%%%%%%%%%%%%%%%%%%%%%%%%%%%%%%%%%%%%%%%%%%%%

The PAR(2) methodology is subject to a pending UK patent application (priority date established prior to public disclosure). The author declares no other conflicts of interest.

%%%%%%%%%%%%%%%%%%%%%%%%%%%%%%%%%%%%%%%%%%%%%%%%%%%%%%%%%%%%%
\section*{Author Contributions}
%%%%%%%%%%%%%%%%%%%%%%%%%%%%%%%%%%%%%%%%%%%%%%%%%%%%%%%%%%%%%

M.W.: Conceptualization, Methodology, Software, Validation, Formal Analysis, Data Curation, Writing -- Original Draft, Writing -- Review \& Editing, Visualization.

%%%%%%%%%%%%%%%%%%%%%%%%%%%%%%%%%%%%%%%%%%%%%%%%%%%%%%%%%%%%%
\section*{Acknowledgments}
%%%%%%%%%%%%%%%%%%%%%%%%%%%%%%%%%%%%%%%%%%%%%%%%%%%%%%%%%%%%%

We thank the creators of the public datasets analyzed in this study: the Hughes laboratory for the Circadian Atlas (GSE54650) and the high-resolution liver dataset (GSE11923), Storch et al.\ for the Bmal1-KO data (GSE70499), and the investigators responsible for GSE157357, GSE221103, GSE17739, GSE59396, GSE201207, GSE93903, and GSE245295. We also acknowledge the broader circadian biology and cancer research communities for establishing the conceptual foundations upon which this work builds.

%%%%%%%%%%%%%%%%%%%%%%%%%%%%%%%%%%%%%%%%%%%%%%%%%%%%%%%%%%%%%
\bibliographystyle{unsrt}
\begin{thebibliography}{80}

\bibitem{takahashi2017}
Takahashi JS. Transcriptional architecture of the mammalian circadian clock. \textit{Nat Rev Genet}. 2017;18(3):164--179.

\bibitem{bass2010circadian}
Bass J, Takahashi JS. Circadian integration of metabolism and energetics. \textit{Science}. 2010;330(6009):1349--1354.

\bibitem{reppert2002}
Reppert SM, Weaver DR. Coordination of circadian timing in mammals. \textit{Nature}. 2002;418(6901):935--941.

\bibitem{gekakis1998role}
Gekakis N, Staknis D, Nguyen HB, et al. Role of the CLOCK protein in the mammalian circadian mechanism. \textit{Science}. 1998;280(5369):1564--1569.

\bibitem{kume1999mcry1}
Kume K, Zylka MJ, Sriram S, et al. mCRY1 and mCRY2 are essential components of the negative limb of the circadian clock feedback loop. \textit{Cell}. 1999;98(2):193--205.

\bibitem{zhang2014}
Zhang R, Lahens NF, Ballance HI, Hughes ME, Hogenesch JB. A circadian gene expression atlas in mammals. \textit{Proc Natl Acad Sci USA}. 2014;111(45):16219--16224.

\bibitem{straif2007}
Straif K, et al. Carcinogenicity of shift-work, painting, and fire-fighting. \textit{Lancet Oncol}. 2007;8(12):1065--1066.

\bibitem{schernhammer2003night}
Schernhammer ES, Laden F, Speizer FE, et al. Night-shift work and risk of colorectal cancer in the Nurses' Health Study. \textit{J Natl Cancer Inst}. 2003;95(11):825--828.

\bibitem{fu2002}
Fu L, Pelicano H, Liu J, Huang P, Lee CC. The circadian gene Period2 plays an important role in tumor suppression and DNA damage response in vivo. \textit{Cell}. 2002;111(1):41--50.

\bibitem{matsuo2003}
Matsuo T, et al. Control mechanism of the circadian clock for timing of cell division in vivo. \textit{Science}. 2003;302(5643):255--259.

\bibitem{gery2006circadian}
Gery S, Komatsu N, Baldjyan L, et al. The circadian gene Per1 plays an important role in cell growth and DNA damage control in human cancer cells. \textit{Mol Cell}. 2006;22(3):375--382.

\bibitem{masri2015circadian}
Masri S, Sassone-Corsi P. The emerging link between cancer, metabolism, and circadian rhythms. \textit{Nat Med}. 2018;24(12):1795--1803.

\bibitem{janich2011}
Janich P, et al. The circadian molecular clock creates epidermal stem cell heterogeneity. \textit{Nature}. 2011;480(7376):209--214.

\bibitem{rosen2026}
Rosen SJ, Witteveen O, Baxter N, Lach RS, Hopkins E, Bauer M, Wilson MZ. Anti-resonance in developmental signaling regulates cell fate decisions. \textit{eLife}. 2026;14:RP107794.

\bibitem{chun2022}
Chun SK, et al. Disruption of the circadian clock drives Apc loss of heterozygosity to accelerate colorectal cancer. \textit{Sci Adv}. 2022;8(32):eabo2389.

\bibitem{zaidi2010mitotic}
Zaidi SK, Young DW, Montecino MA, et al. Mitotic bookmarking of genes: a novel dimension to epigenetic control. \textit{Nat Rev Genet}. 2010;11(8):583--589.

\bibitem{zhu2023mitotic}
Zhu F, Farnung L, Kaasinen E, et al. Mitotic bookmarking by SWI/SNF subunits. \textit{Nature}. 2023;618(7967):580--587.

\bibitem{bmcbiol2024memory}
Puri S, Kumar V. Differentiation is accompanied by a progressive loss in transcriptional memory. \textit{BMC Biology}. 2024;22:67.

\bibitem{box1994}
Box GEP, Jenkins GM, Reinsel GC. \textit{Time Series Analysis: Forecasting and Control}. 3rd ed. Prentice Hall; 1994.

\bibitem{hurd2007}
Hurd HL, Miamee A. \textit{Periodically Correlated Random Sequences: Spectral Theory and Practice}. Wiley; 2007.

\bibitem{paap2025shrinkage}
Paap R, Franses PH. Shrinkage estimators for periodic autoregressions. \textit{J Econometrics}. 2025;247:103888.

\bibitem{lopezdelacalle2005partsm}
L\'opez-de-Lacalle J. Periodic Autoregressive Time Series Models in R: The partsm Package. \textit{University of the Basque Country Working Papers}. 2005.

\bibitem{cornelissen2014cosinor}
Cornelissen G. Cosinor-based rhythmometry. \textit{Theor Biol Med Model}. 2014;11(1):16.

\bibitem{benjamini1995}
Benjamini Y, Hochberg Y. Controlling the false discovery rate. \textit{J R Stat Soc B}. 1995;57(1):289--300.

\bibitem{cohen1988statistical}
Cohen J. \textit{Statistical Power Analysis for the Behavioral Sciences}. 2nd ed. Lawrence Erlbaum Associates; 1988.

\bibitem{wold1938study}
Wold H. A Study in the Analysis of Stationary Time Series. Stockholm: Almqvist \& Wiksell; 1938.

\bibitem{smallbone2014crosstalk}
Smallbone K, Corfe BM. A mathematical model of the colon crypt capturing compositional dynamic interactions between cell types. \textit{Int J Exp Pathol}. 2014;95(1):1--7.

\bibitem{vanleeuwen2007wnt}
van Leeuwen IMM, Byrne HM, Jensen OE, King JR. Elucidating the interactions between the adhesive and transcriptional functions of $\beta$-catenin in normal and cancerous cells. \textit{J Theor Biol}. 2007;247(1):77--102.

\bibitem{leloup2003}
Leloup JC, Goldbeter A. Toward a detailed computational model for the mammalian circadian clock. \textit{Proc Natl Acad Sci USA}. 2003;100(12):7051--7056.

\bibitem{burnham2002}
Burnham KP, Anderson DR. \textit{Model Selection and Multimodel Inference}. 2nd ed. Springer; 2002.

\bibitem{boman2025}
Boman RM, Schleiniger G, Raymond C, Palazzo JP, Shehab A, Boman BM. A tissue renewal-based mechanism drives colon tumorigenesis. \textit{Cancers}. 2025;18(1):44.

\bibitem{whiteside2026zenodo}
Whiteside M. PAR(2) Discovery Engine: Phase-gated autoregressive analysis of circadian gene expression. Zenodo. 2026.

\bibitem{altman2015}
Altman BJ, et al. MYC disrupts the circadian clock and metabolism in cancer cells. \textit{Cell Metab}. 2015;22(6):1009--1019.

\bibitem{barker2007}
Barker N, van Es JH, Kuipers J, et al. Identification of stem cells in small intestine and colon by marker gene Lgr5. \textit{Nature}. 2007;449(7165):1003--1007.

\bibitem{storch2007}
Storch KF, et al. Intrinsic circadian clock of the mammalian retina. \textit{Cell}. 2007;130(4):730--741.

\bibitem{purvis2012}
Purvis JE, Karhohs KW, Mock C, Batchelor E, Loewer A, Lahav G. p53 dynamics control cell fate. \textit{Science}. 2012;336(6087):1440--1444.

\bibitem{mcgowan1995}
McGowan CH, Russell P. Cell cycle regulation of human WEE1. \textit{EMBO J}. 1995;14(10):2166--2175.

\bibitem{leijen2016}
Leijen S, et al. Phase II study of WEE1 inhibitor AZD1775. \textit{J Clin Oncol}. 2016;34(36):4354--4361.

\bibitem{shmueli2010}
Shmueli G. To explain or to predict? \textit{Stat Sci}. 2010;25(3):289--310.

\bibitem{hanahan2011}
Hanahan D, Weinberg RA. Hallmarks of cancer: the next generation. \textit{Cell}. 2011;144(5):646--674.

\bibitem{wang2008}
Wang N, et al. Vascular PPAR$\gamma$ controls circadian variation in blood pressure. \textit{Cell Metab}. 2008;8(6):482--491.

\bibitem{yang2006}
Yang X, et al. Nuclear receptor expression links the circadian clock to metabolism. \textit{Cell}. 2006;126(4):801--810.

\bibitem{pavlova2016}
Pavlova NN, Thompson CB. The emerging hallmarks of cancer metabolism. \textit{Cell Metab}. 2016;23(1):27--47.

\bibitem{michalik2004}
Michalik L, et al. Peroxisome-proliferator-activated receptors and cancers. \textit{Nat Rev Cancer}. 2004;4(1):61--70.

\bibitem{sato2017}
Sato S, et al. Circadian reprogramming in the liver identifies metabolic pathways of aging. \textit{Cell}. 2017;170(4):664--677.

\bibitem{chalhoub2009}
Chalhoub N, Baker SJ. PTEN and the PI3-kinase pathway in cancer. \textit{Annu Rev Pathol}. 2009;4:127--150.

\bibitem{fodde2002}
Fodde R. The APC gene in colorectal cancer. \textit{Eur J Cancer}. 2002;38(7):867--871.

\bibitem{geyfman2012brain}
Geyfman M, Kumar V, Liu Q, et al. Brain and muscle Arnt-like protein-1 (BMAL1) controls circadian cell proliferation and susceptibility to UVB-induced DNA damage in the epidermis. \textit{PNAS}. 2012;109(29):11758--11763.

\bibitem{scheffer2009early}
Scheffer M, Bascompte J, Brock WA, et al. Early-warning signals for critical transitions. \textit{Nature}. 2009;461(7260):53--59.

\bibitem{chen2012biomarkers}
Chen L, Liu R, Liu ZP, Li M, Aihara K. Detecting early-warning signals for sudden deterioration of complex diseases by dynamical network biomarkers. \textit{Sci Rep}. 2012;2:342.

\bibitem{waddington1957}
Waddington CH. \textit{The Strategy of the Genes}. Allen \& Unwin; 1957.

\bibitem{dang2009}
Dang L, et al. Cancer-associated IDH1 mutations produce 2-hydroxyglutarate. \textit{Nature}. 2009;462(7274):739--744.

\bibitem{allis2016}
Allis CD, Jenuwein T. The molecular hallmarks of epigenetic control. \textit{Nat Rev Genet}. 2016;17(8):487--500.

\bibitem{sancar2010}
Sancar A, Lindsey-Boltz LA, Kang TH, Reardon JT, Lee JH, Ozturk N. Circadian clock control of the cellular response to DNA damage. \textit{FEBS Lett}. 2010;584(12):2618--2625.

\bibitem{kolch2005}
Kolch W. Coordinating ERK/MAPK signalling through scaffolds and inhibitors. \textit{Nat Rev Mol Cell Biol}. 2005;6(11):827--837.

\bibitem{briscoe2013}
Briscoe J, Th\'erond PP. The mechanisms of Hedgehog signalling and its roles in development and disease. \textit{Nat Rev Mol Cell Biol}. 2013;14(7):416--429.

\bibitem{elowitz2000}
Elowitz MB, Leibler S. A synthetic oscillatory network of transcriptional regulators. \textit{Nature}. 2000;403(6767):335--338.

\bibitem{jensen2003}
Jensen MH, Sneppen K, Tiana G. Sustained oscillations and time delays in gene expression of protein Hes1. \textit{FEBS Lett}. 2003;541(1--3):176--177.

\bibitem{droin2019space}
Droin C, Paquet ER, Naef F. Low-dimensional dynamics of two coupled biological oscillators. \textit{Nat Phys}. 2019;15(10):1086--1094.

\bibitem{fujimoto2025}
Fujimoto K, et al. Personalized chronotherapy in glioblastoma: integrating circadian profiling and PK--PD modelling to optimize temozolomide timing. \textit{npj Precis Oncol}. 2025;9:205.

\bibitem{newman2015}
Newman AM, et al. Robust enumeration of cell subsets from tissue expression profiles. \textit{Nature Methods}. 2015;12(5):453--457.

\bibitem{avila2018}
Avila Cobos F, et al. Computational deconvolution of transcriptomics data from mixed cell populations. \textit{Bioinformatics}. 2018;34(11):1969--1979.

\bibitem{ripperger2000}
Ripperger JA, Shearman LP, Reppert SM, Schibler U. CLOCK controls expression of the circadian transcription factor DBP. \textit{Genes Dev}. 2000;14(6):679--689.

\bibitem{wuarin1990}
Wuarin J, Schibler U. Expression of the liver-enriched transcriptional activator protein DBP follows a stringent circadian rhythm. \textit{Cell}. 1990;63(6):1257--1266.

\bibitem{gachon2006}
Gachon F, et al. The circadian PAR-domain basic leucine zipper transcription factors modulate xenobiotic detoxification. \textit{Cell Metab}. 2006;4(1):25--36.

\bibitem{reick2001}
Reick M, Garcia JA, Dudley C, McKnight SL. NPAS2: an analog of clock operative in the mammalian forebrain. \textit{Science}. 2001;293(5529):506--509.

\bibitem{debruyne2007}
DeBruyne JP, Weaver DR, Reppert SM. CLOCK and NPAS2 have overlapping roles in the suprachiasmatic circadian clock. \textit{Nat Neurosci}. 2007;10(5):543--545.

\bibitem{takeda2012}
Takeda Y, et al. ROR$\gamma$ directly regulates the circadian expression of clock genes and downstream targets in vivo. \textit{Nucleic Acids Res}. 2012;40(17):8519--8535.

\bibitem{lowrey2011genetics}
Lowrey PL, Takahashi JS. Genetics of circadian rhythms in mammalian model organisms. \textit{Adv Genet}. 2011;74:175--230.

\bibitem{vandermaaten2008}
van der Maaten L, Hinton G. Visualizing data using t-SNE. \textit{J Mach Learn Res}. 2008;9:2579--2605.

\bibitem{wang2011}
Wang J, Zhang K, Xu L, Wang E. Quantifying the Waddington landscape and biological paths for development and differentiation. \textit{Proc Natl Acad Sci USA}. 2011;108(20):8257--8262.

\bibitem{raj2020memoryseq}
Shaffer SM, Emert BL, Reyes Hueros RA, et al. Memory sequencing reveals heritable single-cell gene expression programs associated with distinct cellular behaviors. \textit{Cell}. 2020;182(4):947--959.

\bibitem{eisele2024gemli}
Eisele AS, Tarbier M, Dormann AA, Pelechano V, Suter DM. Gene-expression memory-based prediction of cell lineages from scRNA-seq datasets. \textit{Nat Commun}. 2024;15:2744.

\bibitem{granada2024phenotyping}
Alers I, Schmitt K, Adamovich Y, Tsimring LS, Asher G, Granada AE. Time-of-day effects of cancer drugs revealed by high-throughput deep phenotyping. \textit{Nat Commun}. 2024;15:7085.

\bibitem{granada2025drivers}
Alers I, Schmitt K, Granada AE. A combined mathematical and experimental approach reveals the drivers of time-of-day drug sensitivity in human cells. \textit{Commun Biol}. 2025;8:452.

\bibitem{boman2025fibonacci}
Boman BM. How Does Multicellular Life Happen? Modeling Fibonacci Patterns in Biological Tissues. \textit{The Fibonacci Quarterly}. 2025;September issue.

\bibitem{liu2025genesdiseases}
Liu J, Jiang Z, Zha J, Lin Q, He W. Crosstalk between the circadian clock, intestinal stem cell niche, and epithelial cell fate decision. \textit{Genes \& Diseases}. 2025;12(6):101650.

\bibitem{cellcommsignal2025per}
The role of circadian rhythm regulator PERs in oxidative stress, immunity, and cancer development. \textit{Cell Commun Signal}. 2025;23:30.

\bibitem{li2021per3wnt}
Li Q, et al. Circadian Rhythm Gene PER3 Negatively Regulates Stemness of Prostate Cancer Stem Cells via WNT/$\beta$-Catenin Signaling. \textit{Front Cell Dev Biol}. 2021;9:656981.

\bibitem{hughes2010jtk}
Hughes ME, Hogenesch JB, Kornacker K. JTK\_CYCLE: an efficient nonparametric algorithm for detecting rhythmic components in genome-scale data sets. \textit{J Biol Rhythms}. 2010;25(5):372--380.

\bibitem{thaben2014rain}
Thaben PF, Westermark PO. Detecting rhythms in time series with RAIN. \textit{J Biol Rhythms}. 2014;29(6):391--400.

\bibitem{wu2016metacycle}
Wu G, Anafi RC, Hughes ME, Kornacker K, Hogenesch JB. MetaCycle: an integrated R package to evaluate periodicity in large scale data. \textit{Bioinformatics}. 2016;32(21):3351--3353.

\bibitem{bues2022disco}
Bues J, et al. Deterministic scRNA-seq captures variation in intestinal crypt and organoid composition. \textit{Nat Methods}. 2022;19(3):323--330.

\bibitem{fujita2007}
Fujita A, et al. Modeling gene expression regulatory networks with the sparse vector autoregressive model. \textit{BMC Syst Biol}. 2007;1:39.

\bibitem{sulli2018interplay}
Sulli G, Manoogian ENC, Taub PR, Panda S. Training the circadian clock, clocking the drugs, and drugging the clock. \textit{Trends Pharmacol Sci}. 2018;39(9):812--827.

\bibitem{ye2018orchestration}
Ye Y, Xiang Y, Ozguc FM, et al. The genomic landscape and pharmacogenomic interactions of clock genes in cancer chronotherapy. \textit{Cell Syst}. 2018;6(3):314--328.

\end{thebibliography}

\newpage
\section*{Supplementary Tables}

\subsection*{Table S1: Complete tissue-by-tissue significant findings (selected)}

\begin{longtable}{lllcc}
\toprule
\textbf{Tissue} & \textbf{Clock Gene} & \textbf{Target Gene} & \textbf{Conservation} & \textbf{P-value} \\
\midrule
\endfirsthead
\multicolumn{5}{c}{\textit{Table S1 continued}} \\
\toprule
\textbf{Tissue} & \textbf{Clock Gene} & \textbf{Target Gene} & \textbf{Conservation} & \textbf{P-value} \\
\midrule
\endhead
\midrule
\multicolumn{5}{r}{\textit{Continued on next page}} \\
\endfoot
\bottomrule
\endlastfoot
Liver & Cry1 & Wee1 & Conserved ($\geq$6) & 0.0104 \\
Liver & Nr1d1 & Wee1 & Moderate (3-5) & 0.0104 \\
Liver & Cry2 & Wee1 & Moderate (3-5) & 0.0110 \\
Brown Fat & Cry1 & Wee1 & Conserved ($\geq$6) & 0.0100 \\
Brown Fat & Nr1d2 & Chek2 & Tissue-specific & 0.0076 \\
White Fat & Cry1 & Wee1 & Conserved ($\geq$6) & 0.0130 \\
Aorta & Cry1 & Wee1 & Conserved ($\geq$6) & 0.0105 \\
Heart & Cry1 & Wee1 & Conserved ($\geq$6) & 0.0095 \\
Heart & Nr1d1 & Tead1 & Tissue-specific & 0.0061 \\
Lung & Cry1 & Wee1 & Conserved ($\geq$6) & 0.0190 \\
Brainstem & Cry1 & Wee1 & Conserved ($\geq$6) & 0.0085 \\
Muscle & Cry1 & Wee1 & Conserved ($\geq$6) & 0.0310 \\
Adrenal & Cry1 & Wee1 & Conserved ($\geq$6) & 0.0015 \\
Adrenal & Nr1d2 & Wee1 & Moderate (3-5) & 0.0006 \\
Cerebellum & Cry1 & Wee1 & Conserved ($\geq$6) & 0.0200 \\
Cerebellum & Nr1d1 & Cdk1 & Tissue-specific & 0.0110 \\
Hypothalamus & Cry1 & Wee1 & Conserved ($\geq$6) & 0.0480 \\
Kidney & Nr1d2 & Myc & Tissue-specific & 0.0431 \\
\end{longtable}

\subsection*{Table S2: Model Order Comparison by Information Criteria}

\begin{table}[H]
\centering
\caption{\textbf{AIC/BIC model comparison across datasets.} $\Delta$AIC$_{2-1}$ = AIC(AR2) $-$ AIC(AR1); negative values favor AR(2).}
\label{tab:aic_comparison}
\begin{tabular}{lcccccc}
\toprule
\textbf{Dataset} & \textbf{$N$ genes} & \textbf{Median $\Delta$AIC$_{2-1}$} & \textbf{Median $\Delta$BIC$_{2-1}$} & \textbf{Mean $R^2$ AR(1)} & \textbf{Mean $R^2$ AR(2)} & \textbf{Hierarchy} \\
\midrule
Liver (GSE54650) & 35 & $-$11.3 & $-$10.3 & 0.276 & 0.352 & Both \\
Liver 48h (GSE11923) & 36 & $-$12.2 & $-$10.4 & 0.346 & 0.367 & Both \\
Kidney (GSE54650) & 35 & $-$10.0 & $-$9.0 & 0.328 & 0.420 & Both \\
Heart (GSE54650) & 35 & $-$10.7 & $-$9.7 & 0.311 & 0.375 & Both \\
Lung (GSE54650) & 35 & $-$10.2 & $-$9.2 & 0.347 & 0.441 & Both \\
Organoid WT & 37 & $-$4.7 & $-$3.7 & 0.146 & 0.191 & Both \\
Organoid APC-KO & 37 & $-$9.0 & $-$8.0 & 0.148 & 0.227 & AR(1) only \\
Human Blood & 36 & $-$5.3 & $-$4.3 & 0.289 & 0.322 & AR(1) only \\
\bottomrule
\end{tabular}
\end{table}

\end{document}
