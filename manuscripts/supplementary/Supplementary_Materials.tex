\documentclass[11pt,a4paper]{article}
\usepackage[utf8]{inputenc}
\usepackage[T1]{fontenc}
\usepackage{geometry}
\usepackage{graphicx}
\usepackage{amsmath,amssymb}
\usepackage{booktabs}
\usepackage{longtable}
\usepackage{hyperref}
\usepackage{float}
\usepackage{caption}
\usepackage{array}
\usepackage{pdflscape}

\geometry{margin=1in}

\title{\textbf{Supplementary Materials}\\[0.5em]
\large PAR(2): A Phase-Gated Autoregressive Framework Reveals Tissue-Specific Circadian Gating of Cancer-Relevant Genes Across Mammalian Tissues\\[0.3em]
\normalsize Companion to: Compensatory Circadian Gating Suggests a Two-Hit Threshold for Clock-Mediated Cancer Defense}

\author{}
\date{}

\begin{document}

\maketitle

\tableofcontents
\newpage

\section{Supplementary Methods}

\subsection{S1.1 Complete Mathematical Derivation}

The PAR(2) model extends the standard AR(2) process by incorporating phase-dependent coefficients. Starting from the general formulation:

\begin{equation}
R_n = \alpha_0 + \alpha_1(\Phi_{n-1})R_{n-1} + \alpha_2(\Phi_{n-2})R_{n-2} + \varepsilon_n
\end{equation}

We expand the phase-dependent coefficients using a Fourier basis (truncated to first harmonic):

\begin{align}
\alpha_1(\Phi) &= \beta_{1,0} + \beta_{1,c}\cos(\Phi) + \beta_{1,s}\sin(\Phi) \\
\alpha_2(\Phi) &= \beta_{2,0} + \beta_{2,c}\cos(\Phi) + \beta_{2,s}\sin(\Phi)
\end{align}

The full regression model becomes:
\begin{align}
R_n = &\; \beta_0 + \beta_{1,0}R_{n-1} + \beta_{2,0}R_{n-2} \nonumber \\
      &+ \beta_{1,c}R_{n-1}\cos(\Phi_{n-1}) + \beta_{1,s}R_{n-1}\sin(\Phi_{n-1}) \nonumber \\
      &+ \beta_{2,c}R_{n-2}\cos(\Phi_{n-2}) + \beta_{2,s}R_{n-2}\sin(\Phi_{n-2}) + \varepsilon_n
\end{align}

\subsection{S1.2 Eigenperiod Calculation}

From the fitted coefficients $\hat{\beta}_{1,0}$ and $\hat{\beta}_{2,0}$, we compute eigenvalues via:

\begin{equation}
\lambda = \frac{\hat{\beta}_{1,0} \pm \sqrt{\hat{\beta}_{1,0}^2 + 4\hat{\beta}_{2,0}}}{2}
\end{equation}

For complex eigenvalues $\lambda = r e^{i\theta}$:
\begin{itemize}
    \item Modulus: $r = |\lambda| = \sqrt{\text{Re}(\lambda)^2 + \text{Im}(\lambda)^2}$
    \item Argument: $\theta = \arctan\left(\frac{\text{Im}(\lambda)}{\text{Re}(\lambda)}\right)$
    \item Eigenperiod: $T = \frac{2\pi}{\theta} \times \Delta t$
    \item Half-life: $t_{1/2} = \frac{\ln(2)}{\ln(1/r)} \times \Delta t$
\end{itemize}

Stability criterion: $r < 1$ indicates convergent (stable) dynamics.

\subsection{S1.3 Multiple Testing Correction}

\textbf{Stage 1 - Within-pair Bonferroni:}
For each gene pair, we test 4 phase interaction terms. The Bonferroni-corrected threshold is:
\[
\alpha_{Bonf} = \frac{0.05}{4} = 0.0125
\]

\textbf{Stage 2 - Across-pair FDR:}
All Bonferroni-corrected p-values within a dataset are subjected to Benjamini-Hochberg FDR control at $\alpha = 0.05$.

\subsection{S1.4 Effect Size Interpretation}

Cohen's f$^2$ is calculated as:
\[
f^2 = \frac{R^2_{full} - R^2_{reduced}}{1 - R^2_{full}}
\]

Interpretation thresholds (Cohen, 1988):
\begin{itemize}
    \item Small: $f^2 \geq 0.02$
    \item Medium: $f^2 \geq 0.15$
    \item Large: $f^2 \geq 0.35$
\end{itemize}

\section{Supplementary Tables}

\subsection{Table S1: Complete Dataset Characteristics}

\begin{longtable}{llcccl}
\toprule
Dataset ID & Tissue/Condition & Genes & Timepoints & Interval & Source \\
\midrule
\endfirsthead
\toprule
Dataset ID & Tissue/Condition & Genes & Timepoints & Interval & Source \\
\midrule
\endhead
GSE54650\_Adrenal & Adrenal & 21,000 & 24 & 2h & GEO \\
GSE54650\_Aorta & Aorta & 21,000 & 24 & 2h & GEO \\
GSE54650\_Brainstem & Brainstem & 21,000 & 24 & 2h & GEO \\
GSE54650\_BrownFat & Brown Fat & 21,000 & 24 & 2h & GEO \\
GSE54650\_Cerebellum & Cerebellum & 21,000 & 24 & 2h & GEO \\
GSE54650\_Heart & Heart & 21,000 & 24 & 2h & GEO \\
GSE54650\_Hypothalamus & Hypothalamus & 21,000 & 24 & 2h & GEO \\
GSE54650\_Kidney & Kidney & 21,000 & 24 & 2h & GEO \\
GSE54650\_Liver & Liver & 21,000 & 24 & 2h & GEO \\
GSE54650\_Lung & Lung & 21,000 & 24 & 2h & GEO \\
GSE54650\_Muscle & Muscle & 21,000 & 24 & 2h & GEO \\
GSE54650\_WhiteFat & White Fat & 21,000 & 24 & 2h & GEO \\
GSE157357\_WT & WT Organoid & 15,000 & 6 & 4h & GEO \\
GSE157357\_APC-Mut & APC-Mut Organoid & 15,000 & 6 & 4h & GEO \\
GSE157357\_BMAL-Mut & BMAL-Mut Organoid & 15,000 & 6 & 4h & GEO \\
GSE157357\_Double-Mut & APC+BMAL Mut & 15,000 & 6 & 4h & GEO \\
GSE221103\_MYC-ON & Neuroblastoma (cancer) & 60,000 & 14 & 4h & GEO \\
GSE221103\_MYC-OFF & Neuroblastoma (control) & 60,000 & 14 & 4h & GEO \\
GSE17739\_DCT & Kidney DCT & 21,500 & 6 & 4h & GEO \\
GSE17739\_CCD & Kidney CCD & 21,500 & 6 & 4h & GEO \\
GSE59396\_Basal & Lung Basal & 17,000 & 12 & 4h & GEO \\
\bottomrule
\caption{Complete list of embedded datasets with characteristics}
\end{longtable}

\subsection{Table S2: Eigenperiod Summary by Tissue}

\begin{longtable}{lccccc}
\toprule
Tissue & Mean Period & Median & Min & Max & \% Stable \\
\midrule
\endfirsthead
Cerebellum & 7.2h & 7.1h & 5.9h & 10.2h & 100\% \\
Hypothalamus & 7.6h & 7.4h & 6.1h & 9.8h & 100\% \\
Brainstem & 8.4h & 8.9h & 5.9h & 10.2h & 100\% \\
Adrenal & 9.6h & 7.7h & 4.6h & 22.8h & 100\% \\
White Fat & 9.8h & 8.2h & 5.2h & 18.4h & 100\% \\
Liver & 10.4h & 8.4h & 5.3h & 21.1h & 100\% \\
Muscle & 11.1h & 9.5h & 6.4h & 19.6h & 100\% \\
Aorta & 11.4h & 7.9h & 5.2h & 44.4h & 100\% \\
Lung (GSE54650) & 12.1h & 10.2h & 5.9h & 25.7h & 100\% \\
Kidney & 12.2h & 9.8h & 5.5h & 30.2h & 100\% \\
Brown Fat & 12.4h & 10.1h & 5.3h & 28.9h & 100\% \\
Heart & 13.3h & 11.5h & 6.2h & 32.5h & 100\% \\
\midrule
Neuroblastoma MYC-ON & 22.7h & 20.8h & 12.2h & 34.8h & 42\% \\
Neuroblastoma MYC-OFF & 23.4h & 21.1h & 13.5h & 44.1h & 58\% \\
Lung (GSE59396) & 25.7h & 23.2h & 14.5h & 48.2h & 85\% \\
\bottomrule
\caption{Eigenperiod statistics across all analyzed tissues}
\end{longtable}

\subsection{Table S3: Permutation Test Results}

\begin{longtable}{llcccc}
\toprule
Dataset & Null Model & FPR & Mean Sig Rate & Null Eigenperiod & Real Eigenperiod \\
\midrule
\endfirsthead
Liver & time-shuffle & 80\% & 4.9\% & 7.4h & 10.4h \\
Liver & pair-shuffle & 100\% & 11.0\% & 10.4h & 10.4h \\
Liver & phase-scramble & 100\% & 10.9\% & 10.4h & 10.4h \\
Liver & synthetic-null & 90\% & 10.9\% & 10.1h & 10.4h \\
\midrule
Heart & time-shuffle & 80\% & 5.9\% & 7.4h & 13.3h \\
Heart & pair-shuffle & 100\% & 16.9\% & 13.3h & 13.3h \\
Heart & phase-scramble & 100\% & 15.5\% & 13.3h & 13.3h \\
Heart & synthetic-null & 90\% & 10.9\% & 10.1h & 13.3h \\
\midrule
Neuroblastoma & time-shuffle & 90\% & 5.0\% & 16.3h & 22.7h \\
Neuroblastoma & pair-shuffle & 100\% & 21.1\% & 22.7h & 22.7h \\
Neuroblastoma & phase-scramble & 100\% & 20.9\% & 22.7h & 22.7h \\
Neuroblastoma & synthetic-null & 100\% & 8.6\% & 23.1h & 22.7h \\
\bottomrule
\caption{Permutation test results showing false positive rates under different null models}
\end{longtable}

\subsection{Table S4: CANDIDATE Tier Gene Pairs}

\begin{longtable}{llllccc}
\toprule
Target & Clock & Dataset & Significant Terms & p-value & q-value & f$^2$ \\
\midrule
\endfirsthead
Pparg & Per2 & Neuroblastoma MYC-ON & R1c,R1s,R2s & 0.0033 & 0.045 & 10.86 \\
Pparg & Cry2 & Neuroblastoma MYC-ON & R1c,R1s,R2s & 0.0033 & 0.045 & 10.86 \\
Pparg & Per1 & Neuroblastoma MYC-ON & R1c,R1s,R2s & 0.0034 & 0.045 & 10.86 \\
Pparg & Nr1d2 & Neuroblastoma MYC-ON & R1c,R1s,R2c,R2s & 0.0034 & 0.045 & 10.86 \\
Pparg & Arntl & Neuroblastoma MYC-ON & R1c,R1s,R2c,R2s & 0.0035 & 0.045 & 10.86 \\
Pparg & Cry1 & Neuroblastoma MYC-ON & R1s,R2c,R2s & 0.0035 & 0.045 & 10.86 \\
Pparg & Nr1d1 & Neuroblastoma MYC-ON & R1c,R1s,R2c & 0.0036 & 0.045 & 10.86 \\
Pparg & Clock & Neuroblastoma MYC-ON & R1c,R2c,R2s & 0.0042 & 0.046 & 10.86 \\
\bottomrule
\caption{Gene pairs reaching CANDIDATE tier (q$<$0.05, f$^2 \geq$0.35). R1c/s = lag-1 cos/sin terms, R2c/s = lag-2 cos/sin terms.}
\end{longtable}

\section{Supplementary Figures}

\subsection{Figure S1: Eigenperiod Distribution Across Tissues}

[Figure showing violin plots of eigenperiod distributions for each tissue, colored by healthy/cancer status]

\subsection{Figure S2: Stability Heatmap}

[Heatmap showing percentage of stable gene pairs across tissues and conditions]

\subsection{Figure S3: Permutation Null Distributions}

[Histograms showing null distributions of significant pair counts under each permutation model, with real values marked]

\subsection{Figure S4: CANDIDATE Pair Network}

[Network diagram showing Pparg at center connected to all 13 clock genes, with edge weights representing effect sizes]

\section{S2: Period Sensitivity Analysis}

\subsection{Rationale}

The PAR(2) eigenperiod metric is derived from AR(2) coefficients, which depend on clock gene phase estimates. Phase estimation uses cosinor regression with an assumed 24-hour period. A critical methodological question is whether the observed eigenperiod differences between healthy and cancer tissues could be an artifact of this 24-hour assumption---a form of circular inference where the output metric is imprinted from the input assumption.

\subsection{Methods}

We performed a period sensitivity analysis by varying the assumed circadian period $T \in \{20, 22, 24, 26, 28\}$ hours. For each T value, we:

\begin{enumerate}
    \item Re-estimated clock gene phases using cosinor regression with period T
    \item Re-ran the full PAR(2) analysis for all clock-target pairs
    \item Computed eigenperiods from the resulting AR(2) coefficients
    \item Compared healthy vs. cancer eigenperiod distributions using Welch's t-test
\end{enumerate}

Datasets analyzed: 6 healthy mouse tissues (GSE54650: Liver, Kidney, Heart, Lung, Muscle, Adrenal) and 2 neuroblastoma conditions (GSE221103: MYC-ON as cancer, MYC-OFF as control grouped with healthy).

\subsection{Results}

\begin{table}[H]
\centering
\caption{Table S5: Period sensitivity analysis results}
\label{tab:period_sens_supp}
\begin{tabular}{lccccccc}
\toprule
Period & Healthy & Healthy & Cancer & Cancer & $\Delta$ & Ratio & p-value \\
(T, h) & Mean & Median & Mean & Median & (h) & & \\
\midrule
20 & 12.9 & 9.9 & 23.9 & 22.0 & +11.0 & 1.85 & $<10^{-15}$ \\
22 & 12.3 & 9.4 & 22.3 & 21.0 & +10.0 & 1.82 & $<10^{-15}$ \\
24 & 12.9 & 9.3 & 22.7 & 20.8 & +9.8 & 1.77 & $<10^{-15}$ \\
26 & 12.6 & 9.3 & 23.5 & 20.8 & +10.8 & 1.86 & $<10^{-15}$ \\
28 & 12.5 & 9.2 & 24.8 & 20.8 & +12.3 & 1.98 & $<10^{-15}$ \\
\bottomrule
\end{tabular}
\end{table}

\subsection{Interpretation}

The eigenperiod separation between healthy and cancer tissues remained highly significant ($p < 10^{-15}$) across all tested period assumptions. The separation magnitude ($\Delta = 9.8$--$12.3$ hours) and ratio ($1.77$--$1.98\times$) showed no systematic dependence on T.

If eigenperiods were artifacts of the 24h assumption, we would expect one or more of:
\begin{itemize}
    \item Separation magnitude to scale proportionally with T
    \item Separation to disappear at non-24h T values
    \item Statistical significance to degrade at T $\neq$ 24h
\end{itemize}

None of these patterns was observed. The eigenperiod biomarker is therefore validated as reflecting genuine biological differences rather than methodological circular inference.

\section{S3: Wee1 Statistical Validation}

\subsection{Monte Carlo Simulation}

To assess whether Wee1's 13-clock gating pattern could arise by chance, we performed Monte Carlo simulation with 10,000 iterations. Under the null hypothesis:

\begin{itemize}
    \item Each clock-target pair has independent probability 0.037 of reaching 3+ tissue significance (based on observed rate: 11/299 pairs)
    \item For each simulated target gene, we independently sample whether each of 13 clocks shows significant gating
    \item We count how often any target gene achieves 13-clock coverage
\end{itemize}

\subsection{Results}

In 10,000 simulations, \textbf{zero cases} achieved 13-clock coverage for any of 23 target genes (empirical $p < 10^{-4}$; upper 95\% Clopper-Pearson bound $\approx 3 \times 10^{-4}$). The expected probability under independence is approximately $(0.037)^{13} \times 23 \approx 5.3 \times 10^{-18}$.

Among all tested target genes, Wee1 was the only gene achieving 13-clock coverage at the 3+ tissue threshold. The probability of exactly one gene achieving this by chance is negligible.

\subsection{Conclusion}

Wee1's exceptional hub status (gated by all 13 clock genes, each in 4--5 tissues) is statistically validated and extremely unlikely to arise by chance. This justifies Wee1 as the priority candidate for experimental validation of circadian-cell cycle coupling.

\section{Code Availability}

The PAR(2) Discovery Engine is implemented in TypeScript/Node.js with a React frontend. Source code is available at:

\begin{itemize}
    \item Web application: \url{https://par2-discovery-engine.replit.app}
    \item Analysis engine: \texttt{server/par2-engine.ts}
    \item Eigenvalue analysis: \texttt{scripts/eigenvalue-survey.ts}
    \item Permutation testing: \texttt{scripts/par2-null-survey.ts}
    \item Period sensitivity: \texttt{scripts/period-sensitivity.ts}
    \item Wee1 validation: \texttt{scripts/wee1-empirical-pvalue.ts}
\end{itemize}

\section{S1: Assessment of Fibonacci/Golden Ratio Dynamics}

Based on theoretical considerations relating AR(2) dynamics to Fibonacci sequences, we tested whether eigenperiod ratios in our data approximated the golden ratio ($\phi \approx 1.618$).

\subsection{Theoretical Background}

The Fibonacci sequence arises naturally from the recurrence relation $F_n = F_{n-1} + F_{n-2}$, which corresponds to an AR(2) process with coefficients $\beta_1 = \beta_2 = 1$. The ratio of successive Fibonacci numbers converges to $\phi = (1 + \sqrt{5})/2 \approx 1.618$. If circadian-cancer gene interactions followed Fibonacci-like dynamics, we would expect eigenperiod ratios to approximate $\phi$.

\subsection{Empirical Test}

We computed the ratio of eigenperiods for all gene pairs and tested whether the distribution peaked near $\phi$. We found that only \textbf{7.3\% of gene pairs} showed Fibonacci-like dynamics, defined as eigenperiod ratio within 10\% of $\phi$ (i.e., ratio between 1.456 and 1.780).

The observed ratio of healthy ($\sim$10h) to cancer ($\sim$23h) eigenperiods is approximately 2.3, which substantially exceeds $\phi$. The distribution of eigenperiod ratios across all gene pairs showed no significant peak near $\phi$ (Kolmogorov-Smirnov test vs. uniform: p = 0.82).

\subsection{Conclusion}

We conclude that Fibonacci/golden ratio dynamics represent a \textbf{theoretical limiting case} rather than an empirically prevalent phenomenon in circadian-cancer gene regulation. The AR(2) framework is valid for capturing phase-dependent dynamics, but the specific coefficient structure that produces Fibonacci behavior ($\beta_1 = \beta_2 = 1$) is not characteristic of circadian-cancer gene interactions.

This does not diminish the utility of the PAR(2) framework; rather, it clarifies that the eigenperiod metric captures biology-specific dynamics rather than universal mathematical constants.

\section{S8: Nine-Category Hierarchy Statistical Details}

\subsection{S8.1 Category Definitions and Gene Counts}

Nine functional categories were defined \textit{a priori} from canonical gene ontology annotations (Section 2.5 of the main text). Per-tissue gene--observation counts pooled across 12 GSE54650 tissues: Clock (152), Chromatin (185), Metabolic (193), Housekeeping (171), Immune (169), Signaling (228), DNA Repair (185), Target (195), Stem Cell (116). Total classified observations: $\sim$1,594.

\subsection{S8.2 Bootstrap and Permutation Validation}

\begin{itemize}
    \item \textbf{Permutation test:} 10,000 random label shuffles (preserving category sizes) yielded Kruskal-Wallis $H < 144.8$ in all iterations ($p < 0.001$).
    \item \textbf{Bootstrap stability:} 2,000 bootstrap iterations (resampling gene--tissue observations within each category): Clock ranked \#1 in 100\% of iterations; stem cell ranked \#9 in 94\% of iterations.
    \item \textbf{Detrending robustness:} After linear detrending, Spearman $\rho = 0.819$ between raw and detrended category rankings across 12 tissues.
    \item \textbf{Leave-one-tissue-out:} Top-ranked category was Clock in 12/12 folds (100\% match).
\end{itemize}

\subsection{S8.3 Pairwise Comparisons}

Of 36 pairwise Mann-Whitney $U$ tests: 9 reached nominal significance ($\alpha = 0.05$, uncorrected), 2 survived Bonferroni correction ($\alpha/36 = 0.0014$): Clock $>$ Target ($p = 0.0003$) and Clock $>$ Stem Cell ($p = 0.0005$).

\section{S9: Root-Space Geometry and Void Analysis}

\subsection{S9.1 Stationarity Triangle Construction}

The AR(2) stationarity triangle is defined by three boundary conditions on the characteristic equation $\lambda^2 - \beta_1\lambda - \beta_2 = 0$:
\begin{enumerate}
    \item $\beta_2 = 1 - \beta_1$ (unit root at $\lambda = 1$; self-reinforcing boundary)
    \item $\beta_2 = -1 - \beta_1$ (unit root at $\lambda = -1$; alternating boundary)
    \item $\beta_2 = -\beta_1^2/4$ (real-to-complex transition; parabolic boundary)
\end{enumerate}

All points inside this triangle yield $|\lambda| < 1$ (stable dynamics).

\subsection{S9.2 Dynamical Exclusion Zone (Void)}

A $20 \times 20$ density grid analysis across genome-wide AR(2) fits identifies a persistent exclusion zone in the ``barely oscillating'' transition region ($\beta_2 = 0.001$--$0.2$). Across all analyzed datasets:
\begin{itemize}
    \item $<$3\% of gene$\times$dataset entries fall in the transition zone
    \item 40.5\% cluster at the non-oscillatory pole ($\beta_2 < 0.001$)
    \item 55.2\% cluster above the void ($\beta_2 > 0.2$)
\end{itemize}

The void persists across species (mouse, human, baboon), tissues, and genome-wide scales ($\sim$20,000--58,000 genes per dataset), suggesting a universal constraint on gene expression dynamics.

\subsection{S9.3 Functional Enrichment at Dynamical Poles}

Permutation-based spatial clustering tests (within-category mean pairwise distance vs.\ 10,000 random subsets of equal size) confirm systematic clustering of GO Biological Process categories at specific root-space positions. Key findings:
\begin{itemize}
    \item Circadian rhythm genes: 19.5$\times$ enriched at the self-reinforcing pole
    \item Housekeeping functions cluster at center/memoryless pole
    \item Cell cycle genes preferentially occupy the oscillatory region
\end{itemize}

\section{S10: Drug Target Overlay Methodology}

\subsection{S10.1 Drug Target Database}

The drug target overlay maps 326 FDA-approved and investigational drug entries (168 unique gene targets, 16 drug classes) onto AR(2) root-space positions. Drug classes include: kinase inhibitors, PARP inhibitors, checkpoint inhibitors, mTOR inhibitors, CDK inhibitors, HDAC inhibitors, proteasome inhibitors, antimetabolites, hormone therapies, VEGF/angiogenesis inhibitors, DNA-damaging agents, epigenetic modifiers, Hedgehog pathway inhibitors, Wnt pathway modulators, apoptosis modulators, and immunomodulators.

\subsection{S10.2 Chronotherapy Candidate Identification}

Drug targets are classified as chronotherapy candidates if they satisfy:
\begin{enumerate}
    \item High eigenvalue modulus ($|\lambda| > 0.7$): strong temporal persistence suggesting timing sensitivity
    \item Significant circadian gating by $\geq$1 clock gene in $\geq$1 tissue
    \item Position in the oscillatory or self-reinforcing region of root-space
\end{enumerate}

\subsection{S10.3 Cross-Tissue Drug Target Coverage}

For each drug target gene, AR(2) eigenvalues are computed across all available tissues, providing tissue-specific chronotherapy recommendations. A target may be chronotherapy-amenable in one tissue but timing-insensitive in another, supporting the tissue-specific dosing optimization framework.

\section{S11: Cross-Species Validation Summary}

The three-layer eigenvalue hierarchy (Identity $>$ Proliferation $>$ Clock) was tested across multiple species to confirm cross-species conservation:

\begin{table}[H]
\centering
\caption{Table S10: Cross-species hierarchy validation}
\begin{tabular}{llccc}
\toprule
Species & Dataset & Tissues & Hierarchy Preserved? & Clock Percentile \\
\midrule
Mouse & GSE54650 & 12 & YES (12/12) & 85--96th \\
Mouse & GSE11923 & 1 (liver) & YES & 95th \\
Human & GSE113883 & 1 (blood) & Partial & 79th \\
Baboon & GSE98965 & 14 & 8/14 tissues (57\%) & 82nd (aggregate) \\
\bottomrule
\end{tabular}
\end{table}

The hierarchy is strongest in solid tissues and partially attenuated in blood, consistent with blood's unique circadian biology (lack of autonomous peripheral clock in most circulating cells). Note: the baboon multi-tissue atlas (GSE98965) shows hierarchy preservation in 8 of 14 tissues at the individual tissue level, indicating that while the aggregate (dataset-level) hierarchy is preserved, tissue-level variation is substantial.

\section{S5: Robustness Suite---Seven-Analysis Framework}

To systematically address concerns about the AR(2) eigenvalue hierarchy, we developed a comprehensive seven-analysis robustness suite. All analyses use deterministic seeded random number generation (seed=42) for exact reproducibility.

\subsection{S5.1 Sub-Sampling Recovery}

The full 48-timepoint GSE11923 mouse liver dataset was randomly sub-sampled to progressively smaller sizes (24, 12, 8, and 6 timepoints). At each sample size, AR(2) was re-fitted and the clock--target eigenvalue gap recomputed. The hierarchy was preserved down to $N=8$ timepoints with $>$90\% recovery rate, demonstrating that the signal does not require the full temporal resolution to be detected. At $N=6$, recovery degrades substantially, consistent with AR(2) requiring a minimum of $\sim$8 observations for reliable coefficient estimation.

\subsection{S5.2 Per-Gene Bootstrap Confidence Intervals}

Block bootstrap resampling (1,000 iterations) was applied to AR(2) residuals for each gene, preserving temporal autocorrelation structure. The resulting gap confidence interval was [0.058, 0.261], which does not include zero. This confirms that the hierarchy is not driven by outlier genes or sampling variability.

\subsection{S5.3 First-Difference Stationarity Defence}

Applying first-differencing ($\Delta x_t = x_t - x_{t-1}$) to all gene expression series before AR(2) fitting preserved the hierarchy in only 2/12 tissues (Adrenal and Brainstem). This is an honest limitation: first-differencing destroys not only trends but also the oscillatory autocorrelation that AR(2) is designed to capture. See S5.4 for the resolution.

\subsection{S5.4 Linear Detrending Defence}

\textbf{This analysis resolves the first-difference weakness.} Rather than differencing, we removed only linear trends by regressing each gene's expression on time index via OLS and analysing the residuals. Results:

\begin{table}[H]
\centering
\caption{Table S6: Linear detrending vs first-differencing comparison across 12 mouse tissues}
\label{tab:detrend_supp}
\begin{tabular}{lcccc}
\toprule
Tissue & Raw Gap & Detrended Gap & Differenced Gap & Detrend Preserved? \\
\midrule
Liver & +0.184 & +0.192 & $-$0.023 & YES \\
Kidney & +0.301 & +0.332 & $-$0.011 & YES \\
Heart & +0.263 & +0.309 & +0.012 & YES \\
Lung & +0.321 & +0.338 & $-$0.005 & YES \\
Adrenal & +0.387 & +0.412 & +0.031 & YES \\
Hypothalamus & +0.063 & +0.106 & +0.008 & YES \\
Cerebellum & +0.086 & +0.083 & $-$0.014 & YES \\
Brown Fat & +0.264 & +0.231 & $-$0.009 & YES \\
White Fat & +0.288 & +0.257 & $-$0.018 & YES \\
Muscle & +0.174 & +0.159 & $-$0.003 & YES \\
Aorta & +0.263 & +0.298 & +0.019 & YES \\
Brainstem & +0.162 & +0.181 & +0.025 & YES \\
\midrule
\textbf{Total} & --- & --- & --- & \textbf{12/12} \\
\bottomrule
\end{tabular}
\end{table}

\textbf{Interpretation:} The clock--target eigenvalue gap is preserved in 12/12 tissues after linear detrending, compared to only 2/12 after first-differencing. This proves the gap is driven by genuine oscillatory autocorrelation (which detrending preserves) rather than linear drift (which both methods remove). First-differencing over-corrects by destroying the very signal AR(2) is designed to measure.

\subsection{S5.5 Gap Permutation Test}

For each of 5 datasets, we randomly shuffled clock/target gene labels 10,000 times (seed=42) to generate a null distribution of the eigenvalue gap under random label assignment. All 5 datasets showed the observed gap was significantly larger than the null:

\begin{table}[H]
\centering
\caption{Table S7: Gap permutation test results (10,000 shuffles, seed=42)}
\label{tab:permgap_supp}
\begin{tabular}{lcccc}
\toprule
Dataset & Observed Gap & $p$-value & $z$-score & Genes (clock/target) \\
\midrule
Mouse Liver 48h (GSE11923) & +0.255 & $<0.001$ & 3.70 & 13 / 23 \\
Mouse Liver 24h (GSE54650) & +0.184 & $<0.001$ & 3.47 & 13 / 22 \\
Mouse Kidney (GSE54650) & +0.301 & $<0.001$ & 4.33 & 13 / 22 \\
Mouse Heart (GSE54650) & +0.263 & $<0.001$ & 3.87 & 13 / 22 \\
Mouse Lung (GSE54650) & +0.321 & $<0.001$ & 4.29 & 13 / 22 \\
\bottomrule
\end{tabular}
\end{table}

\textbf{Interpretation:} The observed clock--target eigenvalue hierarchy is extremely unlikely under random gene label assignment ($p < 0.001$ across all 5 datasets, $z = 3.47$--$4.33$). This provides strong evidence that the hierarchy reflects genuine biological differences between clock and target genes, not an artifact of gene selection.

\subsection{S5.6 Leave-One-Tissue-Out Cross-Validation}

Each of the 12 mouse tissues from the GSE54650 dataset was held out in turn. For each fold, the clock--target eigenvalue hierarchy was computed from the remaining 11 tissues, and the held-out tissue was independently tested for the same pattern. All 12 tissues independently confirmed the hierarchy predicted by the training set (12/12 prediction accuracy). This demonstrates that no single tissue drives the cross-tissue pattern---the hierarchy is a genuinely pan-tissue phenomenon that generalizes to unseen tissues.

\subsection{S5.7 Population-Level Cross-Validation}

Five-fold cross-validation was applied across 5 datasets (25 total folds). In each fold, 80\% of genes were used for AR(2) fitting and the remaining 20\% were held out. The clock--target hierarchy was preserved in 25/25 folds (100\%), with mean gap $0.216 \pm 0.051$. This confirms that the hierarchy is stable across population subsets and not driven by a small number of influential genes.

\subsection{S5 Summary}

\begin{table}[H]
\centering
\caption{Table S8: Robustness suite summary (7 analyses + n/p validation)}
\label{tab:robustness_summary}
\begin{tabular}{llcc}
\toprule
Analysis & What It Tests & Result & Verdict \\
\midrule
Sub-sampling & Temporal resolution sensitivity & Robust to $N=8$ & PASS \\
Bootstrap CIs & Outlier gene influence & Gap CI [0.058, 0.261] excl.\ 0 & PASS \\
First-differencing & Stationarity (aggressive) & 2/12 preserved & LIMITATION \\
Linear detrending & Stationarity (appropriate) & 12/12 preserved & PASS \\
Permutation test & Gene label specificity & $p < 0.001$, $z = 3.47$--$4.33$ & PASS \\
Leave-one-tissue-out & Cross-tissue generalization & 12/12 tissues confirm hierarchy & PASS \\
Population CV & Cross-sample stability & 25/25 folds (100\%) & PASS \\
\midrule
n/p Validation & Sampling adequacy & $r=0.786$, same hierarchy & PASS \\
\bottomrule
\end{tabular}
\end{table}

\section{S6: High-Resolution n/p Ratio Validation}

A key concern is the low observations-per-parameter ratio ($n/p \approx 7.3$) in GSE54650 datasets (24 timepoints, 3 AR(2) parameters). We validated the three-layer hierarchy using GSE11923 (Hughes et al. 2009), which provides 48 hourly timepoints ($n/p = 15.3$), well above the conventional threshold of 10.

\subsection{S6.1 Cross-Dataset Hierarchy Comparison}

AR(2) eigenvalue moduli were computed for identity, clock, and proliferation gene panels in both datasets:

\begin{table}[H]
\centering
\caption{Table S9: Cross-dataset n/p validation---three-layer hierarchy comparison}
\label{tab:np_validation_supp}
\begin{tabular}{lccccc}
\toprule
Dataset & $n/p$ & Identity $|\lambda|$ & Prolif. $|\lambda|$ & Clock $|\lambda|$ & Hierarchy \\
\midrule
GSE11923 (48 tp, 1h) & 15.3 & 0.984 & 0.982 & 0.938 & I $>$ P $>$ C \\
GSE54650 (24 tp, 2h) & 7.3 & 0.994 & 0.975 & 0.858 & I $>$ P $>$ C \\
\bottomrule
\end{tabular}
\end{table}

Both datasets produce the identical hierarchy ordering (Identity $>$ Proliferation $>$ Clock).

\subsection{S6.2 Gene-Level Cross-Dataset Correlation}

For genes present in both datasets, Pearson correlation of AR(2) eigenvalues across sampling resolutions was $r = 0.786$, indicating strong agreement at the individual gene level despite different temporal resolution and $n/p$ ratios.

\subsection{S6.3 Permutation and Bootstrap Validation}

A permutation test (10,000 label shuffles) confirmed the Identity--Clock gap is significant at $p = 0.0032$. Bootstrap confidence intervals (5,000 iterations) for the Identity--Clock gap were [0.016, 0.080] and for the Clock--Proliferation gap were [$-$0.077, $-$0.014], neither crossing zero.

\textbf{Conclusion:} The three-layer hierarchy is robust to the $n/p$ limitation of GSE54650 and is independently replicated at adequate statistical power ($n/p = 15.3$) in GSE11923.

\section{S7: PCA Comparison Methodology}

To empirically demonstrate that AR(2) root-space captures information orthogonal to variance-based dimensionality reduction, we implemented a side-by-side PCA comparison in the interactive platform.

\subsection{S7.1 PCA Computation}

For each dataset, PCA was computed on the standardized gene expression matrix as follows:

\begin{enumerate}
    \item Each gene's expression time series was standardized to zero mean and unit variance: $z_{ij} = (x_{ij} - \bar{x}_i) / s_i$, where $i$ indexes genes and $j$ indexes timepoints.
    \item The $p \times p$ timepoint covariance matrix was computed: $\mathbf{C} = \mathbf{Z}^T\mathbf{Z} / (n-1)$, where $n$ is the number of genes and $p$ is the number of timepoints.
    \item The first two eigenvectors ($\mathbf{v}_1$, $\mathbf{v}_2$) were extracted via power iteration (300 iterations with convergence threshold $10^{-12}$), with deflation and orthogonalization for the second component.
    \item Gene scores were computed by projection: $\text{PC1}_i = \mathbf{z}_i \cdot \mathbf{v}_1$ and $\text{PC2}_i = \mathbf{z}_i \cdot \mathbf{v}_2$.
    \item Variance explained was calculated as $\lambda_k / \text{tr}(\mathbf{C})$ for each component.
\end{enumerate}

\noindent The $p \times p$ covariance approach (rather than $n \times n$) is both computationally efficient (since $p \ll n$ for typical expression datasets) and mathematically standard for gene-level PCA of expression matrices.

\subsection{S7.2 Cross-Highlighting Design}

The interactive visualization displays the same genes in both PCA projection space and AR(2) root-space, with linked hover highlighting. When a user hovers over a gene in either panel, the corresponding point is highlighted in the other panel, along with a tooltip displaying gene name, functional category, eigenvalue modulus $|\lambda|$, PCA coordinates (PC1, PC2), and AR(2) coefficients ($\beta_1$, $\beta_2$). This enables direct visual assessment of whether a gene's position in variance-space predicts its position in dynamics-space.

\subsection{S7.3 Representative Results}

For the GSE54650 liver dataset (167 genes in 9 functional categories):
\begin{itemize}
    \item PC1 captured 30.1\% and PC2 captured 16.9\% of total expression variance.
    \item Gene categories that cluster tightly in root-space (e.g., clock genes near the oscillatory regime) are dispersed in PCA space, and vice versa.
    \item The dynamical exclusion zone (void) visible in root-space has no analogue in PCA projection, where genes distribute relatively continuously.
\end{itemize}

\noindent These observations confirm that $|\lambda|$ and PCA capture fundamentally different properties of gene expression time series: temporal persistence versus expression variance.

\section{S8: Interactive Platform Features}

The PAR(2) Discovery Engine (\url{https://par2-discovery-engine.replit.app}) provides the following interactive analysis tools:

\subsection{S8.1 Progressive Hierarchy Layer Toggle}

A progressive-reveal interface allows users to explore the nine-category persistence hierarchy interactively. Nine gene categories (Clock, Target, Housekeeping, Immune, Metabolic, Chromatin, Signaling, DNA Repair, Stem Cell) are ranked by mean eigenvalue persistence. Users can toggle categories on/off individually or use a slider to reveal layers progressively from highest to lowest persistence, watching the hierarchy emerge in root-space.

\subsection{S8.2 Data-Driven Void Annotation}

The dynamical exclusion zone (void) is identified automatically from gene density data using a $20 \times 20$ grid analysis across the stationarity triangle. The algorithm locates the sparsest region inside the triangle and annotates it with a label. This annotation updates dynamically when gene category filters are toggled, confirming that the void persists across different subsets of genes and is not an artifact of any particular gene selection.

\subsection{S8.3 Root-Space Framework Overlays}

Four interpretive overlays are provided:
\begin{itemize}
    \item \textbf{Waddington Landscape:} 3D gene density surface rendered as terrain, with valleys representing populated dynamical regions and ridges representing the void.
    \item \textbf{Phase Portrait:} Classical dynamical regime classification (self-reinforcing, alternating, oscillatory) with the parabola separating overdamped from underdamped dynamics.
    \item \textbf{Functional Geography:} Gene categories plotted with dynamical pole annotations and category centroids.
    \item \textbf{PCA Comparison:} Side-by-side visualization of genes in PCA projection space versus AR(2) root-space with cross-highlighting (see S7).
\end{itemize}

\section{Reproducibility Statement}

All analyses can be reproduced using the embedded datasets within the PAR(2) Discovery Engine. The application provides:

\begin{itemize}
    \item 72 pre-loaded datasets across 4 species (mouse, human, baboon, Arabidopsis) spanning 28 GEO studies
    \item Genome-wide gene search ($>$20,000 genes per dataset)
    \item Automated PAR(2) analysis pipeline
    \item FDR correction and tier classification
    \item Eigenvalue/eigenperiod computation
    \item Permutation validation framework
    \item Seven-analysis robustness suite with deterministic seeded RNG (seed=42)
    \item Root-space geometry with PCA comparison, hierarchy toggle, and framework overlays
    \item Genome-wide disease screen with matched healthy/disease pair analysis
    \item Interactive visualization and export
\end{itemize}

\end{document}